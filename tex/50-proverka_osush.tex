\chapter{Оценка технической осуществимости вариантов схемы сети}
\label{cha:proverka_osush}

Под технической осуществимостью варианта схемы сети понимается, прежде всего, возможность обеспечения требуемых уровней напряжения на шинах 10 кВ в соответствии с принципом встречного регулирования напряжения у наиболее электрически удаленных от источника питания подстанций (в пределах электрической сети одного номинального напряжения).

Поскольку в современных электрических сетях 35-220 кВ практически на всех вновь сооружаемых подстанциях устанавливаются трансформаторы с устройством РПН, которое компенсирует падение напряжения за счет соответствующего выбора ответвления, то для этих сетей отсутствует нормирование потерь напряжения.

Вместе с тем для предварительной технической оценки вариантов схем и номинальных напряжений сети можно рекомендовать ориентироваться на значения допустимых потерь напряжения в сети в режиме наибольших нагрузок (нормальных и послеаварийных) до 15-17 \% \cite{глазунов_шведов}. Эти значения определены с учетом уровня рабочего напряжения на шинах источника питания, минимально возможного коэффициента трансформации трансформаторов и потерь напряжения в трансформаторах подстанций (потери напряжения в двух параллельно работающих трансформаторах при их загрузке на 70 \% от номинальной мощности не превышают 6 \% \cite{глазунов_шведов}).

Таким образом, оценка технической осуществимости варианта схемы сети сводится к проверке в послеаварийных режимах не превышения суммарных потерь напряжения в линиях электропередачи от источника питания до наиболее электрически удаленной подстанции (в пределах сети одного номинальго напряжения) допустимых значений.

Оценку потерь напряжения на этой стадии проектирования допустимо выполнять на основе приближенного потокораспределения (без учета потерь мощности в линиях по приведенным нагрузкам подстанции) путем последовательного определения напряжений в узлах по заданному напряжению на шинах источника питаня. В замкнутых сетях одного номинального напряжения допускается определять потокораспределение по длинам линий. Потери напряжения определяются по активным и реактивным сопротивлениям выбранных сечений проводов.

Если суммарные потери напряжения превосходят допустимые значения, то считается, что в данном варианте схемы сети нельзя обеспечить требуемых уровней напряжения на шинах 10 кВ подстанций в соответствии с принципом встречного регулирования, и такой вариант исключается из дальнейшего рассмотрения как технически не осуществимый.

\section{Оценка технической осуществимости схемы с кольцевой сетью}

Поскольку сечения и марки проводов вл теперь известны, можно рассчитать активные и реактивные параметры схемы замещения для всех линий в сети по формулам:
\begin{eqndesc}[H]
	\begin{equation}
		R = \frac{R_0\cdot L}{n_\textup{ц}}; \; X = \frac{X_0\cdot L}{n_\textup{ц}}; \; \frac{B}{2} = \frac{n_\textup{ц}\cdot B_0\cdot L}{2},
		\label{eqn:параметры_линии}
	\end{equation}
где $R_{\text{0}}$ и $X_{\text{0}}$ "--- удельные активные и реактивные сопротивления в линии, Ом/км; $B_0$ "--- удельная емкостная проводимость, См/км
\end{eqndesc}

В качестве примера рассмотрим одноцепную линию К1 с проводами марки АС 400/51 с длиной 41,8 км. По табл. 3.9 \cite{файбисович} для ВЛ 220 кВ определяем значения удельных параметров для провода АС 400/51:
\[R_0 = 0,073\; \frac{\textup{Ом}}{\textup{км}}; \; X_0 = 0,42\; \frac{\textup{Ом}}{\textup{км}}; \; B_0 = 2,701\cdot 10^{-6} \; \frac{\textup{См}}{\textup{км}}\]

Тогда рассчитаем рассчитаем по формуле \eqref{eqn:параметры_линии} активные и реактивные параметры схемы:
\[R_{K1} = \frac{R_{0(K1)}\cdot L_{K1}}{n_{\textup{ц(К1)}}} = \frac{0,073\cdot 41,8}{1} = 3,05\; \textup{Ом}\]
\[X_{K1} = \frac{X_{0(K1)} \cdot L_{K1}}{n_{\textup{ц(К1)}}} = \frac{0,42\cdot 41,8}{1} = 17,6\; \textup{Ом}\]
\[\frac{B_{K1}}{2} = \frac{n_{\textup{ц(К1)}}\cdot B_{0(K1)}\cdot L_{K1}}{2} = \frac{1\cdot 2,701\cdot 10^{-6} \cdot 41,8}{2} = 56,5\; \textup{См}\]

Сведем в табл. \ref{tab:резы_расчета_параметров_линии} результаты расчета параметров ВЛ схемы сети.

\begin{table}[H]
	\small
	\caption{Результаты расчета параметров ВЛ}
	\label{tab:резы_расчета_параметров_линии}
	\begin{tabularx}{\textwidth}{|Z|Z|Z|Z|Z|Z|Z|}
		\hline
		Линия                                                 & K1        & K2        & 13        & 23        & 34        & 45        \\ \hline
		Марка провода                                         & АС 400/51 & АС 400/51 & АС 240/32 & АС 240/32 & АС 240/32 & АС 120/19 \\ \hline
		\(U_\textup{ном}\), кВ                                & 220       & 220       & 220       & 220       & 110       & 110       \\ \hline
		\(n_\textup{ц}\)                                      & 1         & 1         & 1         & 1         & 2         & 2         \\ \hline
		L, км                                                 & 41,8      & 49,2      & 49,2      & 41,8      & 25,9      & 25,9      \\ \hline
		\(R_0, \frac{\textup{Ом}}{\textup{км}}\)              & 0,073     & 0,073     & 0,118     & 0,118     & 0,118     & 0,244     \\ \hline
		\(X_0, \frac{\textup{Ом}}{\textup{км}}\)              & 0,42      & 0,42      & 0,435     & 0,435     & 0,405     & 0,427     \\ \hline
		\(B_0\cdot 10^{-6}, \frac{\textup{См}}{\textup{км}}\) & 2,701     & 2,701     & 2,604     & 2,604     & 2,808     & 2,658     \\ \hline
		\(R\), Ом                                             & 3,05      & 3,59      & 5,81      & 4,93      & 1,53      & 3,16      \\ \hline
		\(X\), Ом                                             & 17,6      & 20,7      & 21,4      & 18,2      & 5,24      & 5,53      \\ \hline
		\(\frac{B}{2}\cdot 10^{-6}\), См                      & 56,5      & 66,4      & 64,1      & 54,4      & 72,7      & 68,8      \\ \hline
	\end{tabularx}
\end{table}

\subsection{Сеть 220 кВ}

Как следует из табл. \ref{tab:эконом_сечение_результаты} наиболее нагруженным участком сети является участок К1 (\(S_{K1} = 134,9\; \textup{МВА} > S_{K2} = 129,0\; \textup{МВА}\)). Поэтому самым тяжелым аварийным режимом в кольцевой сети 220 кВ является отключение линии К1. В этом случае наиболее электрически удаленной точкой будет пункт потребления 1. Таким образом, нам необходимо последовательно определить напряжения до ПС3 и оценивать потери электроэнергии. Рассчитаем потокораспределение активной и реактивной мощностей, передаваемых по линиям сети 220 кВ в самом тяжелом послеаварийном режиме (при отключении ВЛ К1).
\[P_{K2} = P_\textup{нб1} + P_\textup{нб3}^{'} + P_\textup{нб2} = 70 + 105 + 70 = 245\; \textup{МВт}\]
\[Q_{K2} = Q_\textup{прив1} + Q_\textup{прив2} + Q_\textup{прив3} = 26 + 26 + 46,1 = 98,1\; \textup{МВар}\]
\[P_{23} = P_\textup{нб1} + P_\textup{нб3}^{'} = 70 + 105 = 175\; \textup{МВт}\]
\[Q_{23} = Q_\textup{прив1} + Q_\textup{прив3} = 26 + 46,1 = 72,1\; \textup{МВар}\]
\[P_{13} = P_\textup{нб1} = 70\; \textup{МВт}\]
\[Q_{K2} = Q_\textup{прив1} = 26\; \textup{МВар}\]

Выполним оценку потерь напряжения на участке от шин ИП К до пункта 1 путем последовательного вычисления напряжений в узлах сети 220 кВ по заданному напряжению \(U_K = U_\textup{ном} \cdot 1,1 = 220 \cdot 1,1 = 242\) кВ.

\[\Delta U_{K2} = \frac{P_{K2}\cdot R_{K2} + Q_{K2} \cdot X_{K2}}{U_{K}} = \frac{245\cdot 3,59 + 98,1\cdot 20,7}{242} = 12,0\; \textup{кВ}\]
\[\delta U_{K2} = \frac{P_{K2}\cdot X_{K2} - Q_{K2} \cdot R_{K2}}{U_{K}} = \frac{245\cdot 20,7 - 98,1\cdot 3,59}{242} = 19,5\; \textup{кВ}\]
\[U_2 = \sqrt{(U_K - \Delta U_{K2})^2 + (\delta U_{K2})^2} = \sqrt{(242 - 12,0)^2 + 19,5^2} = 230,8\; \textup{кВ}\]

\[\Delta U_{23} = \frac{P_{23}\cdot R_{23} + Q_{23} \cdot X_{23}}{U_{2}} = \frac{175\cdot 4,93 + 72,1\cdot 18,2}{230,8} = 9,42\; \textup{кВ}\]
\[\delta U_{23} = \frac{P_{23}\cdot X_{23} - Q_{23} \cdot R_{23}}{U_{2}} = \frac{175\cdot 18,2 - 72,1\cdot 4,93}{230,8} = 12,3\; \textup{кВ}\]
\[U_3 = \sqrt{(U_2 - \Delta U_{23})^2 + (\delta U_{23})^2} = \sqrt{(230,8 - 9,42)^2 + 12,3^2} = 221,7\; \textup{кВ}\]


\[\Delta U_{13} = \frac{P_{13}\cdot R_{13} + Q_{13} \cdot X_{13}}{U_{3}} = \frac{70\cdot 5,81 + 26\cdot 21,4}{221,7} = 4,34\; \textup{кВ}\]
\[\delta U_{13} = \frac{P_{13}\cdot X_{13} - Q_{13} \cdot 4_{13}}{U_{3}} = \frac{70\cdot 21,4 - 26\cdot 5,81}{221,7} = 6,08\; \textup{кВ}\]
\[U_1 = \sqrt{(U_3 - \Delta U_{13})^2 + (\delta U_{13})^2} = \sqrt{(221,7 - 4,34)^2 + 6,08^2} = 217,4\; \textup{кВ}\]

Суммарные потери напряжения на участке от шин ИП К до пункта 1:
\[\Delta U_{K-2-3-1,\%} = \frac{U_K - U_1}{U_\textup{ном}}\cdot 100 \% = \frac{242 - 217,4}{220}\cdot 100\% = 11,2 \%\]

Вывод: в результате получилось, что в сети 220 кВ суммарные потери напряжения в самом тяжелом послеаварийном режиме не выходят за пределы допустимых значений \cite{глазунов_шведов}.

\subsection{Сеть 110 кВ}

В сети 110 кВ наиболее тяжелым послеаварийным режимом будет отключение головного участка 3-4. В этом случае наиболее электрически удаленным является пункт потребления 5.

Рассчитаем потокораспределение активном и реактивной мощностей передаваемой по ВЛ 110 кВ в самом тяжелом послеаварийном режиме (при отключении одной цепи 34):
\[P_{34} = P_\textup{нб4} + P_\textup{нб5} = 40 + 35 = 75\; \textup{МВт}\]
\[Q_{34} = Q_\textup{прив4} + Q_\textup{прив5} = 14,2 + 14,1 = 28,3\; \textup{МВар}\]
\[P_{45} = P_\textup{нб5} = 35\; \textup{МВт}\]
\[Q_{45} = Q_\textup{прив5} = 14,1\; \textup{МВар}\]

Примем, что на шинах ПС3 в режиме НБ поддерживается напряжение \(1,1\cdot U_\textup{ном}\)

\[U_3^\textup{с} = 1,1 \cdot U_\textup{ном} = 1,1 \cdot 110 = 121\; \textup{кВ}\]

Выполним оценку потерь напряжения на участке от шин СН ПС3 до пункта 5:

\[\Delta U_{34} = \frac{P_{34}\cdot 2\cdot R_{34} + Q_{34}\cdot 2\cdot X_{34}}{U_3^\textup{с}} = \frac{75\cdot 2\cdot 1,53 + 28,3\cdot 2\cdot 5,24}{121} = 4,35\; \textup{кВ}\]

Поперечной составляющей потерь напряжения в сети 110 кВ пренебрегаем. Таким образом напряжение в точке 4:
\[U_4 = U_3^\textup{с} - \Delta U_{34} = 121 - 4,35 = 115,1\; \textup{кВ}\]
\[\Delta U_{45} = \frac{P_{45}\cdot R_{45} + Q_{45}\cdot X_{45}}{U_4} = \frac{35\cdot 3,16 + 14,1\cdot 5,5}{115,1} = 1,61\; \textup{кВ}\]
\[U_5 = U_4 - \Delta U_{45} = 116,7 - 1,61 = 115,1\; \textup{кВ}\]

Суммарные потери напряжения на участке шин ПС3, являющегося источником питания для сети 110 кВ, до пункта 5:

\[\Delta U_{3-4-5, \%} = \frac{U_3^\textup{с} - U_5}{U_\textup{ном}} \cdot 100\% = \frac{121 - 115,1}{110} \cdot 100\% = 5,36 \%\]

В сети 110 кВ суммарные потери напряжения в самом тяжелом послеаварийном режиме получились меньше диапазона допустимых значений \cite{глазунов_шведов}.

Вывод: рассматриваемая схема сети технически осуществима.

\section{Оценка технической осуществимости варианта схемы сети 2}

Рассчитаем по формулам \eqref{eqn:параметры_линии} параметры линии и сведем результаты расчетов в табл. \ref{tab:резы_расчета_параметров_линии_магистраль}.

\begin{table}[H]
	\small
	\caption{Результаты расчета параметров ВЛ}
	\label{tab:резы_расчета_параметров_линии_магистраль}
	\begin{tabularx}{\textwidth}{|Z|Z|Z|Z|Z|Z|}
		\hline
		Линия                                                 & K1        & 13        & 23        & 34        & 45        \\ \hline
		Марка провода                                         & АС 400/51 & АС 300/39 & АС 240/32 & АС 240/32 & АС 120/19 \\ \hline
		\(U_\textup{ном}\), кВ                                & 220       & 220       & 110       & 110       & 110       \\ \hline
		\(n_\textup{ц}\)                                      & 2         & 2         & 2         & 2         & 2         \\ \hline
		L, км                                                 & 41,8      & 49,2      & 41,8      & 25,9      & 25,9      \\ \hline
		\(R_0, \frac{\textup{Ом}}{\textup{км}}\)              & 0,073     & 0,096     & 0,118     & 0,118     & 0,244     \\ \hline
		\(X_0, \frac{\textup{Ом}}{\textup{км}}\)              & 0,42      & 0,429     & 0,405     & 0,405     & 0,427     \\ \hline
		\(B_0\cdot 10^{-6}, \frac{\textup{См}}{\textup{км}}\) & 2,701     & 2,645     & 2,808     & 2,808     & 2,658     \\ \hline
		\(R\), Ом                                             & 1,53      & 2,36      & 2,47      & 1,53      & 3,16      \\ \hline
		\(X\), Ом                                             & 8,78      & 10,6      & 8,46      & 5,24      & 5,53      \\ \hline
		\(\frac{B}{2}\cdot 10^{-6}\), См                      & 112,9     & 130,1     & 117,4     & 72,7      & 68,8      \\ \hline
	\end{tabularx}
\end{table}

\subsection{Сеть 220 кВ}

Как следует из табл. \ref{tab:эконом_сечение_результаты_магистраль} наиболее нагруженным участком сети является участок К1. Поэтому самым тяжелым аварийным режимом в радиально-магистральной сети 220 кВ является отключение одной цепи линии К1. В этом случае наиболее электрически удаленной точкой будет пункт потребления 3. Таким образом, нам необходимо последовательно определить напряжения до ПС3 и оценивать потери электроэнергии. Сошлемся на потокораспределение активной одной цепи в п. 2 данного курсового проекта, но при этом умножим ее на 2 в каждой линии, чтобы получить потокораспределения активной мощности для двухцепных линий, а реактивную мощность возьмем из п. 4.1.2. Итоговое потокораспределение в линиях сведем в табл. \ref{tab:потокораспределение_магистраль_п/ав}

\begin{table}[H]
	\small
	\caption{Потокораспределение в радиально-магистральной схеме сети}
	\label{tab:потокораспределение_магистраль_п/ав}
	\begin{tabularx}{\linewidth}{|Z|Z|Z|Z|Z|Z|}
		\hline
		Линия       & К1    & 13    & 23   & 34   & 45   \\ \hline
		\(P\), МВт  & 245   & 175   & 70   & 75   & 35   \\ \hline
		\(Q\), МВар & 94,3  & 70,7  & 23,6 & 23,5 & 11,7 \\ \hline
	\end{tabularx}
\end{table}

Выполним оценку потерь напряжения на участке от шин источника питания К до пункта 3:
\[\Delta U_{K1} = \frac{P_{K1} \cdot 2\cdot R_{K1} + Q_{K1}\cdot 2\cdot X_{K1}}{U_K} = \frac{245\cdot 2\cdot 1,53 + 94,3\cdot 2\cdot 8,78}{242} = 9,94\; \textup{кВ}\]

\[\delta U_{K1} = \frac{P_{K1} \cdot 2\cdot X_{K1} - Q_{K1}\cdot 2\cdot R_{K1}}{U_K} = \frac{245\cdot 2\cdot 8,78 - 94,3\cdot 2\cdot 1,53}{242} = 16,6\; \textup{кВ}\]

\[U_1 = \sqrt{(U_K - \Delta U_{K1})^2 + (\delta U_{K1})^2} = \sqrt{(242 - 9,94)^2 + (16,6)^2} = 235,4\; \textup{кВ}\]

\[\Delta U_{13} = \frac{P_{13} \cdot R_{13} + Q_{13}\cdot X_{13}}{U_1} = \frac{175\cdot 2,36 + 70,7\cdot 10,6}{232,7} = 5,00\; \textup{кВ}\]

\[\delta U_{13} = \frac{P_{13} \cdot X_{13} - Q_{13}\cdot R_{13}}{U_1} = \frac{175\cdot 10,6 - 70,7\cdot 2,36}{232,7} = 7,25\; \textup{кВ}\]

\[U_3 = \sqrt{(U_1 - \Delta U_{13})^2 + (\delta U_{13})^2} = \sqrt{(235,4 - 5,00)^2 + (7,25^2} = 227,8\; \textup{кВ}\]

Суммарные потери напряжения на участке от шин ИП К до пункта 1:
\[\Delta U_{K-1-3,\%} = \frac{U_K - U_3}{U_\textup{ном}}\cdot 100 \% = \frac{242 - 227,8}{220}\cdot 100\% = 6,45 \%\]

Вывод: в результате получилось, что в сети 220 кВ суммарные потери напряжения в самом тяжелом послеаварийном режиме не выходят за пределы допустимых значений \cite{глазунов_шведов}.

\subsection{Сеть 110 кВ}

В сети 110 кВ наиболее тяжелым послеаварийным режимом будет отключение головного участка 3-4. В этом случае наиболее электрически удаленным является пункт потребления 5.

Примем, что на шинах ПС3 в режиме НБ поддерживается напряжение \(1,1\cdot U_\textup{ном}\)

\[U_3^\textup{с} = 1,1 \cdot U_\textup{ном} = 1,1 \cdot 110 = 121\; \textup{кВ}\]

Выполним оценку потерь напряжения на участке от шин СН ПС3 до пункта 5:

\[\Delta U_{34} = \frac{P_{34}\cdot 2\cdot R_{34} + Q_{34}\cdot 2\cdot X_{34}}{U_3^\textup{с}} = \frac{75\cdot 2\cdot 1,53 + 23,5\cdot 2\cdot 5,24}{121} = 3,93\; \textup{кВ}\]

Поперечной составляющей потерь напряжения в сети 110 кВ пренебрегаем. Таким образом напряжение в точке 4:
\[U_4 = U_3^\textup{с} - \Delta U_{34} = 121 - 3,93 = 117,1\; \textup{кВ}\]
\[\Delta U_{45} = \frac{P_{45}\cdot R_{45} + Q_{45}\cdot X_{45}}{U_4} = \frac{35\cdot 3,16 + 11,7\cdot 5,53}{117,1} = 1,50\; \textup{кВ}\]
\[U_5 = U_4 - \Delta U_{45} = 117,1 - 1,50 = 115,6\; \textup{кВ}\]

Суммарные потери напряжения на участке шин ПС3, являющегося источником питания для сети 110 кВ, до пункта 5:

\[\Delta U_{3-4-5, \%} = \frac{U_3^\textup{с} - U_5}{U_\textup{ном}} \cdot 100\% = \frac{121 - 115,6}{110} \cdot 100\% = 4,91 \%\]

В сети 110 кВ суммарные потери напряжения в самом тяжелом послеаварийном режиме получились меньше диапазона допустимых значений \cite{глазунов_шведов}.

Вывод: рассматриваемая схема сети технически осуществима.


%%% Local Variables:
%%% mode: latex
%%% TeX-master: "rpz"
%%% End: