\chapter{Выбор сечений проводов линий электропередачи и их проверка по условиям технических ограничений}
\label{cha:sech_provod}

В справочнике под редакцией Д.Л. Файбисовича \cite{файбисович} указывается, что технико-экономические расчеты по выбору сечения проводов каждой конкретной линии выполняется только для ВЛ 750 кВ и передач постоянного тока. Для ВЛ до 500 кВ включительно выбор сечения проводов производится по нормированным обощенным показателям. В качестве таких показателей используется нормированные в ПУЭ \cite{пуэ7} значения экономической плотности тока (см. табл. \ref{tab:j_ek_norm}).

\begin{table}[H]
	\small
	\caption{Нормированные значения плотности тока}
	\begin{tabularx}{\linewidth}{|Z|Z|Z|Z|}
		\hline
		\multirow{2}{*}{Проводники} & \multicolumn{3}{c|}{\(j_\textup{эк}\), \(\frac{\textup{А}}{\textup{мм}}^2\) при \(T_\textup{нб}\), \(\frac{\textup{ч}}{\textup{год}}\)} \\ \cline{2-4}
		                            & от 1000 до 3000 & от 3000 до 5000 & более 5000                                                                                          \\ \hline
		\multicolumn{4}{|c|}{Неизолированные провода и шины}                                                                                                                  \\ \hline
		Медные                      & 2,0             & 1,7             & 1,4                                                                                                 \\ \hline
		Алюминиевые                 & 1,0             & 0,9             & 0,8                                                                                                 \\ \hline
	\end{tabularx}
	\label{tab:j_ek_norm}
\end{table}

\section{Выбор экономически целесообразных сечений проводов}

Выбор сечений проводов ВЛ до 500 кВ выполняется по методу экономической плотности тока, согласно которому экономически целесообразное сечение токопроводящей (алюминиевой) части провода:
\begin{eqndesc}[H]
	\begin{equation}
		\label{eqn:ekonom_sech}
		F_\textup{эк} = \frac{I_\textup{р}}{j_\textup{эк}},
	\end{equation}
где \(I_\textup{р}\) "--- расчетный ток одной цепи ВЛ, А; \(j_\textup{эк}\) "--- нормированное значение экономической плотности тока, \(\frac{\textup{А}}{\textup{мм}}^2\) (см. табл. \ref{tab:j_ek_norm}).
\end{eqndesc}

Значение расчетного тока одной цепи линии:
\begin{eqndesc}[H]
	\begin{equation}
		I_\textup{р} = \alpha_i \cdot T_\textup{нб(5)},
		\label{eqn:расч.ток_1цепи}
	\end{equation}
где \(I_\textup{нб(5)}\) "--- наибольший ток одной цепи ВЛ на пятый год эксплуатации, А; \(\alpha_i\) "--- коэффициент, учитывающий рост нагрузки по годам эксплуатации линии за расчетный период (\(T_\textup{р}\) = 10 лет).
\end{eqndesc}



Согласно \cite{глазунов_шведов} для ВЛ 110-220 кВ значение коэффициента \(\alpha_i = 1,05\), что соответствует наиболее часто встречающимся темпам роста нагрузки. Значение наибольшего тока одной цепи линии на пятый год эксплуатации вычисляется по полной мощности \textit{S}, передаваемой по линии в режиме наибольших нагрузок и определяется по приведенным к шинам ВН нагрузкам ПС:
\begin{equation}
	I_\textup{нб(5)} = \frac{S}{\sqrt{3}\cdot U_\textup{ном}\cdot n_\textup{ц}}
	\label{eqn:наиб_ток_1цепи}
\end{equation}

Полученное по формуле (\ref{eqn:ekonom_sech}) экономически целесообразное сечение округляется до ближайшего стандартного. Для ВЛ 110 кВ: \(F_\textup{а} = 70 \div 240\; \textup{мм}^2\). Для ВЛ 220 кВ: \(F_\textup{а} = 240 \div 400\; \textup{мм}^2\).

При проектировании ВЛ 220 кВ и выше с проводами АС конструкцию провода, то есть \(m = \frac{F_a}{F_c}\) следует выбирать с учетом рекомендаций в соответствии с п. 2.5.80 ПУЭ \cite{пуэ7} в зависимости от района по гололеду. Для Костромской области с районом по гололеду I с толщиной стенки 25 мм и менее при \(F_\textup{а} \leq 185\; \textup{мм}^2\) "--- \(m \approx 6\), а для \(F_\textup{а} \geq 240\; \textup{мм}^2\) "--- \(m \approx 8\).

\subsection{Расчет для варианта схемы сети 1}
\label{sec:эконом_сечение_кольца}

Поскольку потокораспределение активной мощности в проектируемой сети известно, для определения значения полной мощности передаваемой по ВЛ в режиме наибольших нагрузок, необходимо выполнить расчет предварительного (без учета потерь мощности) потокораспределения реактивной мощности по участкам сети.
\[Q_{45} = Q_\textup{прив5} = 14,1\; \textup{МВар}\]
\[Q_{34} = Q_{45} + Q_\textup{прив4} = 14,1 + 14,2 = 28,3\; \textup{МВар}\]

На данном этапе сечения проводов пока неизвестны, поэтому потокораспределение реактивной мощности в кольцевой сети оценивается по длинам линии, вместо комплексно-сопряженных сопротивлений.

Реактивная мощность передаваемая по головным участкам сети:
\[
\begin{split}
	Q_{K1} &= \frac{Q_\textup{прив1} (L_{13} + L_{32} + L_{K2}) + Q_\textup{прив3}(L_{32} + L_{K2}) + Q_\textup{прив2}\cdot L_{K2}}{L_{13} + L_{23} + L_{K2} + L_{K1}} = \\ &= \frac{26(49,2 + 41,8 + 49,2) + 46,1(41,8 + 49,2) + 26\cdot 49,2}{49,2 + 41,8 + 49,2 + 41,8} = 50,1\; \textup{МВар}
\end{split}
\]

\[
	\begin{split}
		Q_{K2} &= \frac{Q_\textup{прив2}(L_{23} + L_{13} + L_{K1}) + Q_\textup{прив3}(L_{13} + L_{K1}) + Q_\textup{прив1}\cdot L_{K1}}{L_{23} + L_{13} + L_{K1} + L_{K2}} = \\ &= \frac{26(41,8 + 49,2 + 41,8) + 46,1(49,2 + 41,8) + 26\cdot 41,8}{41,8 + 49,2 + 41,8 + 49,2} = 48,0\; \textup{МВар}
	\end{split}
\]

Сделаем проверку:
\[Q_{K1} + Q_{K3} = 50,1 + 48,0 = 98,1\; \textup{МВар}\]
\[Q_\textup{прив1} + Q_\textup{прив2} + Q_\textup{прив3} = 26 + 26 + 46,1 = 98,1\; \textup{МВар},\]
следовательно можно сделать вывод, что потокораспределение посчитано правильно.

Выполним расчет потокораспределения реактивной мощности по неголовным участкам сети:
\[Q_{13} = Q_{K1} - Q_\textup{прив1} = 50,1 - 26 = 24,1\; \textup{МВар}\]
\[Q_{23} = Q_{K2} - Q_\textup{прив2} = 48 - 26 = 22,0\; \textup{МВар}\]

Так как \(Q_{13} > 0\) и \(Q_{23} > 0\), следовательно точкой потокораздела по реактивной мощности является точка 3.

Согласно табл. \ref{tab:j_ek_norm} для проводов с токопроводящими алюминиевыми проволоками при числе часов использования \(T_\textup{нб} = 4350\; \frac{\textup{ч}}{\textup{год}}\), норматив \(j_\textup{эк} = 0,9\; \frac{\textup{А}}{\textup{мм}^2}\). В качестве примера выполним расчет экономически целесообразного сечения \(F_\textup{эк}\) и выберем марку проводов для линии K-1.

Полная мощность, передаваемая по линии K-1 в режиме наибольших нагрузок:
\[S_{K1} = \sqrt{P_{K1}^2 + Q_{K1}^2} = \sqrt{125,3^2 + 50,1^2} = 134,9\; \textup{МВА}\]

Наибольший ток одной цепи на пятый год эксплуатации по формуле (\ref{eqn:наиб_ток_1цепи}):
\[I_\textup{нб(5)К1} = \frac{S_{K1}}{\sqrt{3}\cdot U_\textup{ном}\cdot n_\textup{ц(К1)}} = \frac{134,9}{\sqrt{3}\cdot 220\cdot 1} = 0,354\; \textup{кА} = 354\; \textup{А}\]

Расчетный ток одной цепи линии вычислим по формуле (\ref{eqn:расч.ток_1цепи}):
\[I_\textup{р(К1)} = \alpha_i\cdot I_\textup{нб(5)К1} = 1,05\cdot 354 = 371,7\; \textup{А}\]

Экономически целесообразное сечение токопроводящей алюминиевой части провода: \(F_\textup{эк(К1)} = \frac{I_\textup{р(К1)}}{j_\textup{эк}} = \frac{377,7}{0,9} = 413\; \textup{мм}^2\). Так как сечение алюминиевой части получилось больше \(240\; \textup{мм}^2\), то выбираем провод облегченной конструкции марки АС 400/51.

Аналогично выбирается сечение для остальных участков ВЛ. Результаты расчета экономически целесообразных сечений и выбора марок проводов ВЛ сведем в табл. \ref{tab:эконом_сечение_результаты}.

\begin{table}[H]
	\small
	\caption{Результаты расчета экономически целесообразных сечений и выбора марок проводов для варианта схемы сети 1}
	\label{tab:эконом_сечение_результаты}
	\begin{tabularx}{\textwidth}{|Z|Z|Z|Z|Z|Z|Z|}
		\hline
		Линия                             & К1        & К2        & 13        & 23        & 34        & 45        \\ \hline
		\(U_\textup{ном}\), кВ            & 220       & 220       & 220       & 220       & 110       & 110       \\ \hline
		\(n_\textup{ц}\)                  & 1         & 1         & 1         & 1         & 2         & 2         \\ \hline
		\textit{P}, МВт                   & 125,3     & 119,7     & 55,3      & 49,7      & 75        & 35        \\ \hline
		\textit{Q}, МВар                  & 50,1      & 48,0      & 24,1      & 22,0      & 28,3      & 14,1      \\ \hline
		\textit{S}, МВА                   & 134,9     & 129,0     & 60,3      & 54,4      & 80,2      & 37,7      \\ \hline
		\(I_\textup{нб(5)}\), А           & 354       & 338       & 158       & 143       & 210       & 99        \\ \hline
		\(I_\textup{р}\), А               & 372       & 355       & 166       & 150       & 221       & 104       \\ \hline
		\(F_\textup{эк}\; \textup{мм}^2\) & 413       & 395       & 185       & 166       & 246       & 116       \\ \hline
		\(F_\textup{А}\; \textup{мм}^2\)  & 400       & 400       & 185       & 150       & 240       & 120       \\ \hline
		Марка провода                     & АС 400/51 & АС 400/51 & АС 185/29 & АС 150/24 & АС 240/32 & АС 120/19 \\ \hline
	\end{tabularx}	
\end{table}

\subsection{Расчет для варианта схемы сети 2}

Выполним расчет потокораспределения реактивной мощности по участкам сети:
\[Q_{45} = Q_\textup{прив5} = 11,7\; \textup{МВар}\]
\[Q_{34} = Q_{45} + Q_\textup{прив4} = 11,7 + 11,8 = 23,5\; \textup{МВар}\]
\[Q_{23} = Q_\textup{прив2} = 23,6\; \textup{МВар}\]
\[Q_{13} = Q_\textup{прив3} = 70,7\; \textup{МВар}\]
\[Q_{\textup{К1}} = Q_{13} + Q_\textup{прив1} = 70,7 + 23,6 = 94,3\; \textup{МВар}\]

Расчет экономически целесообразного сечения токопроводящей алюминиевой части провода проводится аналогично разделу \ref{sec:эконом_сечение_кольца}, поэтому сведем результаты расчетов в табл. \ref{tab:эконом_сечение_результаты_магистраль}.

\begin{table}[H]
	\small
	\caption{Результаты расчета экономически целесообразных сечений и выбора марок проводов для варианта схемы сети 2}
	\label{tab:эконом_сечение_результаты_магистраль}
	\begin{tabularx}{\textwidth}{|Z|Z|Z|Z|Z|Z|}
		\hline
		Линия                             & К1        & 13        & 23        & 34        & 45        \\ \hline
		\(U_\textup{ном}\), кВ            & 220       & 220       & 110       & 110       & 110       \\ \hline
		\(n_\textup{ц}\)                  & 2         & 2         & 2         & 2         & 2         \\ \hline
		\textit{P}, МВт                   & 245       & 175       & 70        & 75        & 35        \\ \hline
		\textit{Q}, МВар                  & 94,3      & 70,7      & 23,6      & 23,5      & 11,7      \\ \hline
		\textit{S}, МВА                   & 262,5     & 188,7     & 73,9      & 78,6      & 36,9      \\ \hline
		\(I_\textup{нб(5)}\), А           & 345       & 248       & 194       & 206       & 96,8      \\ \hline
		\(I_\textup{р}\), А               & 362       & 260       & 204       & 217       & 102       \\ \hline
		\(F_\textup{эк}\; \textup{мм}^2\) & 402       & 289       & 226       & 241       & 113       \\ \hline
		\(F_\textup{А}\; \textup{мм}^2\)  & 400       & 300       & 240       & 240       & 120       \\ \hline
		Марка провода                     & АС 400/51 & АС 300/39 & АС 240/32 & АС 240/32 & АС 120/19 \\ \hline
	\end{tabularx}	
\end{table}


\section{Проверка выбранных сечений по условиям технических ограничений}

Выбранные экономически целесообразные сечения проводов ВЛ 110-220 кВ должно удовлетворять следующим технических ограничениям \cite{глазунов_шведов}:
\begin{itemize}
	\item по условиям механической прочности;
	\item по ограничениям потерь на корону и уровня радиопомех;
	\item по условию длительно допустимого нагрева.
\end{itemize}

Сечения проводов линий 35 кВ и выше проверке по допустимым потерям напряжения не подлежат, так как уменьшение потерь напряжения путем увеличения сечений линий экономически нецелесообразно по сравнению с применением трансформаторов с устройством РПН и устройств компенсации реактивной мощности \cite{файбисович}. 

В соответствии с \cite{пуэ7} провода воздушных линий напряжением выше 1 кВ на термическую стойкость к токам короткого замыкания не проверяются, за исключением линий, оборудованных устройствами быстродействующего автоматического повторного включения, поскольку в этом случае необходимо учитывать повышение нагрева из-за увеличения суммарной продолжительности протекания тока короткого замыкания по таким линиям.

\subsection{Проверка сечений проводов по условиям технических ограничений для схемы варианта сети 1}

\subsection*{\textit{Проверка по условиям механической прочности}}

Поскольку при одной и той же толщине стенки гололеда наибольшему риску обрыва подвержены провода малого диаметра и сечения, в п. 2.5.77 \cite{пуэ7} введены ограничения на минимальные сечения проводов по условиям механической прочности в зависимости от района по гололеду (см. табл. \ref{tab:мин_допустим_сечение_по_мех_прочности}). ВЛ 35 кВ и выше, сооружаемые на двухцепных и многоцепных опорах, относятся к более ответственным объектам, поэтому по условиям механической прочности для этих линий не допускается применение сталеалюминиевых проводов с сечением менее чем АС 120/19 вне зависимости от района по гололеду.

\begin{table}[h]
	\small
	\caption{Минимально допустимые сечения сталеалюминиевых проводов воздушных линий по условиям механической прочности}
	\label{tab:мин_допустим_сечение_по_мех_прочности}
	\begin{tabularx}{\textwidth}{|Z|Z|}
		\hline
		Характеристика воздушной линии & \(F_\textup{min.мех}, \; \textup{мм}^2\) \\ \hline
		\multicolumn{2}{|c|}{Воздушные линии, сооружаемые на одноцепных опорах в районах по гололеду:} \\ \hline
		до II & 35/6,2 \\ \hline
		до III-IV & 50/8 \\ \hline
		в V и более & 70/11 \\ \hline
		Воздушные линии, сооружаемые на двухцепных или многоцепных опорах & 120/19 \\ \hline
	\end{tabularx}
\end{table}

В соответствии с табл. П1 \cite{глазунов_шведов} Костромская область находится в I районе по гололеду. Поэтому для одноцепных ВЛ К1, К2, 13, 23 (без учета пересечений) минимальное сечение сталеалюминиевых проводов по условиям механической прочности \(F_\textup{мех.min} = 35/6,2 \; \textup{мм}^2\).

Вывод: результаты расчета экономически целесообразных сечений показывают, что для всех одноцепных ВЛ 220 кВ приведенные в табл. \ref{tab:эконом_сечение_результаты} сечения проводов удовлетворяют условиям механической прочности. Как следует из табл. \ref{tab:мин_допустим_сечение_по_мех_прочности}, минимально допустимое сечение сталеалюминиевых проводов ВЛ 35 кВ и выше, сооружаемых на двухцепных опорах, равно 120/19. Так как для линий 34 и 45 полученные значения экономически целесообразных сечений не менее минимально допустимых сечений сталеалюминиевых проводов для двухцепных ВЛ 110 кВ, сечения проводов также удовлетворяют условию механической прочности.

\subsection*{\textit{Проверка по условиям ограничения потерь на корону и уровня радиопомех}}

При выборе сечений проводов воздушных линий напряжением 110 кВ и выше необходимо соблюдать ограничение напряженности электрического поля на поверности проводов до уровней, допустимых по короне и радиопомехам. По условиям ограничения потерь мощности на корону и уровня радиопомех рекомендуется применять на воздушных линиях провода диаметром не менее указанных в табл. \ref{tab:мин_сечение_по_усл_на_корону} \cite{пуэ7}.

\begin{table}[H]
	\small
	\caption{Минимально допустимые диаметры проводов воздушных линий 110-220 кВ и соответствующие им сечения сталеалюминиевых проводов по условиям ограничения потерь на корону и уровня радиопомех}
	\label{tab:мин_сечение_по_усл_на_корону}
	\begin{tabularx}{\textwidth}{|Z|Z|Z|}
		\hline
		\(U_\textup{ном}\), кВ & 110 & 220 \\ \hline
		\(d_\textup{min.кор}\), мм & 11,4 & 21,6 \\ \hline
		\(F_\textup{min.кор}, \; \textup{мм}^2\) & 70/11 & 240/32 \\ \hline
	\end{tabularx}
\end{table}

Так как каждому значению \(d_\textup{min.кор}\) соответствует вполне определенная марка провода, то проверка выбранного экономически целесообразного сечения провода по условию ограничения потерь на корону и уровня радипомех сводится к условию:
\begin{equation}
	F_\textup{эк} \geq F_\textup{min.кор}.
	\label{eqn:неравенство_огр_потерь_на_корону}
\end{equation}

В случае несоблюдения неравенства \eqref{eqn:неравенство_огр_потерь_на_корону} необходимо увеличить сечение провода до значения \(F_\textup{min.кор}\). 

В рассматриваемой схеме кольцевой сети не удовлетворяют неравенству \eqref{eqn:неравенство_огр_потерь_на_корону} участки ВЛ 13 и 23, следовательно их сечения необходимо увеличить до минимально допустимого путем замены проводов марок АС 185/29 и АС 150/24 на АС 240/32.

Для линий 34 и 45 номинальным напряжением 110 кВ полученные значения экономически целесообразных сечений превышают \(F_\textup{min.кор} = 70/11\), то есть условие ограничения потерь на корону и уровня радиопомех соблюдаются.

\subsection*{\textit{Проверка по условию длительно допустимого нагрева}}

В соответствии с п. 1.3.2 ПУЭ \cite{пуэ7} проводники любого назначения должны удовлетворять требованием, в отношении предельно-допустимого нагрева с учетом не только нормальных, но и послеаварийных, ремонтных режимов. При невыполнении проверки по длительно допустимому нагреву проводов следует рассматривать такие послеаварийные режимы, которые приводят к наибольшему увеличению, протекающего по линии тока.

Для проводов воздушных линий на основе практики эксплуатации установлено значение длительно допустимой температуры нагрева проводов \(T_\textup{дл.доп}\), равной \(70\; ^oC\) \cite{пуэ7}. В справочных данных для расчетной температуры воздуха \(T_\textup{расч}\), равной \(25\; ^oC\), приводятся соответствующие длительно допустимые токи \(I_\textup{дл.доп}\), при протекании которых провод нагревается до длительно допустимой температуры.

В случае если \(T_\textup{факт} = 25\; ^oC\), то значение \(I_\textup{дл.доп}\) следует пересчитать по формуле:
\begin{equation}
	I_\textup{дл.доп}^{'} = I_\textup{дл.доп} \sqrt{\frac{T_\textup{дл.доп} - T_\textup{факт}}{T_\textup{дл.доп} - T_\textup{расч}}} = I_\textup{дл.доп} \sqrt{\frac{70 - T_\textup{факт}}{70 - 25}} = I_\textup{дл.доп} \cdot k_T,
\end{equation}
где \(k_T\) "--- поправочный коэффициент на температуру воздуха, учитывающий отличие фактической температуры воздуха от расчетной.

Таким образом, проверка выбранного экономически целесообразного сечения \(F_\textup{эк}\) провода по условию длительно допустимого нагрева сводится к условию:
\begin{eqndesc}[h]
	\begin{equation}
		I_\textup{л} \leq I_\textup{дл.доп} \cdot k_T,
		\label{eqn:условие_дл_доп_ток}
	\end{equation}
где \(I_\textup{л}\) "--- ток, протекающий по проводам линии в расчетном режиме проверки.
\end{eqndesc}

При невыполнении неравенства \eqref{eqn:условие_дл_доп_ток} следует пошагого увеличить сечение провода.

Проверку экономически целесообразных сечений проводов ВЛ 220 кВ и ниже по условию длительно допустимого нагрева требуется выполнять только в послеаварийных режимах для линий кольцевых сетей. При этом следует рассматривать послеаварийный режим, приводящий к наибольшему увеличению протекающего по линии тока, то есть отключение наиболее нагруженных головных участков.

В качестве примера рассмотрим проверку по условию длительно допустимого нагрева провода марки АС 400/51 линии К1.

Согласно п. 1.3.29 ПУЭ \cite{пуэ7} длительно допустимый ток для марки провода АС 400/51 ВЛ К1 \(I_\textup{дл.доп} = 825\) А.

Как правило, режим наибольших нагрузок приходится на наиболее холодный зимний месяц "--- январь. Поэтому в качестве фактической применяется среднеянварская температура, тогда поправочный коэффициент на температуру воздуха:
\[k_T = \sqrt{\frac{T_\textup{дл.доп} - T_\textup{факт}}{T_\textup{дл.доп} - 25}} = \sqrt{\frac{70 - (-11,8)}{70 - 25}} = 1,35\]

Тогда скорректированный длительно допустимый ток вычислим по формуле \eqref{eqn:условие_дл_доп_ток}:
\[I_\textup{дл.доп}^{'} = I_\textup{дл.доп}\cdot k_T = 825\cdot 1,35 = 1114\; \textup{А}\]

При проверке по условию длительно допустимого нагрева провода линии К1 в качестве наиболее тяжелого послеаварийного режима следует рассмотреть отключение другого головного участка линии К2. В этом случае активная, реактивная и полная мощности, передаваемые по линии К1 будут соответственно равны:
\[P_\textup{К1(п/ав)} = P_{нб1} + P_{нб2} + P_{нб3}^{'} = 70 + 70 + 105 = 245\; \textup{МВт};\]
\[Q_\textup{К1(п/ав)} = Q_\textup{прив1} + Q_\textup{прив2} + Q_\textup{прив3} = 26 + 26 + 46,1 = 98,1\; \textup{МВар};\]
\[S_\textup{К1(п/ав)} = \sqrt{P_\textup{К1(п/ав)}^2 + Q_\textup{К1(п/ав)}^2} = \sqrt{245^2 + 98,1^2} = 263,9\; \textup{МВА}\]

Ток, протекающий в послеаварийном режиме по линии К1 при отключении линии К2:
\[I_\textup{К1(п/ав)} = \frac{S_\textup{К1(п/ав)}}{\sqrt{3} \cdot U_\textup{ном} \cdot n\textup{цК1}} = \frac{263,9}{\sqrt{3}\cdot 220\cdot 1} = 0,693\; \textup{кА} = 693\; \textup{А}\]

Так как \(I_\textup{К1(п/ав)} = 693\) А не превышает длительно допустимый ток \(I_\textup{дл.доп}^{'} = 1114\) А для линии К1 с маркой провода АС 400/51, то можно сделать вывод, что данная марка удовлетворяет техническим ограничениям по длительно допустимому нагреву провода.

Расчет для остальных воздушных линий сети 220 кВ проводится аналогично, поэтому сведем результаты расчетов в таблицу \ref{tab:результаты_проверки_по_дл.доп}.

\begin{table}[H]
	\small
	\caption{Результаты проверки сечений проводов линий по условию длительно допустимого нагрева}
	\label{tab:результаты_проверки_по_дл.доп}
	\begin{tabularx}{\linewidth}{|Z|Z|Z|Z|Z|}
		\hline
		Линия                        & К1        & К2        & 13        & 23        \\ \hline
		Марка провода                & АС 400/51 & АС 400/51 & АС 240/32 & АС 240/32 \\ \hline
		\(k_T\)                      & 1,35      & 1,35      & 1,35      & 1,35      \\ \hline
		\(I_\textup{дл.доп}^{'}\), А & 1114      & 1114      & 817       & 817       \\ \hline
		Отключение линии             & К2        & К1        & К2        & К1        \\ \hline
		\textit{P}, МВт              & 245       & 245       & 175       & 175       \\ \hline
		\textit{Q}, МВар             & 98,1      & 98,1      & 72,1      & 72,1      \\ \hline
		\textit{S}, МВА              & 263,9     & 263,9     & 189,3     & 189,3     \\ \hline
		\(I_\textup{л}\), А          & 693       & 693       & 497       & 497       \\ \hline
	\end{tabularx}
\end{table}

Как видно из табл. \ref{tab:результаты_проверки_по_дл.доп}, для всех линий кольцевой сети сечения проводов удовлетворяют условию длительно допустимого нагрева.

\subsection{Проверка сечений проводов по условиям технических ограничений для схемы варианта сети 2}

Помимо того, что для этой схемы не учитываются ограничения по потерям напряжения и термической стойкости к токам короткого замыкания, в радиально-магистральной сети так же не проводится проверка по условию длительно допустимого нагрева, так как для проводов ВЛ напряжением до 220 кВ включительно норматив экономической плотности тока более чем в два раза меньше длительно допустимой плотности тока. Следовательно, для двухцепных радиальных и магистральных линий, провода которых выбраны по экономической плотности тока, в послеаварийных режимах, связанных с отключением одной цепи, удвоенный ток нормального режима, протекающий по оставшейся в работе сторой цепи, будет меньше длительно допустимого тока, то есть неравенство \eqref{eqn:условие_дл_доп_ток} при этих условиях всегда будет выполняться.

\subsection*{Проверка по условиям механической прочности}

Результаты расчета, приведенные в табл. \ref{tab:эконом_сечение_результаты_магистраль}, показывают, что для всех двухцепных линий 110-220 кВ в соответствии с табл. \ref{tab:мин_допустим_сечение_по_мех_прочности} сечения проводов удовлетворяют условиям механической прочности.

\subsection*{Проверка по условиям ограничения потерь на корону и уровня радиопомех}

В рассматриваемой радиально-магистральной схеме сети все марки проводов ВЛ удовлетворяют неравенству \eqref{eqn:неравенство_огр_потерь_на_корону}.

Вывод: Выбранные в табл. \ref{tab:эконом_сечение_результаты_магистраль} экономически целесообразные сечения проводов радиально-магистральной сети удовлетворяют всем техническим ограничениям.

%%% Local Variables:
%%% mode: latex
%%% TeX-master: "rpz"
%%% End: