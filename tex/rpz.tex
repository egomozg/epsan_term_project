%% Преамбула TeX-файла

% 1. Стиль и язык
\documentclass[utf8x, 14pt]{G7-32} % Стиль (по умолчанию будет 14pt)
\bibliographystyle{gost780u}

% Остальные стандартные настройки убраны в preamble.inc.tex.
\sloppy

% Настройки стиля ГОСТ 7-32
% Для начала определяем, хотим мы или нет, чтобы рисунки и таблицы нумеровались в пределах раздела, или нам нужна сквозная нумерация.
\EqInChapter % формулы будут нумероваться в пределах раздела
\TableInChapter % таблицы будут нумероваться в пределах раздела
\PicInChapter % рисунки будут нумероваться в пределах раздела

% Добавляем гипертекстовое оглавление в PDF
\usepackage[
bookmarks=true, colorlinks=true, unicode=true,
urlcolor=black,linkcolor=black, anchorcolor=black,
citecolor=black, menucolor=black, filecolor=black,
]{hyperref}

\AfterHyperrefFix

\usepackage{microtype}% полезный пакет для микротипографии, увы под xelatex мало чего умеет, но под pdflatex хорошо улучшает читаемость

% Тире могут быть невидимы в Adobe Reader
\ifInvisibleDashes
\MakeDashesBold
\fi


\usepackage{graphicx}   % Пакет для включения рисунков

% С такими оно полями оно работает по-умолчанию:
% \RequirePackage[left=20mm,right=10mm,top=20mm,bottom=20mm,headsep=0pt,includefoot]{geometry}
% Если вас тошнит от поля в 10мм --- увеличивайте до 20-ти, ну и про переплёт не забывайте:
\geometry{right=20mm}
\geometry{left=30mm}
\geometry{bottom=20mm}
\geometry{ignorefoot}% считать от нижней границы текста

\usepackage{rotating}

% доп. позиционирование таблиц
\usepackage{float}

% таблицы с автоматическим определением ширины
\usepackage{tabularx}

% для ультра лютой большой таблицы
\usepackage{xltabular}

% Пакет Tikz
\usepackage{tikz}
\usetikzlibrary{arrows,positioning,shadows}

% Произвольная нумерация списков.
\usepackage{enumerate}

% ячейки в несколько строчек
\usepackage{multirow}

% itemize внутри tabular
\usepackage{paralist,array}
\newcolumntype{Y}{>{\centering\arraybackslash}X}

%\setlength{\parskip}{1ex plus0.5ex minus0.5ex} % разрыв между абзацами
\setlength{\parskip}{1ex} % разрыв между абзацами
\setlength{\intextsep}{4pt}
\setlength{\abovecaptionskip}{4pt}
\setlength{\floatsep}{5pt plus 1.0pt minus 1.0pt}
\usepackage{blindtext}

% Центрирование подписей к плавающим окружениям
%\usepackage[justification=centering]{caption}

\usepackage{newfloat}
\DeclareFloatingEnvironment[
placement={!ht},
name=Equation
]{eqndescNoIndent}
\edef\fixEqndesc{\noexpand\setlength{\noexpand\parindent}{\the\parindent}\noexpand\setlength{\noexpand\parskip}{\the\parskip}}
\newenvironment{eqndesc}[1][!ht]{%
    \begin{eqndescNoIndent}[#1]%
\fixEqndesc%
}
{\end{eqndescNoIndent}}
\usepackage{mathtext}

% Настройки листингов.
\ifPDFTeX
% 8 Листинги

\usepackage{listings}

% Значения по умолчанию
\lstset{
  basicstyle= \footnotesize,
  breakatwhitespace=true,% разрыв строк только на whitespacce
  breaklines=true,       % переносить длинные строки
%   captionpos=b,          % подписи снизу -- вроде не надо
  inputencoding=koi8-r,
  numbers=left,          % нумерация слева
  numberstyle=\footnotesize,
  showspaces=false,      % показывать пробелы подчеркиваниями -- идиотизм 70-х годов
  showstringspaces=false,
  showtabs=false,        % и табы тоже
  stepnumber=1,
  tabsize=4,              % кому нужны табы по 8 символов?
  frame=single
}

% Стиль для псевдокода: строчки обычно короткие, поэтому размер шрифта побольше
\lstdefinestyle{pseudocode}{
  basicstyle=\small,
  keywordstyle=\color{black}\bfseries\underbar,
  language=Pseudocode,
  numberstyle=\footnotesize,
  commentstyle=\footnotesize\it
}

% Стиль для обычного кода: маленький шрифт
\lstdefinestyle{realcode}{
  basicstyle=\scriptsize,
  numberstyle=\footnotesize
}

% Стиль для коротких кусков обычного кода: средний шрифт
\lstdefinestyle{simplecode}{
  basicstyle=\footnotesize,
  numberstyle=\footnotesize
}

% Стиль для BNF
\lstdefinestyle{grammar}{
  basicstyle=\footnotesize,
  numberstyle=\footnotesize,
  stringstyle=\bfseries\ttfamily,
  language=BNF
}

% Определим свой язык для написания псевдокодов на основе Python
\lstdefinelanguage[]{Pseudocode}[]{Python}{
  morekeywords={each,empty,wait,do},% ключевые слова добавлять сюда
  morecomment=[s]{\{}{\}},% комменты {а-ля Pascal} смотрятся нагляднее
  literate=% а сюда добавлять операторы, которые хотите отображать как мат. символы
    {->}{\ensuremath{$\rightarrow$}~}2%
    {<-}{\ensuremath{$\leftarrow$}~}2%
    {:=}{\ensuremath{$\leftarrow$}~}2%
    {<--}{\ensuremath{$\Longleftarrow$}~}2%
}[keywords,comments]

% Свой язык для задания грамматик в BNF
\lstdefinelanguage[]{BNF}[]{}{
  morekeywords={},
  morecomment=[s]{@}{@},
  morestring=[b]",%
  literate=%
    {->}{\ensuremath{$\rightarrow$}~}2%
    {*}{\ensuremath{$^*$}~}2%
    {+}{\ensuremath{$^+$}~}2%
    {|}{\ensuremath{$|$}~}2%
}[keywords,comments,strings]

% Подписи к листингам на русском языке.
\renewcommand\lstlistingname{Листинг}
\renewcommand\lstlistlistingname{Листинги}

\else
\usepackage{local-minted}
\fi

% Полезные макросы листингов.
% Любимые команды
\newcommand{\Code}[1]{\textbf{#1}}


% Стиль титульного листа и заголовки
%\include{00-title}


\begin{document}

\frontmatter % выключает нумерацию ВСЕГО; здесь начинаются ненумерованные главы: реферат, введение, глоссарий, сокращения и прочее.

\maketitle %создает титульную страницу


%\begin{executors}
%\personalSignature{Первый исполнитель}{ФИО}
%
%\personalSignature{Второй исполнитель}{ФИО}
%\end{executors}


%\listoffigures                         % Список рисунков

%\listoftables                          % Список таблиц

%\NormRefs % Нормативные ссылки 
% Команды \breakingbeforechapters и \nonbreakingbeforechapters
% управляют разрывом страницы перед главами.
% По-умолчанию страница разрывается.

% \nobreakingbeforechapters
% \breakingbeforechapters

%% Также можно использовать \Referat, как в оригинале
\begin{abstract}

   Для электрической сети, схема которой показана на *, на основании исходных данных по узлам и ветвям, приведенных в табл. * и *:
   
   \begin{enumerate}[1.]
   	\item Составить схему замещения сети и определить ее параметры
   	\item Выполнить расчеты потокораспределения и напряжений в узлах сети в нормальном режиме наибольших нагрузок.
   	\item Выполнить расчеты потокораспределения и напряжений в узлах сети в нормальном режиме наименьших нагрузок.
   	\item Выполнить расчеты потокораспределения и напряжений в узлах сети в послеаварийном
   	режиме (отключение одной цепи линии).
   	\item Оценить достаточность регулировочных диапазонов устройств РПН трансформаторов на
   	подстанции.
   	\item Рассчитать потери активной мощности и годовые потери электроэнергии в сети.
   \end{enumerate}

\end{abstract}

%%% Local Variables: 
%%% mode: latex
%%% TeX-master: "rpz"
%%% End: 


\tableofcontents

%\printnomenclature % Автоматический список сокращений

\Introduction

Целью выполнения расчетного задания является получение практических навыков по решению задач, связанных с составлением схем замещения основных элементов электрических сетей, к которым относятся линии электропередачи (ЛЭП) и подстанции (ПС), а также расчетам установившихся нормальных и послеаварийных режимов электрической сети и анализу полученных результатов.

Кроме того, в рамках расчетного задания решаются задачи оценки достаточности регулировочных диапазонов устройств регулирования напряжения трансформаторов на понижающей подстанции в различных режимах работы электрической сети, рассчитываются потери активной мощности и годовые потери электроэнергии в сети.

Полученные практические навыки необходимы для выполнения предстоящего курсового проекта по дисциплине «Электроэнергетические системы и сети» и будут полезны при подготовке выпускной квалификационной работы бакалавра.


\chapter{Исходные данные}
\label{cha:data}

Источником питания (ИП) электрической сети (рис.) является районная подстанция, работающая в составе электроэнергетической системы, с распределительными устройствами 110 и 220 кВ.

Номинальное напряжение сети $U_{\text{ном}}$, марка проводов и длина линии $L_{\text{ИП-ПС}}$ от источника питания (ИП) до ПС, трансформаторы, установленные на ПС, приведены в табл.~\ref{tab:tabl1}. На шинах ИП района в режиме наибольших нагрузок обеспечивается напряжение, указанное в табл.~\ref{tab:tabl1} , а в режиме наименьших нагрузок 97 \% от номинального. Низшее напряжение на ПС составляет 10 кВ.

Мощность нагрузок на шинах низшего и среднего напряжений (НН и СН) ПС в режиме наибольших нагрузок ($P_{\text{н}}$ и $P_{\text{с}}$), наименьшая нагрузка (относительное снижение нагрузки $\alpha$) и число часов использования наибольших нагрузок $T_{\text{нб}}$ приведены в табл.~\ref{tab:tabl2}.

\renewcommand{\thetable}{\arabic{table}}
\begin{table}[H]
	\caption{Исходные данные для расчета (часть 1)}
	\begin{tabular}{|r|c|c|c|l|}
		\hline
		$U_{\text{ном}}$ , кВ & $U_{\text{ИП}}$, \% & $L_{\text{ИП-ПС}}$ & Марка провода & Трансформатор \\
		\hline
		110 & 113 & 84 & АС 150/24 & ТДТН-40000/110/35 \\
		\hline
	\end{tabular}
	\label{tab:tabl1}
\end{table}

\begin{table}[H]
	\caption{Исходные данные для расчета (часть 2)}
	\begin{tabular}{|r|c|c|c|c|l|}
		\hline
		$P_{c}$, кВт & $cos\varphi_c$ & $P_{\text{н}}$, кВт & $cos\varphi_{\text{н}} / tg\varphi_{\text{н}} $ & $\alpha$, отн. ед. & $T_{\text{нб}}$, ч/год \\
		\hline
		33000 & 0,82 & 22880 & - / 0,44 & 0,4 & 6820 \\
		\hline
	\end{tabular}
	\label{tab:tabl2}
\end{table}

По табл. 3.5 и 3.8  определяем необходимые для расчета исходные данные по проводу АС 150/24 и сводим их в табл.~\ref{tab:tabl3}

\begin{table}[H]
	\caption{Расчетные данные сталеалюминевого провода АС 150/24}
	\begin{tabular}{|c|c|c|}
		\hline
		Номинальное сечение & Диаметр провода  & Активное сопротивление \\
		(алюминий/сталь) & $d_{\text{пр}}$, мм & постоянному току при \\
		провода, $\text{мм}^2$ & & температуре 20 $^{\circ}$C \\
		 & & $R_0$, Ом/км \\
		\hline
		150/24 & 17,1 & 0,204 \\
		\hline
	\end{tabular}
	\label{tab:tabl3}
\end{table}


\renewcommand{\thefigure}{\arabic{chapter}.\arabic{figure}} % возвращаем нормальную нумерацию картинок
\renewcommand{\thetable}{\arabic{chapter}.\arabic{table}} % возвращаем нормальную нумерацию таблиц

%%% Local Variables: 
%%% mode: latex
%%% TeX-master: "rpz"
%%% End: 

\mainmatter % это включает нумерацию глав и секций в документе ниже


%\backmatter %% Здесь заканчивается нумерованная часть документа и начинаются ссылки и
            
%\include{80-conclusion}%% заключение


% % Список литературы при помощи BibTeX
% Юзать так:
%
% pdflatex rpz
% bibtex rpz
% pdflatex rpz

\bibliography{rpz}

%%% Local Variables: 
%%% mode: latex
%%% TeX-master: "rpz"
%%% End: 
 %% список литературы


%\appendix   % Тут идут приложения

%\include{90-appendix1}

%\include{91-appendix2}

\end{document}

%%% Local Variables:
%%% mode: latex
%%% TeX-master: t
%%% End:
