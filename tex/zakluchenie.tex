\Conclusion

В данном курсовом проекте были сформированы два предварительных варианта схемы сети, для которых решены следующие задачи:
\begin{itemize}
\item выбраны номинальные напряжения участков схемы сети; 
\item определены номинальные мощности для ТР и АТР, а также произведен выбор ЛРТ;
\item выбраны марки и сечения проводов воздушных линий электропередачи, удовлетворяющие ряду технических ограничений;
\item выбраны схемы РУ в зависимости от топологии схемы сети, а также количества кабелей в сети 10 кВ в зависимости от заданной мощности нагрузки.
\end{itemize}
По методу дисконтированных затрат произведено технико-экономическое сопоставление предложенных вариантов схемы сети и выбран наиболее рациональный вариант 1, для которого в ПВК RastrWin рассчитаны режимы НБ, НМ, П/АВ и показана их осуществимость, а также возможность поддержания требуемого уровня напряжения у потребителей согласно принципу встречного регулирования с помощью устройств РПН и ЛРТ.
В заключительном разделе определены основные технико-экономические показатели спроектированной электрической сети и рассчитана себестоимость передачи электроэнергии по сети.