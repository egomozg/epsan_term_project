\Introduction

Задачами проектирования электрических сетей является разработка и технико-экономическое обоснование решений по формированию целесообразного комплекса линий электропередачи (ЛЭП) и понижающих подстанций (ПС), обеспечивающих требуемый уровень надежности электроснабжения всех потребителей рассматриваемого района качественной электроэнергией с наименьшими затратами.

В данном курсовом проекте осуществляется «эскизное» проектирование электрических сетей заданного района с четырьмя пунктами потребления электроэнергии, в которых будут сооружаться понижающие подстанции.

К основным задачам, решаемым в данном курсовом проекте, относятся:
\begin{itemize}
\item выбор параметров элементов сети (линий электропередачи и понижающих подстанций) для электроснабжения заданного числа пунктов потребления электроэнергии;
\item анализ характерных установившихся режимов работы спроектированной сети;
\item определение основных технико-экономических показателей спроектированной сети.
\end{itemize}
