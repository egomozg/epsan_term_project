\chapter{Оценка баланса реактивной мощности в проектируемой сети}
\label{cha:ocenka_brm}

На основе оптимизационных расчетов распределения реактивной мощности в электроэнергетической системе для каждого ее узла определяется реактивная мощность, которую целесообразно передавать из электроэнергетической системы в распределительные сети, питающиеся от того или иного узла. Поэтому при проектировании электрической сети, получающей питание от подстанций электроэнергетической системы, задается реактивная мощность \(Q_{\textup{расп}\Sigma}\), которую целесообразно потреблять из системы (в заданном узле присоединения) в режиме наибольших нагрузок, или же коэффициент реактивной мощности. Потребление большей реактивной мощности приведет к дополнительным затратам на передачу этой мощности и, следовательно, к отступлению от оптимального режима питающей системы. В связи с этим, в проекте оценивается выполнение баланса реактивной мощности в проектируемой сети и при необходимости устанавливаются компенсирующие устройства.

\section{Проверка по условию выполнения баланса реактивной мощности в сети и определение приведенных к шинам высшего напряжения нагрузок подстанции}

Потребление реактивной мощности в проектируемой сети в период наибольших нагрузок складывается из заданных реактивных нагрузок в пунктах потребления, рассчитанных с учетом предварительного этапа установки БСК с целью выполнения условия \(tg\,{\varphi_i} \le tg\,{\varphi пред}\) и потерь реактивной мощности в линиях и понижающих трансформаторах с учетом зарядных мощностей линий. При определении одновременно потребляемой реактивной мощности следует учитывать несовпадение по времени суток наибольших нагрузок отдельных потребителей.

Среднестатистическое значение коэффициента одновременности реактивных нагрузок на шинах 220 кВ источника питания составляет 0,98 \cite{глазунов_шведов}. Наибольшая суммарная реактивная мощность потребляемая в проектируемой сети в период наибольших нагрузок рассчитывается по формуле:
\begin{eqndesc}[H]
	\begin{equation}
		Q_{\textup{треб}\Sigma} = k_{\textup{одн}Q} \cdot \sum_{i=1}^{n} Q_{\textup{нб}i} + \Delta Q_{\textup{т}\Sigma} + \sum_{j=1}^{m}\left(\Delta Q_{\textup{л}j} - Q_{cj}\right)
		\label{eqn:sum_q_treb}
	\end{equation}

где \(Q_{\textup{нб}i}\) "--- наибольшая реактивная нагрузка \textit{i}-го пункта с учетом установленных конденсаторных батарей по условию не превышения предельных значений коэффициента реактивной мощности; \\
n "--- число пунктов потребления электроэнергии; \\
\(k_{\textup{одн}Q}\) "--- коэффициент одновременности наибольших реактивных нагрузок подстанций; \\
\(\Delta Q_{\textup{т}\Sigma}\) "--- суммарные потери в трансформаторах подстанций сети; \\
\(Q_{\textup{л}j}\) "--- потери реактивной мощности в \textit{j}-й линии электропередачи; \\
\(Q_{cj}\) "--- зарядная мощность \textit{j}-й линии электропередачи; \\
m "--- число линий электропередачи в сети.
\end{eqndesc}

\subsection*{Расчет для варианта схемы №1:}

а) Суммарная реактивная нагрузка на шинах 220 кВ источника питания К:
\[k_{\textup{одн}Q} \cdot  \sum_{i=1}^{5} Q_{\textup{нб}i}^{'} = 0,98 \cdot 95,4 = 93,5\; \textup{МВар}\]

б) Суммарные потери реактивной мощности в трансформаторах проектируемой сети:
\begin{eqndesc}[H]
\[\Delta Q_{\textup{т}\Sigma} = \sum_{i=1}^{n} \Delta Q_{\textup{т}i} \cong 0,08 \sum_{i=1}^{n} m_{\textup{т}i} \cdot S_{\textup{нб}i},\]

где \(m_{\textup{т}i}\) "--- число трансформации \textit{i}-й ПС.
\end{eqndesc}
\[\Delta Q_{\textup{т}\Sigma} = 0,08(1\cdot 75,2 + 1\cdot 75,2 + 1\cdot 32,1 + 2\cdot 43,0 + 2\cdot 37,5) = 27,5\; \textup{МВар}\]

в) Генерация и потери реактивной мощности в ВЛ:

Зарядная мощность в линии рассчитывается по формуле:
\begin{eqndesc}[H]
\begin{equation}
	Q_c = n_\textup{ц} \cdot q_{c0} \cdot L,
	\label{eqn:зарядная_мощность}
\end{equation}

где \(n_\textup{ц}\) "--- количество цепей в линии; \\
\(q_{c0}\) "--- удельная генерация, \(\frac{\textup{МВар}}{\textup{км}}\); \\
\(q_\textup{с0}^{220} = 0,14\; \frac{\textup{МВар}}{\textup{км}}\) - для ВЛ 220 кВ, \(q_{c0}^{110} = 0,036\; \frac{\textup{МВар}}{\textup{км}}\) - для ВЛ 110 кВ.
\end{eqndesc}

Рассчитаем по формуле \eqref{eqn:зарядная_мощность} зарядную мощность в линии К-1:
\[Q_{cK1} = 1 \cdot 0,14 \cdot 41,8 = 5,85\; \textup{МВар}\]

Для оценки потерь реактивной мощности в ВЛ, воспользуемся известным соотношением \cite{глазунов_шведов}:
\begin{eqndesc}[H]
	\begin{equation}
		\frac{\Delta Q_\textup{л}}{Q_c} \cong \left(\frac{P_\textup{л}}{P_\textup{нат}\cdot n_\textup{ц}}\right)^2 \rightarrow \Delta Q_\textup{л} \cong \left(\frac{P_\textup{1ц}}{P_\textup{нат}}\right)^2 \Delta Q_c,
		\label{eqn:потери_реактивной_мощности_вл}
	\end{equation}

где \(P_\textup{нат}\) "--- мощность при передачи которой по одной цепи линии, потери реактивной мощности в сопротивлении линии равны зарядной мощности в линии. \\
\end{eqndesc}

Натуральная мощность для разных уровней напряжения:
\[P_\textup{нат}^{220} = 130\; \textup{МВт}\]
\[P_\textup{нат}^{110} = 30\; \textup{МВт}\]

Для ВЛ 110 кВ с \(P_\textup{1ц} \leq P_\textup{нат}^{110} = 30\; \textup{кВ} \rightarrow \Delta Q_\textup{л} \approx Q_c\). В нашем случае мы можем отбросить линию 4-5, так как \(P_{1c(45)} = 17,5\; \textup{МВт}\leq P_\textup{нат}^{110} = 30\; \textup{МВт}\).

Потери реактивной мощности в линии по формуле \eqref{eqn:потери_реактивной_мощности_вл}:
\[\Delta Q_\textup{лК1} = \left(\frac{P_\textup{1ц(К1)}}{P_\textup{нат}^{220}}\right)^2 \cdot Q_{cK1} = \left(\frac{125,3}{130}\right)^2 \cdot 5,85 = 5,43\; \textup{МВар}\]

Расчет для остальных линий проводится аналогично. Сведем результаты в таблицу \ref{tab:рез_расчета_q_c0}.

\begin{table}[H]
	\small
	\caption{Результаты расчета \(q_\textup{c0}\) и \(\Delta Q_\textup{л}\)}
	\begin{tabularx}{\textwidth}{|l|Z|Z|Z|Z|Z|Z|}
		\hline
		Линия                  & К-1   & К-2   & 1-3  & 2-3  & 3-4  & \(\Sigma\) \\ \hline
		L, км                  & 41,8  & 49,2  & 49,2 & 41,8 & 25,9 & 207,9          \\ \hline
		\(n_\textup{ц}\)       & 1     & 1     & 1    & 1    & 2    & -          \\ \hline
		\(P_\textup{1ц}\), МВт & 125,3 & 119,7 & 55,3 & 49,7 & 37,5 & -        \\ \hline
		\(Q_c\), МВар          & 5,85  & 6,89  & 6,89 & 5,85 & 1,86 & 27,3       \\ \hline
		\(\Delta Q_\textup{л}\), МВар & 5,43 & 5,84 & 1,25 & 0,86 & 2,91 & 16,3 \\ \hline
	\end{tabularx}
	\label{tab:рез_расчета_q_c0}
\end{table}

Суммарное значение разности потерь реактивной мощности в сопротивлениях линий и зарядной мощности линии (без учета ВЛ 4-5):
\[\sum_{j=1}^{5} = (\Delta Q_{\textup{л}j} - Q_{cj}) = 16,3 - 27,3 = -11,0\; \textup{МВар}\]

По формуле \eqref{eqn:sum_q_treb}:
\[Q_{\textup{треб}\Sigma} = 93,5 + 27,5 - 11,0 = 110\; \textup{МВар}\]

Так как \(Q_{\textup{треб}\Sigma} = 110\; \textup{МВар} > Q_{\textup{расп}\Sigma} = 87\; \textup{МВар}\), то в сети требуется установка дополнительных БСК по условию баланса реактивной мощности.

Суммарная мощность требуемых БСК:
\[Q_{\textup{КУдоп}\Sigma} = Q_{\textup{треб}\Sigma} - Q_{\textup{расп}\Sigma} = 110 - 87 = 23,0\; \textup{МВар}\]

г) Расстановка дополнительных компенсирующих устройств (КУ)
Необходимое число дополнительных компенсирующих устройств:
\[N_\textup{БСК}^{\textup{треб}} = \frac{Q_{\textup{КУ}\Sigma}}{1,2} = \frac{23,0}{1,2} = 19,2 \rightarrow N_\textup{БСК} = 20\]

Из условия равенства коэффициентов реактивной мощности нагрузок на шинах 10 кВ \cite{глазунов_шведов}:
\begin{equation}
	\tg\, \varphi_\textup{б} = \frac{\sum_{i=1}^{n} Q_{\textup{нб}i} - Q_{\textup{КУ}\Sigma}}{\sum_{i=1}^{n} P_{\textup{нб}i}} = 0,296
\end{equation}

Тогда мощность устанавливаемых конденсаторных батарей (сверх установленных по условию не превышения предельных значений коэффициента реактивной мощности) на i-й подстанции:
\[Q_{\textup{БСК}i}^{\textup{треб}} = P_{\textup{нб}i}\cdot (\tg\, \varphi_i - \tg \, \varphi_\textup{б})\]

На примере пункта 1 произведем расчет необходимого количества БСК для дополнительной установки:
\[Q_{\textup{БСК1}}^{\textup{треб}} = 70 \cdot (0,391 - 0,296) = 6,65\; \textup{МВар}\]
\[N_\textup{БСК1}^{треб} = \frac{Q_\textup{БСК1}}{1,2} = \frac{6,65}{1,2} = 5,54 \rightarrow N_\textup{БСК1} = 4\]

\begin{table}[H]
	\small
	\caption{Оценка баланса реактивной мощности с учетом установки дополнительных батарей статических конденсаторов.}
	\begin{tabularx}{\textwidth}{|l|Z|Z|Z|Z|Z|Z|}
		\hline
		№ Пункта                           & 1     & 2     & 3     & 4     & 5     & \(\Sigma\) \\ \hline
		\(P_\textup{нб}\), МВт             & 70    & 70    & 30    & 40    & 35    & 245        \\ \hline
		\(Q_\textup{нб}^{'}\), МВар        & 27,4  & 27,4  & 11,3  & 15,7  & 13,6  & 95,4       \\ \hline
		\(\tg\, \varphi^{'}\)              & 0,391 & 0,391 & 0,377 & 0,393 & 0,389 & -          \\ \hline
		\multicolumn{7}{|l|}{\(\tg\, \varphi_\textup{б} = 0,296\)}                                   \\ \hline
		\(Q_\textup{ку.доп}\), МВар        & 6,65  & 6,65  & 2,43  & 3,84  & 3,255 & 22,8       \\ \hline
		\(N_\textup{БСК.доп}^{\textup{треб}}\) & 5,54  & 5,54  & 2,03  & 3,20  & 2,71  & 19,2       \\ \hline
		\(N_\textup{БСК.доп}\)                 & 4     & 6     & 2     & 4     & 4     & 20         \\ \hline
		\(Q_\textup{БСК.доп}\), МВар       & 4,8   & 7,2   & 2,4   & 4,8   & 4,8   & 24         \\ \hline
		\(Q_\textup{нб}^{''}\), МВар       & 22,6  & 20,2  & 8,9   & 10,9  & 8,8   & 71,4       \\ \hline
		\(S_\textup{нб}^{''}\), МВА        & 73,6  & 72,9  & 31,3  & 41,5  & 36,1  & 255,4      \\ \hline
		\(Q_\textup{прив}\), МВар          & 28,5  & 26,0  & 43,6  & 14,2  & 11,7  & 124        \\ \hline
		\(S_\textup{прив}\), МВар          & 75,6  & 74,7  & 105,2 & 42,4  & 36,9  & 334,8      \\ \hline
	\end{tabularx}
\end{table}

После расстановки на подстанциях проектируемой сети батарей статических конденсаторов по условию выполнения баланса реактивной мощности необходимо определить действительные реактивные и полные нагрузи подстанций \(Q_{\textup{нб}i}^{''}\) и \(S_{\textup{нб}i}^{''}\).

В качестве примера рассчитаем приведенную нагрузку подстанции 1:
\[Q_\textup{бск.доп1} = N_\textup{бск.доп1} \cdot Q_\textup{бк(1)} = 4 \cdot 1,2 = 4,8\; \textup{МВар}\]
\[Q_\textup{нб}^{''} = Q_\textup{нб}^{'} - Q_\textup{бск.доп1} = 27,4 - 4,8 = 22,6\; \textup{МВар}\]
\[S_\textup{нб}^{''} = \sqrt{P_\textup{нб1}^2 + Q_\textup{нб1}^{'2}} = \sqrt{70^2 + 27,4^2} = 73,6\; \textup{МВА}\]
\[Q_\textup{прив1} = Q_\textup{нб1}^{''} + 0,08 \cdot S_\textup{нб1}^{''} = 22,6 + 73,6 \cdot 0,08 = 28,5\; \textup{МВар}\]
\[S_\textup{прив1} = \sqrt{P_\textup{нб1}^2 + Q_\textup{прив1}^2} = \sqrt{70^2 + 28,5^2} = 75,6\; \textup{МВар}\]

Для остальных подстанций 2, 4 и 5 с двухобмоточными трансформаторами расчет выполняется аналогично. Исключение составляет автотрансформаторная подстанция 3, для которой приведенная нагрузка складывается из нагрузок на шинах низшего и среднего напряжения:
\begin{equation*}
	\begin{split}
		Q_\textup{прив3} &= Q_\textup{нб3}^{''} + Q_\textup{прив4} + Q_\textup{прив5} + \\
		&+ 0,08 \sqrt{(P_\textup{нб3} + P_\textup{нб4} + P_\textup{прив5})^2 + (Q_\textup{нб3}^{''} + Q_\textup{прив4} + Q_\textup{прив5})^2} \\
		&= 8,9 + 14,2 + 11,7 + \\
		&+ 0,08 \sqrt{(30 + 40 + 35)^2 + (8,9 + 14,2 + 11,7)^2} = 43,6\; \textup{МВар}
	\end{split}
\end{equation*}

\begin{equation*}
	\begin{split}
		S_\textup{прив3} &= \sqrt{(P_\textup{нб3} + P_\textup{нб4} + P_\textup{нб5})^2 + (Q_\textup{прив3})^2} = \\
		&= \sqrt{(30 + 40 + 35)^2 + (43,6)^2} = 105,2\; \textup{МВА}
	\end{split}
\end{equation*}

\subsection*{Расчет для варианта схемы №2:}

Расчёт для варианта схемы 2 во многом выполняется аналогично варианту схемы 1, поэтому упустим пояснения в некоторых подпунктах

а) Суммарная реактивная нагрузка на шинах 220 кВ источника питания К:
\[k_{\textup{одн}Q} \cdot  \sum_{i=1}^{5} Q_{\textup{нб}i}^{'} = 0,98 \cdot 95,4 = 93,5\; \textup{МВар}\]

б) Суммарные потери реактивной мощности в трансформаторах проектируемой сети:

\[\Delta Q_{\textup{т}\Sigma} = 0,08(1\cdot 75,2 + 2\cdot 75,2 + 1\cdot 32,1 + 2\cdot 43,0 + 2\cdot 37,5) = 33,5\; \textup{МВар}\]

в) Генерация и потери реактивной мощности в ВЛ.

Рассчитаем по формуле \eqref{eqn:зарядная_мощность} зарядную мощность в линии К-1:
\[Q_{cK1} = 2 \cdot 0,14 \cdot 41,8 = 11,7\; \textup{МВар}\]

Потери реактивной мощности в линии по формуле \eqref{eqn:потери_реактивной_мощности_вл}:
\[\Delta Q_\textup{лК1} = \left(\frac{P_\textup{1ц(К1)}}{P_\textup{нат}^{220}}\right)^2 \cdot Q_{cK1} = \left(\frac{125,3}{130}\right)^2 \cdot 11,7 = 10,87\; \textup{МВар}\]

Расчет для остальных линий проводится аналогично. Сведем результаты в таблицу \ref{tab:рез_расчета_q_c0_схема2}.

\begin{table}[H]
	\small
	\caption{Результаты расчета \(q_\textup{c0}\) и \(\Delta Q_\textup{л}\)}
	\begin{tabularx}{\textwidth}{|l|Z|Z|Z|Z|Z|Z|}
		\hline
		Линия                         & К-1   & 1-3  & 3-2  & 3-4  & 4-5  & \(\Sigma\) \\ \hline
		L, км                         & 41,8  & 49,2 & 41,8 & 25,9 & 25,9 & 184,6      \\ \hline
		\(n_\textup{ц}\)              & 2     & 2    & 2    & 2    & 2    & -          \\ \hline
		\(P_\textup{1ц}\), МВт        & 122,5 & 87,5 & 35   & 37,5 & 35   & -          \\ \hline
		\(Q_c\), МВар                 & 11,7  & 13,8 & 3,01 & 1,86 & 1,86 & 32,2       \\ \hline
		\(\Delta Q_\textup{л}\), МВар & 10,9  & 11,7 & 10,2 & 5,10 & 2,91 & 40,8       \\ \hline
	\end{tabularx}
	\label{tab:рез_расчета_q_c0_схема2}
\end{table}

Суммарное значение разности потерь реактивной мощности в сопротивлениях линий и зарядной мощности линии:
\[\sum_{j=1}^{5} = (\Delta Q_{\textup{л}j} - Q_{cj}) = 40,8 - 32,2 = 8,6\; \textup{МВар}\]

По формуле \eqref{eqn:sum_q_treb}:
\[Q_{\textup{треб}\Sigma} = 93,5 + 33,5 + 8,6 = 135,6\; \textup{МВар}\]

Так как \(Q_{\textup{треб}\Sigma} = 135,6\; \textup{МВар} > Q_{\textup{расп}\Sigma} = 87\; \textup{МВар}\), то в сети требуется установка дополнительных БСК по условию баланса реактивной мощности.

Суммарная мощность требуемых БСК:
\[Q_{\textup{КУдоп}\Sigma} = Q_{\textup{треб}\Sigma} - Q_{\textup{расп}\Sigma} = 135,6 - 87 = 48,6\; \textup{МВар}\]

г) Расстановка дополнительных КУ

Из условия равенства коэффициентов реактивной мощности нагрузок на шинах 10 кВ \cite{глазунов_шведов}:
\begin{equation}
	\tg\, \varphi_\textup{б} = \frac{\sum_{i=1}^{n} Q_{\textup{нб}i} - Q_{\textup{КУ}\Sigma}}{\sum_{i=1}^{n} P_{\textup{нб}i}} = 0,191
\end{equation}

Необходимое число дополнительных КУ:\\
\[N_\textup{БСК}^{\textup{треб}} = \frac{Q_{\textup{КУ}\Sigma}}{1,2} = \frac{48,6}{1,2} = 40,5 \rightarrow N_\textup{БСК} = 42\]

Мощность устанавливаемых конденсаторных батарей (сверх установленных по условию не превышения предельных значений коэффициента реактивной мощности) на i-й подстанции:
\[Q_{\textup{БСК}i}^{\textup{треб}} = P_{\textup{нб}i}\cdot (\tg\, \varphi_i - \tg \, \varphi_\textup{б})\]

На примере пункта 1 произведем расчет необходимого количества БСК для дополнительной установки:
\[Q_{\textup{БСК1}}^{\textup{треб}} = 70 \cdot (0,391 - 0,191) = 14\; \textup{МВар}\]
\[N_\textup{БСК1}^{треб} = \frac{Q_\textup{БСК1}}{1,2} = \frac{14}{1,2} = 11,7 \rightarrow N_\textup{БСК1} = 10\]
\[Q_\textup{БСК1} = N_\textup{БСК1} \cdot 1,2 = 10 \cdot 1,2 = 12\; \textup{МВар}\]
\[Q_\textup{нб1}^{''} = Q_\textup{нб1}^{'} - Q_\textup{БСК1} = 27,4 - 12 = 15,4\; \textup{МВар}\]


\begin{table}[H]
	\small
	\caption{Оценка баланса реактивной мощности с учетом установки дополнительных батарей статических конденсаторов.}
	\begin{tabularx}{\textwidth}{|l|Z|Z|Z|Z|Z|Z|}
		\hline
		№ Пункта                           & 1 & 2 & 3 & 4 & 5 & \(\Sigma\) \\ \hline
		\(P_\textup{нб}\), МВт             & 70  & 70  & 30  & 40  & 35  & 245           \\ \hline
		\(Q_\textup{нб}^{'}\), МВар        & 27,4  & 27,4  & 11,3  & 15,7  & 13,6  &  95,4          \\ \hline
		\(\tg\, \varphi^{'}\)              & 0,391  & 0,391  & 0,377  & 0,393  & 0,389  & -           \\ \hline
		\multicolumn{7}{|l|}{\(\tg\, \varphi_\textup{б} = 0,191\)}          \\ \hline
		\(Q_\textup{ку.доп}\), МВар        & 14  & 14  & 5,6 & 8,0 & 6,9  & 48,6         \\ \hline
		\(N_\textup{БСК}^{\textup{треб}}\) & 12  & 14,4  & 7,2  & 9,6  & 7,2  &  50,4          \\ \hline
		\(N_\textup{БСК}\)                 & 11,7  & 11,7  & 4,7  & 6,7  & 5,8  & 40,5           \\ \hline
		\(Q_\textup{БСК.доп}\), МВар       & 10  & 12  & 6  & 8  & 6  & 42           \\ \hline
		\(Q_\textup{нб}^{''}\), МВар       & 15,4 & 13,0  & 4,1  & 6,1  & 6,4  & 45           \\ \hline
		\(S_\textup{нб}^{''}\), МВА        & 71,1  & 71,2  & 30,3  & 40,5  & 35,6  & 249,3           \\ \hline
		\(Q_\textup{прив}\), МВар          & 21,1  & 18,7  & 55,7  & 9,3  & 9,2  & 114           \\ \hline
		\(S_\textup{прив}\), МВар          &  73,1 &  72,5 & 175,2  & 41,1  & 36,2  & 398,1           \\ \hline
	\end{tabularx}
\end{table}

После расстановки на подстанциях проектируемой сети батарей статических конденсаторов по условию выполнения баланса реактивной мощности необходимо определить действительные реактивные и полные нагрузи подстанций \(Q_{\textup{нб}i}^{''}\) и \(S_{\textup{нб}i}^{''}\).

В качестве примера рассчитаем приведенную нагрузку подстанции 1:
\[Q_\textup{БСК1} = N_\textup{БСК1} \cdot 1,2 = 10 \cdot 1,2 = 12\; \textup{МВар}\]
\[Q_\textup{нб1}^{''} = Q_\textup{нб1}^{'} - Q_\textup{БСК1} = 27,4 - 12 = 15,4\; \textup{МВар}\]
\[S_\textup{нб}^{''} = \sqrt{P_\textup{нб1}^2 + Q_\textup{нб1}^{''}} = \sqrt{70^2 + 15,4^2} = 71,7\; \textup{МВА}\]
\[Q_\textup{прив1} = Q_\textup{нб1}^{''} + 0,08 \cdot S_\textup{нб1}^{''} = 15,4 + 71,1 \cdot 0,08 = 21,1\; \textup{МВар}\]
\[S_\textup{прив1} = \sqrt{P_\textup{нб1}^2 + Q_\textup{прив1}^2} = \sqrt{70^2 + 21,1^2} = 73,1\; \textup{МВар}\]

Для остальных подстанций 2, 4 и 5 с двухобмоточными трансформаторами расчет выполняется аналогично. Исключение составляет автотрансформаторная подстанция 3, для которой приведенная нагрузка складывается из нагрузок на шинах низшего и среднего напряжения:
\begin{equation*}
	\begin{split}
		&Q_\textup{прив3} = Q_\textup{нб3}^{''} + Q_\textup{прив4} + Q_\textup{прив5} + Q_\textup{прив2} + 0,08 \times\\
		&\times \sqrt{(P_\textup{нб3} + P_\textup{нб4} + P_\textup{нб5} + P_\textup{нб2})^2 + (Q_\textup{нб3}^{''} + Q_\textup{прив4} + Q_\textup{прив5} + Q_\textup{прив2})^2} =\\
		&= 4,1 + 9,3 + 9,2 + 18,7 +\\
		&+ 0,08 \sqrt{(30 + 40 + 35 + 70)^2 + (4,1 + 9,3 + 9,2 + 18,7)^2} = 55,7\; \textup{МВар}
	\end{split}
\end{equation*}

\begin{equation*}
	\begin{split}
		S_\textup{прив3} &= \sqrt{(P_\textup{нб3} + P_\textup{нб4} + P_\textup{нб5} + P_\textup{нб2})^2 + (Q_\textup{прив3})^2} = \\
		&= \sqrt{(30 + 40 + 35 + 70)^2 + (55,7)^2} = 175,2\; \textup{МВА}
	\end{split}
\end{equation*}

%%% Local Variables:
%%% mode: latex
%%% TeX-master: "rpz"
%%% End: