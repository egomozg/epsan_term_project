\chapter{Оценка баланса реактивной мощности в проектируемой сети}
\label{cha:ocenka_brm}

На основе оптимизационных расчетов распределения реактивной мощности в электроэнергетической системе для каждого ее узла определяется реактивная мощность, которую целесообразно передавать из электроэнергетической системы в распределительные сети, питающиеся от того или иного узла. Поэтому при проектировании электрической сети, получающей питание от подстанций электроэнергетической системы, задается реактивная мощность \(Q_{\textup{расп}\Sigma}\), которую целесообразно потреблять из системы (в заданном узле присоединения) в режиме наибольших нагрузок, или же коэффициент реактивной мощности. Потребление большей реактивной мощности приведет к дополнительным затратам на передачу этой мощности и, следовательно, к отступлению от оптимального режима питающей системы. В связи с этим, в проекте оценивается выполнение баланса реактивной мощности в проектируемой сети и при необходимости устанавливаются компенсирующие устройства.

\section{Проверка по условию выполнения баланса реактивной мощности и расстановка дополнительных батарей статических конденсаторов для варианта схемы сети 1}

Потребление реактивной мощности в проектируемой сети в период наибольших нагрузок складывается из заданных реактивных нагрузок в пунктах потребления, рассчитанных с учетом предварительного этапа установки БСК с целью выполнения условия \(\tg\,{\varphi_i} \le \tg\,{\varphi_\textup{пред}}\) и потерь реактивной мощности в линиях и понижающих трансформаторах с учетом зарядных мощностей линий. При определении одновременно потребляемой реактивной мощности следует учитывать несовпадение по времени суток наибольших нагрузок отдельных потребителей.

При четырех и более пунктах потребления среднестатистическое значение коэффициента одновременности реактивных нагрузок на шинах 220 кВ источника питания составляет 0,98 \cite{глазунов_шведов}. Наибольшая суммарная реактивная мощность потребляемая в проектируемой сети в период наибольших нагрузок рассчитывается по формуле:
\begin{eqndesc}[H]
	\begin{equation}
		Q_{\textup{треб}\Sigma} = k_{\textup{одн}Q} \cdot \sum_{i=1}^{n} Q_{\textup{нб}i} + \Delta Q_{\textup{т}\Sigma} + \sum_{j=1}^{m}\left(\Delta Q_{\textup{л}j} - Q_{cj}\right)
		\label{eqn:sum_q_treb}
	\end{equation}
где \(Q_{\textup{нб}i}\) "--- наибольшая реактивная нагрузка \textit{i}-го пункта с учетом установленных конденсаторных батарей по условию не превышения предельных значений коэффициента реактивной мощности;
\textit{n} "--- число пунктов потребления электроэнергии;
\(k_{\textup{одн}Q}\) "--- коэффициент одновременности наибольших реактивных нагрузок подстанций;
\(\Delta Q_{\textup{т}\Sigma}\) "--- суммарные потери в трансформаторах подстанций сети;
\(Q_{\textup{л}j}\) "--- потери реактивной мощности в \textit{j}-й линии электропередачи;
\(Q_{cj}\) "--- зарядная мощность \textit{j}-й линии электропередачи;
\textit{m} "--- число линий электропередачи в сети.
\end{eqndesc}

Для удобства рассчитаем составляющие формулы \eqref{eqn:sum_q_treb} по-отдельности

а) \textit{Суммарная реактивная нагрузка на шинах 220 кВ источника питания К}:
\[k_{\textup{одн}Q} \cdot  \sum_{i=1}^{5} Q_{\textup{нб}i}^{'} = 0,98 \cdot 95,4 = 93,5\; \textup{МВар}\]

Суммарная реактивная нагрузка в пунктах потребления \(Q_{\textup{нб}i}\) берется из табл. \ref{tab:первичная_компенсация}.

б) \textit{Суммарные потери реактивной мощности в трансформаторах проектируемой сети}

В электрических сетях номинальным напряжением до 220 кВ основным типом подстанций являются подстанции с двухобмоточными трансформаторами, для которых при двух параллельно включенных трансформаторах и коэффициенте аварийной перегрузки 1,4 потери реактивной мощности приближенно оцениваются в размере 8 \% от полной нагрузки подстанции \(S_\textup{нб}\) \cite{глазунов_шведов}.

Мощность нагрузки \textit{i}-й подстанции на пути от источника питания проходит через несколько трансформаций. Если считать, что на каждой из них теряется 8 \% от полной мощности этой нагрузки, то можно оценить суммарные потери реактивной мощности в трансформаторах подстанций сети следующим образом:
\begin{eqndesc}[h]
\[
\Delta Q_{\textup{т}\Sigma} = \sum_{i=1}^{n} \Delta Q_{\textup{т}i} \cong 0,08 \sum_{i=1}^{n} m_{\textup{т}i} \cdot S_{\textup{нб}i},\]
где \(m_{\textup{т}i}\) "--- число трансформации \textit{i}-й ПС.
\end{eqndesc}
\[
%\begin{split}
\Delta Q_{\textup{т}\Sigma} \cong 0,08 \sum_{i=1}^{n} m_{\textup{т}i} \cdot S_{\textup{нб}i} =\] \[= 0,08(1\cdot 75,2 + 1\cdot 75,2 + 1\cdot 32,1 + 2\cdot 43,0 + 2\cdot 37,5) = 27,5\; \textup{МВар}
%\end{split}
\]

в) \textit{Генерация и потери реактивной мощности в воздушной линии (ВЛ):}

Зарядная мощность в линии рассчитывается по формуле:
\begin{eqndesc}[h]
\begin{equation}
	Q_c = n_\textup{ц} \cdot q_{c0} \cdot L,
	\label{eqn:зарядная_мощность}
\end{equation}
где \(n_\textup{ц}\) "--- количество цепей линии;
\(q_{c0}\) "--- удельная генерация зарядной мощности, \(\frac{\textup{МВар}}{\textup{км}}\);
\(q_\textup{с0}^{220} = 0,14\; \frac{\textup{МВар}}{\textup{км}}\) - для ВЛ 220 кВ, \(q_{c0}^{110} = 0,036\; \frac{\textup{МВар}}{\textup{км}}\) - для ВЛ 110 кВ \cite{глазунов_шведов}.
\end{eqndesc}

В качестве примера рассчитаем по формуле \eqref{eqn:зарядная_мощность} зарядную мощность в линии К-1:
\[Q_{cK1} = 1 \cdot 0,14 \cdot 41,8 = 5,85\; \textup{МВар}\]

Для оценки потерь реактивной мощности в ВЛ, воспользуемся известным соотношением \cite{глазунов_шведов}:
\begin{eqndesc}[h]
	\begin{equation}
		\frac{\Delta Q_\textup{л}}{Q_c} \cong \left(\frac{P_\textup{л}}{P_\textup{нат}\cdot n_\textup{ц}}\right)^2 \rightarrow \Delta Q_\textup{л} \cong \left(\frac{P_\textup{1ц}}{P_\textup{нат}}\right)^2 \Delta Q_c,
		\label{eqn:потери_реактивной_мощности_вл}
	\end{equation}
где \(P_\textup{нат}\) "--- натуральная мощность при передаче которой по одной цепи линии, потери реактивной мощности в сопротивлении линии равны зарядной мощности в линии.
\end{eqndesc}

Натуральная мощность для ВЛ 220 и 110 кВ \cite{пуэ7}:
\[P_\textup{нат}^{220} = 130\; \textup{МВт};\]
\[P_\textup{нат}^{110} = 30\; \textup{МВт}\]

Для ВЛ 110 кВ с \(P_\textup{1ц} \leq P_\textup{нат}^{110} = 30\; \textup{кВ}\) допускается принять \(\Delta Q_\textup{л} \approx Q_c\). В данном случае можно исключить линию 4-5, так как \(P_\textup{1ц(45)} = 17,5\; \textup{МВт}\leq P_\textup{нат}^{110} = 30\; \textup{МВт}\).

Оценим потери реактивной мощности в сопротивлении линии K1 вычислим по формуле \eqref{eqn:потери_реактивной_мощности_вл}:
\[\Delta Q_\textup{лК1} = \left(\frac{P_\textup{1ц(К1)}}{P_\textup{нат}^{220}}\right)^2 \cdot Q_{cK1} = \left(\frac{125,3}{130}\right)^2 \cdot 5,85 = 5,43\; \textup{МВар}\]

Расчет для остальных линий проводится аналогично. Сведем результаты в таблицу \ref{tab:рез_расчета_q_c0}.

\begin{table}[H]
	\small
	\caption{Результаты расчета зарядной мощности и потерь реактивной мощности в линиях электропередачи для варианта схемы сети 1}
	\begin{tabularx}{\textwidth}{|l|Z|Z|Z|Z|Z|Z|}
		\hline
		Линия                  & К-1   & К-2   & 1-3  & 2-3  & 3-4  & \(\Sigma\) \\ \hline
		\textit{L}, км                  & 41,8  & 49,2  & 49,2 & 41,8 & 25,9 & 207,9          \\ \hline
		\(n_\textup{ц}\)       & 1     & 1     & 1    & 1    & 2    & -          \\ \hline
		\(P_\textup{1ц}\), МВт & 125,3 & 119,7 & 55,3 & 49,7 & 37,5 & -        \\ \hline
		\(Q_c\), МВар          & 5,85  & 6,89  & 6,89 & 5,85 & 1,86 & 27,3       \\ \hline
		\(\Delta Q_\textup{л}\), МВар & 5,43 & 5,84 & 1,25 & 0,86 & 2,91 & 16,3 \\ \hline
	\end{tabularx}
	\label{tab:рез_расчета_q_c0}
\end{table}

Суммарное значение разности потерь реактивной мощности в сопротивлениях линий и зарядной мощности линий (без учета ВЛ 4-5):
\[\sum_{j=1}^{5}(\Delta Q_{\textup{л}j} - Q_{cj}) = 16,3 - 27,3 = -11,0\; \textup{МВар}\]

По формуле \eqref{eqn:sum_q_treb} определим суммарную реактивную мощность, потребляемую в проектируемой сети:
\[Q_{\textup{треб}\Sigma} = k_{\textup{одн}Q} \cdot \sum_{i=1}^{5} Q_{\textup{нб}i} + \Delta Q_{\textup{т}\Sigma} + \sum_{j=1}^{5}\left(\Delta Q_{\textup{л}j} - Q_{cj}\right) =\] \[= 93,5 + 27,5 - 11,0 = 110\; \textup{МВар}\]

Так как \(Q_{\textup{треб}\Sigma} = 110\; \textup{МВар} > Q_{\textup{расп}\Sigma} = 87\; \textup{МВар}\), то в сети требуется установка дополнительных БСК по условию баланса реактивной мощности.

Суммарная мощность требуемых БСК:
\[Q_{\textup{КУдоп}\Sigma} = Q_{\textup{треб}\Sigma} - Q_{\textup{расп}\Sigma} = 110 - 87 = 23,0\; \textup{МВар}\]

\subsection*{Расстановка дополнительных компенсирующих устройств}
Произведем расстановку батарей статических конденсаторов на шинах 10 кВ подстанций проектируемой сети с учетом следующих рекомендаций \cite{глазунов_шведов}:
\begin{enumerate}
	\item В электрических сетях двух и более номинальных напряжений (например, 220/110 кВ) следует в первую очередь устанавливать компенсирующие устройства на шинах 10 кВ подстанций сети более низкого номинального напряжения (например, 110 кВ)
	\item В сети одного номинального напряжения необходимо в первую очередь компенсировать реактивную мощность на наиболее электрически удаленных подстанция (по активному сопротивлению сети) вплоть до полной компенсации реактивной нагрузки подстанции.
	\item При незначительной разнице в электрической удаленности подстанций от источника питания в сети одного номинального напряжения расстановка компенсирующих устройств может производиться по условию равенства коэффициентов реактивной мощности нагрузок на шинах 10 кВ, удовлетворяющему условию выполнения баланса реактивной мощности в проектируемой сети:
	\begin{equation}
		\tg\, \varphi_\textup{б} = \frac{\sum_{i=1}^{n} Q_{\textup{нб}i} - Q_{\textup{КУ}\Sigma}}{\sum_{i=1}^{n} P_{\textup{нб}i}},
		\label{eqn:тангенс_баланса}
	\end{equation}
	где \(Q_{\textup{нб}i}\) "--- действительные реактивные нагрузки подстанций с учетом мощности установленных конденсаторных батарей.
\end{enumerate}

Тогда мощность устанавливаемых конденсаторных батарей (сверх установленных по условию не превышения предельных значений коэффициента реактивной мощности) на \textit{i}-й подстанции:
\begin{equation}
Q_{\textup{БСК}i}^{\textup{треб}} = P_{\textup{нб}i}\cdot (\tg\, \varphi_i - \tg \, \varphi_\textup{б})
\label{eqn:Q_бск_треб}
\end{equation}

Необходимое число дополнительных компенсирующих устройств:
\[N_\textup{БСК}^{\textup{треб}} = \frac{Q_{\textup{КУдоп}\Sigma}}{1,2} = \frac{23,0}{1,2} = 19,2 \rightarrow N_\textup{БСК} = 20\]

Из условия равенства коэффициентов реактивной мощности нагрузок на шинах 10 кВ \eqref{eqn:тангенс_баланса}:
\[\tg\, \varphi_\textup{б} = \frac{\sum_{i=1}^{5} Q_{\textup{нб}i}^{'} - Q_{\textup{КУдоп}\Sigma}}{\sum_{i=1}^{5} P_{\textup{нб}i}} =\]  \[ = \frac{95,4 - 23,0}{245} = 0,296\]

На примере пункта 1 произведем расчет необходимого количества БСК для дополнительной установки по формуле \eqref{eqn:Q_бск_треб}:
\[Q_{\textup{БСК1}}^{\textup{треб}} = P_{\textup{нб}1}\cdot (\tg\, \varphi_1 - \tg \, \varphi_\textup{б}) = 70 \cdot (0,391 - 0,296) = 6,65\; \textup{МВар}\]
\[N_\textup{БСК1}^\textup{треб} = \frac{Q_\textup{БСК1}^\textup{треб}}{1,2} = \frac{6,65}{1,2} = 5,54 \rightarrow N_\textup{БСК1} = 6\]

Результаты расчета для варианта схемы сети 1 сведем в таблицу \ref{tab:нагрузки_с_батареями_кольцо}

\begin{table}[H]
	\small
	\caption{Оценка баланса реактивной мощности с учетом установки дополнительных батарей статических конденсаторов}
	\label{tab:нагрузки_с_батареями_кольцо}
	\begin{tabularx}{\textwidth}{|l|Z|Z|Z|Z|Z|Z|}
		\hline
		№ Пункта                               & 1     & 2     & 3     & 4     & 5     & \(\Sigma\) \\ \hline
		\(P_\textup{нб}\), МВт                 & 70    & 70    & 30    & 40    & 35    & 245        \\ \hline
		\(Q_\textup{нб}^{'}\), МВар            & 27,4  & 27,4  & 11,3  & 15,7  & 13,6  & 95,4       \\ \hline
		\(\tg\, \varphi^{'}\)                  & 0,391 & 0,391 & 0,377 & 0,393 & 0,389 & -          \\ \hline
		\multicolumn{7}{|l|}{\(\tg\, \varphi_\textup{б} = 0,296\)}                                  \\ \hline
		\(Q_\textup{БСК}^\textup{треб}\), МВар            & 6,65  & 6,65  & 2,50  & 3,90  & 3,30  & 23,0       \\ \hline
		\(N_\textup{БСК}^{\textup{треб}}\) & 5,54  & 5,54  & 2,08  & 3,25  & 2,75  & 19,2       \\ \hline
		\(N_\textup{БСК}\)                 & 6     & 6     & 2     & 4     & 2     & 20         \\ \hline
		\(Q_\textup{БСК.доп}\), МВар           & 7,2   & 7,2   & 2,4   & 4,8   & 2,4   & 24         \\ \hline
		\(Q_\textup{нб}^{''}\), МВар           & 20,2  & 20,2  & 8,9   & 10,9  & 11,2  & 71,4       \\ \hline
		\(S_\textup{нб}^{''}\), МВА            & 72,9  & 72,9  & 31,3  & 41,5  & 36,7  & -          \\ \hline
		\(Q_\textup{прив}\), МВар              & 26,0  & 26,0  & 46,1  & 14,2  & 14,1  & -          \\ \hline
		\(S_\textup{прив}\), МВар              & 74,7  & 74,7  & 114,7 & 42,4  & 37,7  & -          \\ \hline
	\end{tabularx}
\end{table}

После расстановки на подстанциях проектируемой сети батарей статических конденсаторов по условию выполнения баланса реактивной мощности необходимо определить действительные реактивные и полные нагрузи подстанций \(Q_{\textup{нб}i}^{''}\) и \(S_{\textup{нб}i}^{''}\).

В качестве примера рассчитаем приведенную нагрузку подстанции 1:
\[Q_\textup{БСК.доп1} = N_\textup{БСК1} \cdot Q_\textup{бк(1)} = 6 \cdot 1,2 = 7,2\; \textup{МВар}\]
\[Q_\textup{нб1}^{''} = Q_\textup{нб1}^{'} - Q_\textup{БСК1} = 27,4 - 7,2 = 20,2\; \textup{МВар}\]
\[S_\textup{нб1}^{''} = \sqrt{P_\textup{нб1}^2 + Q_\textup{нб1}^{''2}} = \sqrt{70^2 + 20,2^2} = 72,9\; \textup{МВА}\]
\[Q_\textup{прив1} = Q_\textup{нб1}^{''} + 0,08 \cdot S_\textup{нб1}^{''} = 20,2 + 72,9 \cdot 0,08 = 26,0\; \textup{МВар}\]
\[S_\textup{прив1} = \sqrt{P_\textup{нб1}^2 + Q_\textup{прив1}^2} = \sqrt{70^2 + 26,0^2} = 74,7\; \textup{МВар}\]

Для остальных подстанций 2, 4 и 5 с двухобмоточными трансформаторами расчет выполняется аналогично. Исключение составляет автотрансформаторная подстанция 3, для которой приведенная нагрузка складывается из нагрузок на шинах низшего и среднего напряжения:
\begin{equation*}
	\begin{split}
		Q_\textup{прив3} &= Q_\textup{нб3}^{''} + Q_\textup{прив4} + Q_\textup{прив5} + \\
		&+ 0,08 \sqrt{(P_\textup{нб3} + P_\textup{нб4} + P_\textup{прив5})^2 + (Q_\textup{нб3}^{''} + Q_\textup{прив4} + Q_\textup{прив5})^2} =\\
		&= 8,9 + 14,2 + 14,1 + \\
		&+ 0,08 \sqrt{(30 + 40 + 35)^2 + (8,9 + 14,2 + 14,1)^2} = 46,1\; \textup{МВар}
	\end{split}
\end{equation*}

\begin{equation*}
	\begin{split}
		S_\textup{прив3} &= \sqrt{(P_\textup{нб3} + P_\textup{нб4} + P_\textup{нб5})^2 + (Q_\textup{прив3})^2} = \\
		&= \sqrt{(30 + 40 + 35)^2 + (46,1)^2} = 114,7\; \textup{МВА}
	\end{split}
\end{equation*}

\section{Проверка по условию выполнения баланса реактивной мощности и расстановка дополнительных батарей статических конденсаторов для варианта схемы сети 2}

Расчёт для варианта схемы сети 2 во многом выполняется аналогично варианту схемы сети 1, поэтому упустим пояснения в некоторых подпунктах.

а) \textit{Суммарная реактивная нагрузка на шинах 220 кВ источника питания К:}
\[k_{\textup{одн}Q} \cdot  \sum_{i=1}^{5} Q_{\textup{нб}i}^{'} = 0,98 \cdot 95,4 = 93,5\; \textup{МВар}\]

б) \textit{Суммарные потери реактивной мощности в трансформаторах проектируемой сети:}

\[\Delta Q_{\textup{т}\Sigma} \cong 0,08 \sum_{i=1}^{5} m_{\textup{т}i} \cdot S_{\textup{нб}i} =\] \[= 0,08(1\cdot 75,2 + 2\cdot 75,2 + 1\cdot 32,1 + 2\cdot 43,0 + 2\cdot 37,5) = 33,5\; \textup{МВар}\]

в) \textit{Генерация и потери реактивной мощности в ВЛ}

Рассчитаем по формуле \eqref{eqn:зарядная_мощность} зарядную мощность в линии К-1:
\[Q_{cK1} = n_\textup{ц.К1}\cdot q_{c0}^{220} L_\textup{К1} =  2 \cdot 0,14 \cdot 41,8 = 11,7\; \textup{МВар}\]

Потери реактивной мощности в сопротивлении линии К1 вычислим по формуле \eqref{eqn:потери_реактивной_мощности_вл}:
\[\Delta Q_\textup{лК1} = \left(\frac{P_\textup{1ц(К1)}}{P_\textup{нат}^{220}}\right)^2 \cdot Q_{cK1} = \left(\frac{122,5}{130}\right)^2 \cdot 11,7 = 10,4\; \textup{МВар}\]

Расчет для остальных линий проводится аналогично. Сведем результаты в таблицу \ref{tab:рез_расчета_q_c0_схема2}.

\begin{table}[H]
	\small
	\caption{Результаты расчета зарядной мощности и потерь реактивной мощности в линиях электропередачи для варианта схемы сети 2}
	\begin{tabularx}{\textwidth}{|l|Z|Z|Z|Z|Z|}
		\hline
		Линия                         & К-1   & 1-3  & 3-2  & 3-4  & \(\Sigma\) \\ \hline
		\textit{L}, км                         & 41,8  & 49,2 & 41,8 & 25,9 & 184,6      \\ \hline
		\(n_\textup{ц}\)              & 2     & 2    & 2    & 2    & -          \\ \hline
		\(P_\textup{1ц}\), МВт        & 122,5 & 87,5 & 35   & 37,5 & 300        \\ \hline
		\(Q_c\), МВар                 & 11,7  & 13,8 & 3,01 & 1,86 & 30,4       \\ \hline
		\(\Delta Q_\textup{л}\), МВар & 10,4  & 6,24 & 4,10 & 2,91 & 23,7       \\ \hline
	\end{tabularx}
	\label{tab:рез_расчета_q_c0_схема2}
\end{table}

Суммарное значение разности потерь реактивной мощности в сопротивлениях линий и зарядной мощности линии (без учета ВЛ 4-5):
\[\sum_{j=1}^{4} (\Delta Q_{\textup{л}j} - Q_{cj}) = 23,7 - 30,4 = -6,70\; \textup{МВар}\]

По формуле \eqref{eqn:sum_q_treb} определим суммарную реактивную мощность, потребляемую в проектируемой сети::
\[Q_{\textup{треб}\Sigma} = k_{\textup{одн}Q} \cdot \sum_{i=1}^{5} Q_{\textup{нб}i} + \Delta Q_{\textup{т}\Sigma} + \sum_{j=1}^{4}\left(\Delta Q_{\textup{л}j} - Q_{cj}\right) =\] \[= 93,5 + 33,5 - 6,70 = 120,3\; \textup{МВар}\]

Так как \(Q_{\textup{треб}\Sigma} = 120,3\; \textup{МВар} > Q_{\textup{расп}\Sigma} = 87\; \textup{МВар}\), то в сети требуется установка дополнительных БСК по условию баланса реактивной мощности.

Суммарная мощность требуемых БСК:
\[Q_{\textup{КУдоп}\Sigma} = Q_{\textup{треб}\Sigma} - Q_{\textup{расп}\Sigma} = 120,3 - 87 = 33,3\; \textup{МВар}\]

\subsection*{Расстановка дополнительных компенсирующих устройств}

Из условия равенства коэффициентов реактивной мощности нагрузок на шинах 10 кВ ПС \eqref{eqn:тангенс_баланса}:
\[\tg\, \varphi_\textup{б} = \frac{\sum_{i=1}^{5} Q_{\textup{нб}i}^{'} - Q_{\textup{КУдоп}\Sigma}}{\sum_{i=1}^{5} P_{\textup{нб}i}} = \frac{95,4 - 33,3}{245} = 0,253\]

Необходимое число дополнительных КУ:
\[N_\textup{БСК}^{\textup{треб}} = \frac{Q_{\textup{КУдоп}\Sigma}}{1,2} = \frac{33,3}{1,2} = 27,8 \rightarrow N_\textup{БСК} = 28\]

На примере пункта 1 произведем расчет необходимого количества БСК для дополнительной установки:
\[Q_{\textup{БСК1}}^{\textup{треб}} = P_{\textup{нб}1}\cdot (\tg\, \varphi_1 - \tg \, \varphi_\textup{б}) = 70 \cdot (0,391 - 0,253) = 9,7\; \textup{МВар}\]
\[N_\textup{БСК1}^{\textup{треб}} = \frac{Q_\textup{БСК1}^\textup{треб}}{1,2} = \frac{9,7}{1,2} = 8,1 \rightarrow N_\textup{БСК1} = 8\]
\[Q_\textup{БСК1} = N_\textup{БСК1} \cdot 1,2 = 8 \cdot 1,2 = 9,6\; \textup{МВар}\]
\[Q_\textup{нб1}^{''} = Q_\textup{нб1}^{'} - Q_\textup{БСК1} = 27,4 - 9,6 = 17,8\; \textup{МВар}\]

Результаты расчета для варианта схемы сети 2 сведем в таблицу \ref{tab:брм_батареи_2}

\begin{table}[h]
	\small
	\caption{Оценка баланса реактивной мощности с учетом установки дополнительных батарей статических конденсаторов.}
	\label{tab:брм_батареи_2}
	\begin{tabularx}{\textwidth}{|l|Z|Z|Z|Z|Z|Z|}
		\hline
		№ Пункта                           & 1     & 2     & 3     & 4     & 5     & \(\Sigma\) \\ \hline
		\(P_\textup{нб}\), МВт             & 70    & 70    & 30    & 40    & 35    & 245        \\ \hline
		\(Q_\textup{нб}^{'}\), МВар        & 27,4  & 27,4  & 11,3  & 15,7  & 13,6  & 95,4       \\ \hline
		\(\tg\, \varphi^{'}\)              & 0,391 & 0,391 & 0,377 & 0,393 & 0,389 & -          \\ \hline
		\multicolumn{7}{|l|}{\(\tg\, \varphi_\textup{б} = 0,253\)}                              \\ \hline
		\(Q_\textup{БСК}^\textup{треб}\), МВар   & 9,7   & 9,7   & 3,5   & 5,6   & 4,8   & 33,3       \\ \hline
		\(N_\textup{БСК}^{\textup{треб}}\) & 8,1   & 8,1   & 2,9   & 4,7   & 4,0   & 27,8       \\ \hline
		\(N_\textup{БСК}\)                 & 8     & 8     & 2     & 6     & 4     & 28         \\ \hline
		\(Q_\textup{БСК.доп}\), МВар       & 9,6   & 9,6   & 2,4   & 7,2   & 4,8   & 33,6       \\ \hline
		\(Q_\textup{нб}^{''}\), МВар       & 17,8  & 17,8  & 8,9   & 8,5   & 8,8   & 61,8       \\ \hline
		\(S_\textup{нб}^{''}\), МВА        & 72,2  & 72,2  & 31,3  & 40,9  & 36,1  & -          \\ \hline
		\(Q_\textup{прив}\), МВар          & 23,6  & 23,6  & 70,7  & 11,8  & 11,7  & -          \\ \hline
		\(S_\textup{прив}\), МВар          & 73,9  & 73,9  & 188,7 & 41,7  & 36,9  & -          \\ \hline
	\end{tabularx}
\end{table}

В качестве примера рассчитаем приведенную нагрузку подстанции 1:
\[Q_\textup{БСК.доп1} = N_\textup{БСК1} \cdot 1,2 = 8 \cdot 1,2 = 9,6\; \textup{МВар}\]
\[Q_\textup{нб1}^{''} = Q_\textup{нб1}^{'} - Q_\textup{БСК.доп1} = 27,4 - 9,6 = 17,8\; \textup{МВар}\]
\[S_\textup{нб}^{''} = \sqrt{P_\textup{нб1}^2 + Q_\textup{нб1}^{''2}} = \sqrt{70^2 + 17,8^2} = 72,2\; \textup{МВА}\]
\[Q_\textup{прив1} = Q_\textup{нб1}^{''} + 0,08 \cdot S_\textup{нб1}^{''} = 17,8 + 72,2 \cdot 0,08 = 23,6\; \textup{МВар}\]
\[S_\textup{прив1} = \sqrt{P_\textup{нб1}^2 + Q_\textup{прив1}^2} = \sqrt{70^2 + 23,6^2} = 73,9\; \textup{МВар}\]

Для остальных подстанций 2, 4 и 5 с двухобмоточными трансформаторами расчет выполняется аналогично. Исключение составляет автотрансформаторная подстанция 3, для которой приведенная нагрузка складывается из нагрузок на шинах низшего и среднего напряжения:
\begin{equation*}
	\begin{split}
		&Q_\textup{прив3} = Q_\textup{нб3}^{''} + Q_\textup{прив4} + Q_\textup{прив5} + Q_\textup{прив2} + 0,08 \times\\
		&\times \sqrt{(P_\textup{нб3} + P_\textup{нб4} + P_\textup{нб5} + P_\textup{нб2})^2 + (Q_\textup{нб3}^{''} + Q_\textup{прив4} + Q_\textup{прив5} + Q_\textup{прив2})^2} =\\
		&= 8,9 + 11,8 + 11,7 + 23,6 +\\
		&+ 0,08 \sqrt{(30 + 40 + 35 + 70)^2 + (8,9 + 11,8 + 11,7 + 23,6)^2} = 70,7\; \textup{МВар}
	\end{split}
\end{equation*}

\begin{equation*}
	\begin{split}
		S_\textup{прив3} &= \sqrt{(P_\textup{нб3} + P_\textup{нб4} + P_\textup{нб5} + P_\textup{нб2})^2 + (Q_\textup{прив3})^2} = \\
		&= \sqrt{(30 + 40 + 35 + 70)^2 + (70,7)^2} = 188,7\; \textup{МВА}
	\end{split}
\end{equation*}

%%% Local Variables:
%%% mode: latex
%%% TeX-master: "rpz"
%%% End: