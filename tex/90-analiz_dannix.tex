\chapter{Расчет и анализ основных режимов работы спроектированной сети}
\label{cha:rastr_win}

В курсовом проекте необходимо рассмотреть четыре установившихся режима: наибольших нагрузок (НБ), наименьших нагрузок (НМ), послеаварийные (П/АВ) режимы в сетях напряжением 220 кВ и 110 кВ. Расчет проводится в программно-вычислительном комплексе (далее ПВК) RastrWin3.

В режиме наибольших нагрузок и послеаварийных режимах напряжение на шинах источника питания:
\[U_\textup{ИПК}^\textup{нб} = 1,1\cdot U_\textup{ном} = 1,1\cdot 220 = 242\; \textup{кВ}\]

В режиме наименьших нагрузок напряжение на шинах источника питания:
\[U_\textup{ИПК}^\textup{нм} = 1,0\cdot 220 = 220\; \textup{кВ}\]

Параметры узлов, ветвей и регулировочных устройств задаются в Приложении (А) в соответствии с приведенными ниже исходными данными.

В данном разделе в табл. \ref{tab:данные_по_нагрузкам_нб} "--- \ref{tab:установленные_бск} приводятся исходные данные для расчёта режимов проектируемой электрической сети.

\begin{table}
	\small
	\caption{Данные по нагрузкам пунктов в режиме наибольших нагрузок}
	\label{tab:данные_по_нагрузкам_нб}
	\begin{tabularx}{\linewidth}{|Z|Z|Z|Z|Z|Z|}
		\hline
		№ ПС & 1 & 2 & 3 & 4 & 5 \\ \hline
		\(P_\textup{нб}\), МВт & 70 & 70 & 30 & 40 & 35 \\ \hline
		\(Q_\textup{нб}\), МВар & 29,8 & 29,8 & 13,7 & 20,5 & 16,0 \\ \hline
	\end{tabularx}
\end{table}

Согласно исходным данным курсового проекта, нагрузка пунктов в режиме наименьших нагрузок принимается равной 37 \% от наибольшей. Данные по нагрузкам пунктов в режиме наименьших нагрузок приведены в табл. \ref{tab:данные_по_нагрузкам_нм}.

\begin{table}[H]
	\small
	\caption{Данные по нагрузкам пунктов в режиме наименьших нагрузок нагрузок}
	\label{tab:данные_по_нагрузкам_нм}
	\begin{tabularx}{\linewidth}{|Z|Z|Z|Z|Z|Z|}
		\hline
		№ ПС & 1 & 2 & 3 & 4 & 5 \\ \hline
		\(P_\textup{нб}\), МВт & 25,9 & 25,9 & 11,1 & 14,8 & 13,0 \\ \hline
		\(Q_\textup{нб}\), МВар & 11,0 & 11,0 & 5,07 & 7,59 & 5,92 \\ \hline
	\end{tabularx}
\end{table}

\begin{table}[H]
	\small
	\caption{Номинальные напряжения и коэффициенты трансформации двухобмоточных трансформаторов}
	\label{tab:кэффы_транса}
	\begin{tabularx}{\linewidth}{|>{\hsize=0.5\hsize\linewidth=\hsize}Z|>{\hsize=1.5\hsize\linewidth=\hsize}Z|Z|Z|Z|}
		\hline
		№ ПС & Тип трансформатора & \(U_\textup{ном}^\textup{ВН}\), кВ & \(U_\textup{ном}^\textup{НН}\), кВ & \(k_{\textup{т.ном}}\) \\ \hline
		1 & ТРДН-63000/220 & 230 & 11,0 & 0,0478 \\ \hline
		2 & ТРДН-63000/220 & 230 & 11,0 & 0,0478 \\ \hline
		4 & ТРДН-40000/110 & 115 & 10,5 & 0,0913 \\ \hline
		5 & ТРДН-25000/110 & 115 & 10,5 & 0,0913 \\ \hline
	\end{tabularx}
\end{table}

\begin{table}[H]
	\small
	\caption{Каталожные данные двухобмоточных трансформаторов}
	\label{tab:кат_данные_двухобм_трансов}
	\begin{tabularx}{\linewidth}{|Z|Z|Zl|Z|Z|Z|}
		\hline
		№ ПС & Тип трансформатора & \(R_\textup{т.экв}\), Ом &  & \(X_\textup{т.экв}\), Ом & \(\Delta P_\textup{х.экв}\), МВт & \(\Delta Q_\textup{х.экв}\), МВар \\ \hline
		1    & ТРДН-63000/220     & 1,95                     &  & 50,4                     & 0,164                            & 1,008                             \\ \hline
		2    & ТРДН-63000/220     & 1,95                     &  & 50,4                     & 0,164                            & 1,008                             \\ \hline
		4    & ТРДН-40000/110     & 0,7                      &  & 17,4                     & 0,072                            & 0,52                              \\ \hline
		5    & ТРДН-25000/110     & 1,27                     &  & 28,0                     & 0,054                            & 0,350                             \\ \hline
	\end{tabularx}
\end{table}

\begin{table}[H]
	\small
	\caption{Номинальные напряжения и коэффициенты трансформации автотрансформатора}
	\label{tab:коэффы_ат}
	\begin{tabularx}{\linewidth}{|>{\hsize=0.5\hsize\linewidth=\hsize}Z|>{\hsize=1.5\hsize\linewidth=\hsize}Z|Z|Z|Z|Z|Z|}
		\hline
		№ ПС & Тип автотрансформатора & \(U_\textup{ном}^\textup{ВН}\), кВ & \(U_\textup{ном}^\textup{ВН}\), кВ & \(U_\textup{ном}^\textup{ВН}\), кВ & \(k_\textup{т}^\textup{в-с}\) & \(k_\textup{т}^\textup{в-с}\) \\ \hline
		3 & АТДЦТН 125000/220/110 & 230 & 121 & 11 & 0,526 & 0,0478 \\ \hline
 	\end{tabularx}
\end{table}

\begin{table}[H]
	\small
	\caption{Каталожные данные автотрансформатора}
	\label{tab:кат_данные_ат}
	\begin{tabularx}{\linewidth}{|>{\hsize=1.3\hsize\linewidth=\hsize}Z|>{\hsize=0.5\hsize\linewidth=\hsize}Z|>{\hsize=0.5\hsize\linewidth=\hsize}Z|>{\hsize=0.5\hsize\linewidth=\hsize}Z|>{\hsize=0.5\hsize\linewidth=\hsize}Z|>{\hsize=0.5\hsize\linewidth=\hsize}Z|>{\hsize=0.5\hsize\linewidth=\hsize}Z|>{\hsize=0.6\hsize\linewidth=\hsize}Z|>{\hsize=0.6\hsize\linewidth=\hsize}Z|}
		\hline
		Тип АТ & \(R_\textup{в.экв}\), Ом & \(R_\textup{с.экв}\), Ом & \(R_\textup{н.экв}\), Ом & \(X_\textup{в.экв}\), Ом & \(X_\textup{с.экв}\), Ом & \(X_\textup{н.экв}\), Ом & \(\Delta P_\textup{х.экв}\), МВт & \(\Delta Q_\textup{х.экв}\), МВар \\ \hline
		АТДЦТН 125000/220/110 & 0,26 & 0,26 & 1,6 & 24,5 & 0 & 65,5 & 0,130 & 1,25 \\ \hline
	\end{tabularx}
\end{table}

\begin{table}[H]
	\small
	\caption{Параметры регулировочных устройств двухобмоточных трансформаторов, установленных на ПС сети}
	\label{tab:рпн_трансов}
	\begin{tabularx}{\linewidth}{|Z|Z|Z|Z|Z|}
		\hline
		№ ПС & Тип регулировочного устройства & Диапазон регулирования & Напряжение регулируемой стороны & Напряжение нерегулируемой стороны \\ \hline
		1 & РПН & \(\pm 8\times 1,5\%\) & 11,0 & 230 \\ \hline
		2 & РПН & \(\pm 8\times 1,5\%\) & 11,0 & 230 \\ \hline
		4 & РПН & \(\pm 9\times 1,78\%\) & 10,5 & 115 \\ \hline
		5 & РПН & \(\pm 9\times 1,78\%\) & 10,5 & 115 \\ \hline
	\end{tabularx}
\end{table}

\begin{table}[H]
	\small
	\caption{Параметры регулировочных устройств автотрансформатора типа АТДЦТН-125000/220/110}
	\label{tab:рпн_атр}
	\begin{tabularx}{\linewidth}{|Z|Z|Z|Z|}
		\hline
		Тип регулировочного устройства & Диапазон регулирования & Напряжение регулируемой стороны & Напряжение нерегулируемой стороны \\ \hline
		 РПН АТ ВН/СН & \(\pm 6\times 2\%\) & 121,0 & 230 \\ \hline
		 ЛРТ АТ & \(\pm 10\times 1,5\%\) & 11,0 & 230 \\ \hline
	\end{tabularx}
\end{table}

\begin{table}[H]
	\small
	\caption{Параметры схемы замещения ВЛ на одну цепь}
	\label{tab:параметры_вл_на_одну_цепь}
	\begin{tabularx}{\linewidth}{|Z|Z|Z|Z|Z|Z|}
		\hline
		Линия & Марка провода & \textit{L}, км & \(R_\textup{л(1ц)}\), Ом & \(X_\textup{л(1ц)}\), Ом & \(B_\textup{л(1ц)}\cdot 10^{-6}\), См \\ \hline
		К1 & АС 400/51 & 41,8 & 3,05 & 17,6 & 113,0 \\ \hline
		К2 & АС 400/51 & 49,2 & 3,59 & 20,7 & 132,8 \\ \hline
		13 & АС 240/32 & 49,2 & 5,81 & 21,4 & 128,2 \\ \hline
		23 & АС 240/32 & 41,8 & 4,93 & 18,2 & 108,8 \\ \hline
		34 & АС 240/32 & 25,9 & 3,06 & 10,5 & 72,7 \\ \hline
		45 & АС 120/19 & 25,9 & 6,32 & 11,1 & 68,8 \\ \hline
	\end{tabularx}
\end{table}

\begin{table}[H]
	\small
	\caption{Количество и суммарная мощность установленных на подстанциях компенсирующих устройств}
	\label{tab:установленные_бск}
	\begin{tabularx}{\linewidth}{|Z|Z|Z|Z|Z|Z|}
		\hline
		№ ПС & 1 & 2 & 3 & 4 & 5 \\ \hline
		\(N_\textup{БСК}\) & 8 & 8 & 4 & 8 & 4 \\ \hline
		\(Q_{\textup{БСК}\Sigma}\), МВар & 9,6 & 9,6 & 4,8 & 9,6 & 4,8 \\ \hline
	\end{tabularx}
\end{table}

\section{Анализ режима наибольших нагрузок}

При расчете режимов сети двух номинальных напряжений 220 и 110 кВ, связанных через автотрансформаторную подстанцию, необходимо сначала отрегулировать с помощью устройства РПН напряжение на шинах СН подстанции 3. В зависимости от протяженности линий 110 кВ и их нагрузки рекомендуется выбирать желаемое напряжение на шинах СН в режиме НБ в диапазоне \((1,05 \div 1,15)\cdot U_\textup{ном}\), выберем \(1,1\cdot U_\textup{ном} = 121\) кВ. В соответствии с принципом встречного регулирования, в режиме НБ напряжение на шинах 10 кВ понижающих ПС должно быть не менее \(1,05\cdot U_\textup{ном} = 10,5\) кВ \cite{глазунов_шведов}.

Результаты регулировки напряжения на шинах НН подстанций приведены на рисунке \ref{fig:ветви_нб_отрег} в приложении А, а также сведены в таблицу \ref{tab:пв_отрег}.

\begin{table}[H]
	\small
	\caption{Результаты регулировки напряжений в режиме НБ}
	\label{tab:пв_отрег}
	\begin{tabularx}{\linewidth}{|Z|Z|Z|Z|Z|Z|Z|}
		\hline
		№ ПС & 3 (РПН) & 3 (ЛРТ) & 1 & 2 & 4 & 5 \\ \hline
		Пределы регулирования & \(\pm 6\times 2\%\) & \(\pm 10\times 1,5\%\) & \(\pm 8\times 1,5\%\) & \(\pm 8\times 1,5\%\) & \(\pm 9\times 1,78\%\) & \(\pm 9\times 1,78\%\) \\ \hline
		\(k_\textup{т.ном}\) & 0,526 & 0,0478 & 0,0478 & 0,0478 & 0,0913 & 0,0913 \\ \hline
		\(k_\textup{т.рег}\) & 0,526 & 0,0464 & 0,0457 & 0,0457 & 0,0913 & 0,0929 \\ \hline
		\(n_\textup{отв}\) & 0 & 2 & 3 & 3 & 0 & -1 \\ \hline
		\(U_\textup{НН(СН)}\), кВ & 120,77 & 10,52 & 10,59 & 10,55 & 10,65 & 10,55 \\ \hline
	\end{tabularx}
\end{table}

Анализируя данные, приведенные в табл. \ref{tab:пв_отрег}, можно сделать вывод о достаточности регулировочного диапазона устройств РПН и ЛРТ трансформаторов и автотрансформаторов и о приемлемости уровней напряжения на шинах 10 кВ понижающих ПС в режиме НБ.

Значение реактивной мощности \(Q_\textup{г}\), генерируемой сетью, оказалось меньше значения располагаемой мощности \(Q_{\textup{расп}\Sigma}\) (рис. \ref{fig:узлы_нб_отрег}):
\[Q_{\textup{расп}\Sigma} = 87\; \textup{МВар} > Q_\textup{г} = 83,4\; \textup{МВар}\]

Полученные в расчете (см. рис. \ref{fig:ветви_нб_отрег}) значения токов \(I_{max}\), протекающие в линиях, примерно соответствуют приведенным в табл. \ref{tab:эконом_сечение_результаты} значениям токов  \(I_\textup{нб(5)}\). Значит, сечения проводов линий выбраны верно.

Суммарные потери активной мощности в проектируемой сети составляют:
\[\Delta P_{\Sigma \%} = \frac{P_\textup{г} - \Sigma P_\textup{нб}}{\Sigma P_\textup{нб}}\cdot 100\% = \frac{249,97 - 245}{245}\cdot 100\% = 2,03\; \%,\]
где \(P_\textup{г}\) "--- значение активной мощности, выдаваемое в спроектированную сеть с шин 220 кВ ПС К.

Значение суммарных потерь активной мощности в проектируемой сети не превышает диапазона \((4 \div 5) \%\), характерного для сетей 110 "--- 220 кВ \cite{глазунов_шведов}.

\section{Анализ режима наименьших нагрузок}

Желаемое напряжение на шинах СН в режиме НМ выбирается из диапазона \((1,0 \div 1,1)\cdot U_\textup{ном}\), выберем \(1,05\cdot U_\textup{ном} = 115,5\; \textup{кВ}\). В соответствии с принципом встречного регулирования, в режиме НМ напряжение на шинах 10 кВ должно быть не более \(U_\textup{ном}\) \cite{глазунов_шведов}. Результаты регулирования напряжений представлены на рисунке \ref{fig:режим_нм_ветви_отрег} в приложении А, а также сведены в табл. \ref{tab:нм_отрег}.

\begin{table}[H]
	\small
	\caption{Результаты регулировки напряжений в режиме НМ}
	\label{tab:нм_отрег}
	\begin{tabularx}{\linewidth}{|Z|Z|Z|Z|Z|Z|Z|}
		\hline
		№ ПС & 3 (РПН) & 3 (ЛРТ) & 1 & 2 & 4 & 5 \\ \hline
		Пределы регулирования & \(\pm 6\times 2\%\) & \(\pm 10\times 1,5\%\) & \(\pm 8\times 1,5\%\) & \(\pm 8\times 1,5\%\) & \(\pm 9\times 1,78\%\) & \(\pm 9\times 1,78\%\) \\ \hline
		\(k_\textup{т.ном}\) & 0,526 & 0,0478 & 0,0478 & 0,0478 & 0,0913 & 0,0913 \\ \hline
		\(k_\textup{т.рег}\) & 0,5366 & 0,0464 & 0,0464 & 0,0464 & 0,0881 & 0,0881 \\ \hline
		\(n_\textup{отв}\) & -1 & 2 & 2 & 2 & 2 & 2 \\ \hline
		\(U_\textup{НН(СН)}\), кВ & 115,26 & 9,89 & 9,99 & 9,97 & 9,95 & 9,86 \\ \hline
	\end{tabularx}
\end{table}

Анализируя данные, приведенные в табл. \ref{tab:нм_отрег}, можно сделать вывод о достаточности регулировочного диапазона устройств РПН и ЛРТ трансформаторов и автотрансформаторов и о приемлемости уровней напряжения на шинах 10 кВ понижающих ПС в режиме НМ.

Значение потребляемой сетью реактивной мощности \(Q_\textup{г} = 24,5\) МВар (рис. \ref{fig:режим_нм_узлы_отрег}) имеет положительный знак, что означает отсутствие недопустимого перетока мощности из проектируемой сети в систему.

\section{Анализ послеаварийных режимов}

Послеаварийные режимы рассматриваются в период наибольших нагрузок, поэтому желаемый уровень напряжения на шинах 10 кВ в таких режимах должен соответствовать уровню напряжения, требуемому в режиме НБ. Результаты регулировки напряжений приведены в приложении А на рисунках \ref{fig:п/ав_в_сети_220_кв_ветви_отрег} и \ref{fig:п/ав_в_сети_110_кв_ветви_отрег}, а так же сведены в табл. \ref{tab:п/ав_220_отрег} и \ref{tab:п/ав_110_отрег}.

\begin{table}[H]
	\small
	\caption{Результаты регулировки напряжений в послеаварийном режиме в сети 220 кВ при отключении линии К-1}
	\label{tab:п/ав_220_отрег}
	\begin{tabularx}{\linewidth}{|Z|Z|Z|Z|Z|Z|Z|}
		\hline
		№ ПС & 3 (РПН) & 3 (ЛРТ) & 1 & 2 & 4 & 5 \\ \hline
		Пределы регулирования & \(\pm 6\times 2\%\) & \(\pm 10\times 1,5\%\) & \(\pm 8\times 1,5\%\) & \(\pm 8\times 1,5\%\) & \(\pm 9\times 1,78\%\) & \(\pm 9\times 1,78\%\) \\ \hline
		\(k_\textup{т.ном}\) & 0,526 & 0,0478 & 0,0478 & 0,0478 & 0,0913 & 0,0913 \\ \hline
		\(k_\textup{т.рег}\) & 0,5682 & 0,0507 & 0,0514 & 0,0478 & 0,0913 & 0,0929 \\ \hline
		\(n_\textup{отв}\) & -4 & -4 & -5 & 0 & 0 & -1 \\ \hline
		\(U_\textup{НН(СН)}\), кВ & 121,04 & 10,64 & 10,65 	& 10,63 & 10,67 & 10,58 \\ \hline
	\end{tabularx}
\end{table}

\begin{table}[H]
	\small
	\caption{Результаты регулировки напряжений в послеаварийном режиме в сети 110 кВ при отключении одной цепи линии 3-4}
	\label{tab:п/ав_110_отрег}
	\begin{tabularx}{\linewidth}{|Z|Z|Z|Z|Z|Z|Z|}
		\hline
		№ ПС & 3 (РПН) & 3 (ЛРТ) & 1 & 2 & 4 & 5 \\ \hline
		Пределы регулирования & \(\pm 6\times 2\%\) & \(\pm 10\times 1,5\%\) & \(\pm 8\times 1,5\%\) & \(\pm 8\times 1,5\%\) & \(\pm 9\times 1,78\%\) & \(\pm 9\times 1,78\%\) \\ \hline
		\(k_\textup{т.ном}\) & 0,526 & 0,0478 & 0,0478 & 0,0478 & 0,0913 & 0,0913 \\ \hline
		\(k_\textup{т.рег}\) & 0,5261 & 0,0471 & 0,0457 & 0,0457 & 0,0929 & 0,0962 \\ \hline
		\(n_\textup{отв}\) & 0 & 1 & 3 & 3 & -1 & -3 \\ \hline
		\(U_\textup{НН(СН)}\), кВ & 120,32 & 10,64 & 10,58 & 10,54 & 10,56 & 10,63 \\ \hline
	\end{tabularx}
\end{table}

Анализируя данные, приведенные в таблицах \ref{tab:п/ав_220_отрег} и \ref{tab:п/ав_110_отрег}, можно сделать вывод о достаточности регулировочного диапазона устройств РПН и ЛРТ трансформаторов и автотрансформаторов и о приемлемости уровней напряжения на шинах 10 кВ понижающих ПС в послеаварийных режимах.

В послеаварийных режимах на участках сети 220 и 110 кВ реактивная мощность \(Q_\textup{г}\), генерируемая сетью, оказалась больше значения располагаемой мощности \(Q_{\textup{расп}\Sigma}\) (рис. \ref{fig:п/ав_в_сети_220_кв_узлы_отрег} и \ref{fig:п/ав_в_сети_110_кв_узлы_отрег}):
\[Q_{\textup{расп}\Sigma} = 87\; \textup{МВар} < Q_\textup{г.п/ав.220} = 126,6\; \textup{МВар}\]
\[Q_{\textup{расп}\Sigma} = 87\; \textup{МВар} < Q_\textup{г.п/ав.110} = 88,2\; \textup{МВар}\]

Так как послеаварийные режимы работы сети не является долгосрочным и не скажутся на экономичности работы сети, допускается отклонение от заданного потребления реактивной мощности \cite{глазунов_шведов}.

Выполним проверку условия длительно допустимого нагрева проводов ВЛ по результатам расчета двух послеаварийных режимов:
\[I_{max} \leq I_\textup{дл.доп}^{'}\cdot k_\textup{Т},\]
где \(I_{max}\) "--- максимальный ток по цепям линий, А; \(k_\textup{Т}\) = 1,35 (из пункта 4.2) "--- поправочный коэффициент на температуру воздуха.

Результаты проверки сечений проводов по длительно допустимому нагреву приведены в таблицах \ref{tab:проверка_сечений_в_п.ав_220} и \ref{tab:проверка_сечений_в_п.ав_110}, из которых видно, что провода ВЛ проектируемой сети удовлетворяют условию длительно допустимого нагрева в послеаварийных режимах.

\begin{table}[H]
	\small
	\caption{Проверка сечений проводов по длительно допустимому нагреву в послеаварийном режиме в сети 220 кВ}
	\label{tab:проверка_сечений_в_п.ав_220}
	\begin{tabularx}{\linewidth}{|Z|Z|Z|Z|Z|Z|}
		\hline
		Линия & Марка провода & \(I_{max}\), А & \(I_\textup{дл.доп}\), А & \(I_\textup{дл.доп}^{'}\), А & статус проверки \\ \hline
		К-2 & АС 400/51 & 691 & 825 & 1114 & + \\ \hline
	\end{tabularx}
\end{table}

\begin{table}[H]
	\small
	\caption{Проверка сечений проводов по длительно допустимому нагреву в послеаварийном режиме в сети 110 кВ}
	\label{tab:проверка_сечений_в_п.ав_110}
	\begin{tabularx}{\linewidth}{|Z|Z|Z|Z|Z|Z|}
		\hline
		Линия & Марка провода & \(I_{max}\), А & \(I_\textup{дл.доп}\), А & \(I_\textup{дл.доп}^{'}\), А & статус проверки \\ \hline
		3-4 & АС 240/32 & 401 & 605 & 816,8 & + \\ \hline
	\end{tabularx}
\end{table}

Сечения проводов ВЛ удовлетворяют условию длительно допустимого нагрева в ПА/В режимах в сетях 220 и 110 кВ.

%%% Local Variables:
%%% mode: latex
%%% TeX-master: "rpz"
%%% End: