\chapter{Выбор числа и мощности трансформаторов понижающих подстанций}
\label{cha:выбор_трансов}

Выбор числа и мощности трансформаторов (ТР) на подстанции зависит от требований к надежности электроснабжения питающихся от подстанции потребителей. В практике проектирования на подстанциях рекомендуется, как правило, установка двух трансформаторов \cite{файбисович}.

При выборе мощности трансформаторов, как правило, определяющим условием является не экономический критерий, согласно которому экономически выгодно трансформаторы перегружать, а их нагрузочная способность.

При отсутствии подробной информации о графиках нагрузки проектируемых подстанций, в соответствии с существующей практикой проектирования, допускается упрощенный выбор мощности трансформаторов из условия допустимой перегрузки трансформаторов в послеаварийных режимах:
\begin{eqndesc}[H]
	\begin{equation}
		S_\textup{т.ном} \geq \frac{S_\textup{нб}}{k_\textup{ав}(n_\textup{т} - 1)},
		\label{eqn:усл_допустимой_нагрузки_тр}
	\end{equation}
где \(S_\textup{нб}\) "--- наибольшая нагрузка на шинах низшего напряжения подстанции с учетом установленных компенсирующих устройств; \(n_\textup{т}\) "--- число трансформаторов, устанавливаемых на подстанции; \(k_\textup{ав}\) "--- коэффициент допустимой перегрузки трансформатора в послеаварийном режиме работы.
\end{eqndesc}

В случае трехобомоточных трансформаторов или автотрансформаторов условие \eqref{eqn:усл_допустимой_нагрузки_тр} преобразуется к виду:
\begin{eqndesc}[H]
	\begin{equation}
		S_\textup{т.ном} \geq \frac{|\dot{S_\textup{сн.нб}} + \dot{S_\textup{нн.нб}}|}{k_\textup{ав}(n_\textup{т} - 1)},
		\label{eqn:усл_доп_нагр_атр}
	\end{equation}
где \(S_\textup{сн.нб}\) и \(S_\textup{нн.нб}\) "--- соответственно нагрузка на шинах среднего и низшего напряжения подстанции с учетом установленных компенсирующих устройств.
\end{eqndesc}

Для автотрансформаторов номинальная мощность обмотки низшего напряжения отличается от номинальной мощности автотрансформатора, поэтому первое условие выбора мощности автотрансформаторов \eqref{eqn:усл_доп_нагр_атр} должно быть дополнено вторым условием:
\begin{eqndesc}[H]
	\begin{equation}
		S_\textup{т.ном} \geq \frac{S_\textup{нн.нб}}{k_\textup{ав} \alpha (n_\textup{т} - 1)},
		\label{eqn:2-е_усл_доп_нагр_атр}
	\end{equation}
где \(\alpha\) "--- отношение номинальной мощности обмотки низшего напряжения автотрансформатора к номинальной мощности автотрансформатора.
\end{eqndesc}

В соответствии с ГОСТом 14209-97 в послеаварийном режиме допускается перегрузка двухобмоточных трансформаторов на 40 \% номинальной мощности, то есть \(k_\textup{ав} = 1,4\). В соответствии с ТУ № 3411-001-498-90-270-2005 в послеаварийном режиме допускается перегрузка автотрансформаторов на 20 \% номинальной мощности, то есть \(k_\textup{ав} = 1,2\).

\section{Подбор трансформаторов для варианта схемы 1}

В качестве примера выберем двухобомоточный трансформатор для пункта потребления 1.

Рассчитаем условие допустимой перегрузки трансформаторов в послеаварийном режиме:
\[S_\textup{т.ном} \geq \frac{S_\textup{нб1}^{''}}{k_\textup{ав}(n_\textup{т} - 1)} = \frac{72,9}{1,4(2 - 1)} = 52,1\; \textup{МВА}\]

Для ПС1 по справочнику Файбисовича \cite{файбисович} выбираем марку трансформатора ТРДН 63000/220. Для остальных подстанций выбор проводится аналогично, исключение составляет автотрансформаторная ПС3, для которого выбор осуществляется по-другим условиям. Сведем результаты выбора в табл. \ref{tab:выбор_двухобмоточных_тр}

\begin{table}[H]
	\small
	\caption{Результаты выбора двухобмоточных трансформаторов для варианта схемы сети 1}
	\label{tab:выбор_двухобмоточных_тр}
	\begin{tabularx}{\linewidth}{|Z|Z|Z|Z|}
		\hline
		№ Подстанции & \(S_\textup{нб}^{''}\), МВА & \(S_\textup{т}\), МВА & Марка трансформатора \\ \hline
		1            & 72,9                        & 52,1                  & ТРДН 63000/220       \\ \hline
		2            & 72,9                        & 52,1                  & ТРДН 63000/220       \\ \hline
		4            & 41,5                        & 29,6                  & ТРДН 40000/110       \\ \hline
		5            & 36,7                        & 26,2                  & ТРДН 25000/110       \\ \hline
	\end{tabularx}
\end{table}

Для выбора автотрансформатора определим допустимую перегрузку по условию \eqref{eqn:усл_доп_нагр_атр}. Для этого сначала определим комплексные нагрузки на шинах среднего и низшего напряжения подстанции. Активные и реактивные нагрузки возьмем из табл. \ref{tab:нагрузки_с_батареями_кольцо}.
\[
\begin{split}
\dot{S_\textup{сн.нб}} &= P_\textup{нб4} + jQ_\textup{прив4} + P_\textup{нб5} + jQ_\textup{прив5} = \\ &= 40 + j14,2 + 35 + j14,1 = 75 + j28,3\; \textup{МВА}
\end{split}
\]

\[\dot{S_\textup{нн.нб}} = P_\textup{нб3} + jQ_\textup{нб3}^{''} = 30 + j8,9\; \textup{МВА}\]

Допустимая перегрузка по первому условию:
\[
\begin{split}
S_\textup{АТ} &\geq \frac{|\dot{S_\textup{сн.нб}} + \dot{S_\textup{нн.нб}}|}{k_\textup{ав}\cdot (n_\textup{т} - 1)} = \frac{|75 + j28,3 + 30 + j8,9|}{1,2(2 - 1)} = \\ &= \frac{\sqrt{105^2 + 37,2^2}}{1,2 (2 - 1)} = 92,8\; \textup{МВА}
\end{split}
\]

По второму условию \eqref{eqn:2-е_усл_доп_нагр_атр} получим:
\[S_\textup{АТ} \geq \frac{S_\textup{нн.нб}}{k_\textup{ав}\cdot \alpha (n_\textup{т} - 1)} = \frac{31,3}{1,2\cdot 0,5 (2 - 1)} = 52,2\; \textup{МВА}\]

Из этих двух условий берем максимальное, то есть \(S_\textup{АТ} \geq 92,8\) и выбираем по справочнику Файбисовича \cite{файбисович} автотрансформатор марки АТДЦТН 125000/220/110

ЛРТ выбирается по тому же условию \eqref{eqn:усл_допустимой_нагрузки_тр}, что и двухобмоточный трансформатор, тогда:
\[S_\textup{ЛРТ} \geq \frac{S_\textup{нб.АТ}}{k_\textup{ав}(n_\textup{лрт} - 1)} = \frac{31,3}{1,4 (2 - 1)} = 22,4\; \textup{МВА}\]

Согласно справочнику \cite{файбисович} выбираем линейный регулировочный трансформатор марки ЛТДН 40000/10.

\section{Подбор трансформаторов для варианта схемы 2}

Выбор происходит аналогично примеру выше, поэтому приведем таблицу \ref{tab:выбор_двухобмоточных_тр_магистраль} с результатами выбора трансформаторов

\begin{table}[H]
	\small
	\caption{Результаты выбора двухобмоточных трансформаторов для варианта схемы сети 2}
	\label{tab:выбор_двухобмоточных_тр_магистраль}
	\begin{tabularx}{\linewidth}{|Z|Z|Z|Z|}
		\hline
		№ Подстанции & \(S_\textup{нб}^{''}\), МВА & \(S_\textup{т}\), МВА & Марка трансформатора \\ \hline
		1            & 72,2                        & 51,6                  & ТРДН 63000/220       \\ \hline
		2            & 72,2                        & 51,6                  & ТРДН 63000/110       \\ \hline
		4            & 40,9                        & 29,2                  & ТРДН 40000/110       \\ \hline
		5            & 36,1                        & 25,8                  & ТРДН 25000/110       \\ \hline
	\end{tabularx}
\end{table}

Автотрансформатор на ПС3 выбираем по условию \eqref{eqn:усл_доп_нагр_атр} \(S_\textup{АТ} \geq 153,1\) МВА. По справочнику \cite{файбисович} выбираем марку автотрансформатора АТДЦТН 200000/220/110.

В качестве линейного трансформатора по условию \eqref{eqn:усл_допустимой_нагрузки_тр} \(S_\textup{ЛРТ} \geq 22,4\) МВА, следовательно выбираем марку линейного регулировочного трансформатора ЛТДН 40000/10.

%%% Local Variables:
%%% mode: latex
%%% TeX-master: "rpz"
%%% End: