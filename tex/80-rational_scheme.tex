\chapter{Выбор рационального варианта схемы сети}
\label{cha:рациональная_схема}

На данном этапе проектирования из двух сформированных вариантов схемы сети необходимо выбрать наиболее рациональный путем их технико-экономического сопоставления.

Проводится технико-экономическое сопоставление вариантов схем сети в их различающихся частях. Одним из основных показателей экономической эффективности проекта является чистый дисконтированный доход (ЧДД) за расчетный период, приведенный к году начала реализации проекта (первому году):
\begin{eqndesc}[h]
	\begin{equation*}
		\textup{ЧДД} = \sum_{t=1}^{T_\textup{р}} [(\textup{Д}_t - \textup{З}_t)(1 + E)^{1-t}],
		\label{eqn:чдд}
	\end{equation*}
где \(\textup{Д}_t\) "--- величина дохода в год \textit{t}; \(\textup{З}_t\) "--- величина затрат в год \textit{t}; \textit{E} "--- норматив дисконтирования, \(E = 0,1\); \(T_\textup{р}\) "--- расчетный период, который для электрических сетей напряжением до 220 кВ включительно рекомендуется принимать равным 10 лет.
\end{eqndesc}

Проект считается экономически эффективным, если его ЧДД больше нуля, при выборе между несколькими вариантами предпочтение отдается варианту с более высоким ЧДД. Так как все варианты имеют одинаковый производственный эффект, т.е. \(\textup{Д}_{t1} = \textup{Д}_{t2}\), сравнение сводится к сравнению дисконтированных затрат за расчетный период:
\begin{eqndesc}[h]
	\begin{equation}
		\textup{З}_\textup{д} = D_\textup{р}\cdot \textup{К}_\Sigma + D_\textup{д}\cdot \textup{И}_{\textup{пот}\Sigma},
		\label{eqn:упрощ.формула_затрат}
	\end{equation}
где \(D_\textup{р}\) "--- расчетный множитель; \(D_\textup{д}\) "--- дисконтирующий множитель; \(\textup{К}_\Sigma\) "--- суммарные капиталовложения в сеть, тыс. руб.; \(\textup{И}_{\textup{пот}\Sigma}\) "--- суммарные ежегодные издержки на возмещение потерь электроэнергии, тыс.руб/год.
\end{eqndesc}

При технико-экономическом сопоставлении вариантов схемы сети должны учитываться только те составляющие затрат, по которым сравниваемые варианты отличаются друг от друга.

Проанализируем отличающиеся части двух рассматриваемых вариантов схемы сети:

\begin{itemize}
	\item одноцепные линии 220 кВ: К-1, К-2, 1-3, 2-3 (вариант схемы сети 1);
	\item двухцепные линии 220 кВ: К-1, 1-3 (вариант схемы сети 2);
	\item двухцепная линия 110 кВ: 3-2 (вариант схемы сети 2);
	\item РУВН на подстанциях 1 и 2;
	\item РУСН на подстанции 3;
	\item трансформаторное оборудование на подстанциях 2 и 3;
	\item дополнительное оборудование в виде компенсирующих устройств на каждой подстанции, кроме ПС3.
\end{itemize}

Капиталовложения на сооружение линии электропередачи:
\begin{equation}
	\textup{К}_\textup{ЛЭП} = \textup{К}_{0\textup{ЛЭП.баз}}\cdot k_\textup{зон}\cdot k_\textup{усл}\cdot k_\textup{деф}\cdot L,
	\label{eqn:капиталовложения_лэп}
\end{equation}
где \(\textup{К}_{0\textup{ЛЭП.баз}}\) "--- укрупненный показатель стоимости сооружения 1 км воздушной линии, тыс. руб./км; \(k_\textup{зон} = 1,0\) "--- учитывающий удорожание строительства ВЛ в Центре; \(k_\textup{усл} = 1\) "--- коэффициент усложняющих условий строительства, в отсутствии данных принимается равным единице; \(k_\textup{деф} = 4,25\) "--- коэффииент дефляции для приведения цен базового года к ценам 2010 года.

Для линии К-1 варианта схемы сети 1 длиной 41,8 км, сооружаемой на железобетонных опорах с проводами марки АС 400/51 в I-м гололедном районе:
\[\textup{К}_\textup{ЛЭП.К-1} = 1188\cdot 1,0\cdot 1,0\cdot 4,25\cdot 41,8 = 211048,2\; \textup{тыс. руб}\]

Капиталовложения в остальные линии для вариантов схем сети 1 и 2 рассчитываются аналогично, результаты расчетов представлены в таблицах \ref{tab:капиталовложения_лэп_1} и \ref{tab:капиталовложения_лэп_2}, соответственно.

\begin{table}[H]
	\small
	\caption{Капиталовложения на сооружение ЛЭП для варианта схемы сети 1}
	\label{tab:капиталовложения_лэп_1}
	\begin{tabularx}{\linewidth}{|Z|Z|Z|Z|Z|Z|}
		\hline
		ЛЭП & К-1 & К-2 & 1-3 & 2-3 & Итого \\ \hline
		\(U_\textup{ном}\), кВ & 220 & 220 & 220 & 220 & - \\ \hline
		\(F_a,\; \textup{мм}^2\) & 400 & 400 & 240 & 240 & - \\ \hline
		\(n_\textup{ц}\) & 1 & 1 & 1 & 1 & - \\ \hline
		\(\textup{К}_{0\textup{ЛЭП.баз}},\; \frac{\textup{тыс.руб}}{\textup{км}}\) & 1188 & 1188 & 990 & 990 & - \\ \hline
		\(L_{i-j}\), км & 41,8 & 49,2 & 49,2 & 41,8 & - \\ \hline
		\(\textup{К}_\textup{ЛЭП}\), тыс. руб & 211048,2 & 248419,8 & 207009,0 & 175873,5 & 842341,5 \\ \hline
	\end{tabularx}
\end{table}

\begin{table}[H]
	\small
	\caption{Капиталовложения на сооружение ЛЭП для варианта схемы сети 2}
	\label{tab:капиталовложения_лэп_2}
	\begin{tabularx}{\linewidth}{|Z|Z|Z|Z|Z|}
		\hline
		ЛЭП & К-1 & 1-3 & 2-3 & Итого \\ \hline
		\(U_\textup{ном}\), кВ & 220 & 220 & 110 & - \\ \hline
		\(F,\; \textup{мм}^2\) & 400 & 300 & 240 & - \\ \hline
		\(n_\textup{ц}\) & 2 & 2 & 2  & - \\ \hline
		\(\textup{К}_{0\textup{ЛЭП.баз}},\; \frac{\textup{тыс.руб}}{\textup{км}}\) & 2219 & 2020 & 1403  & - \\ \hline
		\(L_{i-j}\), км & 41,8 & 49,2 & 41,8  & - \\ \hline
		\(\textup{К}_\textup{ЛЭП}\), тыс. руб & 394205,35 & 422382,0 & 249242,95 & 1065830,3 \\ \hline
	\end{tabularx}
\end{table}

Капиталовложения на сооружение подстанций:
\begin{eqndesc}[h]
	\begin{equation}
		\textup{К}_\textup{ПС} = \textup{К}_{\textup{ТР}\Sigma} + \textup{К}_{\textup{РУ}\Sigma} + \textup{К}_{\textup{доп}\Sigma} + \textup{К}_{\textup{пост}},
		\label{eqn:капит_на_сооруж_пс}
	\end{equation}
где \(\textup{К}_{\textup{ТР}\Sigma}\) "--- стоимость устанавливаемых трансформаторов (автотрансформаторов), тыс. руб; \(\textup{К}_{\textup{РУ}\Sigma}\) "--- суммарная стоимость РУ всех классов напряжения, тыс. руб; \(\textup{К}_{\textup{доп}\Sigma}\) "--- суммарная стоимость дополнительного оборудования, тыс. руб; \(\textup{К}_{\textup{пост}}\) "--- постоянная часть затрат на сооружение ПС, тыс. руб.
\end{eqndesc}

В курсовом проекте принимаем, что при напряжениях 220 и 110 кВ используются воздушные выключатели, а при 10 кВ - вакуумные \cite{глазунов_шведов}

В качестве примера приведем расчет капиталовложений для ПС2.

\textit{Трансформаторы}

В первом и втором варианте схемы отличается класс уровня напряжения трансформатора (ТРДН-63000/220 и ТРДН-63000/110 соответственно), установленные на ПС2. Капиталовложения для каждой пары ячеек трансформаторов рассчитываются по формуле:
	\begin{equation*}
		\textup{К}_\textup{ТР} = \textup{К}_\textup{тр.баз}\cdot k_\textup{зон}\cdot k_\textup{деф}\cdot n_\textup{т},
	\end{equation*}
где \(\textup{К}_\textup{тр.баз}\) "--- базовый укрупненный показатель стоимости ячейки автотрансформаторов, тыс. руб; \(n_\textup{т}\) "--- число трансформаторов, установленных на ПС.

\[\textup{К}_\textup{ТР2(1)} = \textup{К}_\textup{тр.баз.220}\cdot k_\textup{зон}\cdot k_\textup{деф}\cdot n_\textup{т} = 12625\cdot 1\cdot 4,25\cdot 2 = 107312,5\; \textup{тыс. руб}\]
\[\textup{К}_\textup{ТР2(2)} = \textup{К}_\textup{тр.баз.110}\cdot k_\textup{зон}\cdot k_\textup{деф}\cdot n_\textup{т} = 9000\cdot 1\cdot 4,25\cdot 2 = 76500,0\; \textup{тыс. руб}\]

\textit{Схемы РУ}

На ПС2 в двух вариантах схем сети отличаются схемы РУВН, для первого варианта "--- "четырехугольник", а для второго варианта "--- "мостик".

Капиталовложения в РУ одного класса напряжения, состоящего из \(n_\textup{яч}\) ячеек рассчитываются по формуле:
\[\textup{К}_\textup{РУ} = \textup{К}_\textup{выкл.баз}\cdot k_\textup{зон}\cdot k_\textup{деф}\cdot n_\textup{яч},\]
где \(\textup{К}_\textup{выкл.баз}\) "--- базовый укрупненный показатель стоимости ячейки выключателя в базисных ценах на 01.01.2000 для европейской части РФ.
\[\textup{К}_\textup{РУВН2(1)} = \textup{К}_\textup{выкл.баз.220}\cdot k_\textup{зон}\cdot k_\textup{деф}\cdot n_\textup{яч2}^\textup{220} = 8800\cdot 4,25\cdot 4 = 149600\; \textup{тыс. руб}\]
\[\textup{К}_\textup{РУВН2(2)} = \textup{К}_\textup{выкл.баз.110}\cdot k_\textup{зон}\cdot k_\textup{деф}\cdot n_\textup{яч2}^\textup{110} = 4150\cdot 4,25\cdot 3 = 52912,5\; \textup{тыс. руб}\]

Капиталовложения постоянной части затрат:
\begin{equation*}
	\textup{К}_\textup{пост} = \textup{К}_\textup{пост.баз}\cdot k_\textup{зон}\cdot k_\textup{деф},
\end{equation*}
где \(\textup{К}_\textup{пост.баз}\) "--- базовый укрупненный показатель постоянной части затрат на сооружение ПС в базисных ценах на 01.01.2000 г. для европейской части РФ.

Постоянная часть затрат ПС2:
\[\textup{К}_\textup{пост2(1)} = \textup{К}_\textup{пост.баз.220}\cdot k_\textup{зон}\cdot k_\textup{деф} = 19500\cdot 1\cdot 4,25 = 82875\; \textup{тыс. руб}\]
\[\textup{К}_\textup{пост2(2)} = \textup{К}_\textup{пост.баз.110}\cdot k_\textup{зон}\cdot k_\textup{деф} = 9000\cdot 1\cdot 4,25 = 38250\; \textup{тыс. руб}\]

\textit{Дополнительное оборудование}

В первом варианте схемы сети предполагается установка 8 БСК на ПС2, а во втором варианте схемы сети 10 БСК. Дополнительные капиталовложения в дополнительное оборудование для варианта схемы сети 2:
\[\textup{К}_\textup{доп2(2)} = \textup{К}_\textup{БСК.баз}\cdot k_\textup{зон}\cdot k_\textup{деф}\cdot (n_\textup{БСК2(2)} - n_\textup{БСК2(1)})\]
где \(\textup{К}_\textup{БСК.баз}\) "--- укрупненная стоимость шунтовой конденсаторной батареи 10 кВ единичной мощностью 1,2 МВар; \(n_\textup{БСК2(1)}\) "--- общее число КУ, установленных в варианте схемы сети 1; \(n_\textup{БСК2(2)}\) "--- общее число КУ, установленных в варианте схемы сети 2.
\[\textup{К}_\textup{доп2(2)} = \textup{К}_\textup{БСК.баз}\cdot k_\textup{зон}\cdot k_\textup{деф}\cdot (n_\textup{БСК2(2)} - n_\textup{БСК2(1)}) =\] \[= 375\cdot 425\cdot (10-8) = 3187,5\; \textup{тыс. руб}\]

Для остальных ПС с различной стоимостью составляющих в двух вариантах схемы сети расчет проводится аналогично. Результаты расчетов различной стоимости ПС вместе с капиталовложениями в ЛЭП сведем в табл. \ref{tab:сравнение_схем}.

%\begin{table}
%	\small
{\small
	\begin{xltabular}{\linewidth}{|Z|Z|Z|Z|Z|}		
		\caption{Сравнение вариантов схем сети по стоимости ЛЭП и ПС} 
		\label{tab:сравнение_схем} \\ \hline
		\multicolumn{5}{|c|}{ЛЭП} \\ 
		\endfirsthead
		\caption{\textit{(Продолжение)} Сравнение вариантов схем сети по стоимости ЛЭП и ПС}\\
		\hline
		\endhead
		\multicolumn{5}{r}{\textit{Продолжение на следующей странице}} \\
		\endfoot
		\endlastfoot
		\hline
		Вариант & Линия & \(\textup{К}_{0\textup{ЛЭП.баз}}\), тыс. руб/км & \(\textup{К}_\textup{ЛЭП}\), тыс. руб & \(\textup{К}_{\textup{ЛЭП}\Sigma}\), тыс. руб \\ \hline
		\multirow{4}{*}{№ 1} & К-1 & 1188 & 211048,2 & \multirow{4}{*}{842348,5} \\ \cline{2-4}
		& К-2 & 1188 & 248419,8 &                           \\ \cline{2-4}
		& 1-3 & 990  & 207009,0 &                           \\ \cline{2-4}
		& 2-3 & 990  & 175873,5 &                           \\ \hline
		\multirow{3}{*}{№ 2} & К-1 & 2219 & 394205,4 & \multirow{3}{*}{1065830,3}\\ \cline{2-4}
		& 1-3 & 2020 & 422382,0 &                           \\ \cline{2-4}
		& 2-3 & 1403 & 249243,0 &                           \\ \hline
		\multicolumn{5}{|c|}{ПС2} \\ \hline
		Вариант & \multicolumn{2}{c|}{Тип ТР} & \(\textup{К}_\textup{тр.баз}\), тыс. руб & \(\textup{К}_\textup{ТР.2}\), тыс. руб \\ \hline
		№ 1 & \multicolumn{2}{c|}{ТРДН-63000/220} & 12625 & 107312,5 \\ \hline
		№ 2 & \multicolumn{2}{c|}{ТРДН-63000/110} & 9000  & 76500,0  \\ \hline
		\multicolumn{5}{|c|}{РУВН ПС2} \\ \hline
		Вариант & \multicolumn{2}{c|}{\(n_\textup{яч}\)} & \(\textup{К}_\textup{выкл. баз}\), тыс. руб & \(\textup{К}_\textup{РУВН2}\), тыс. руб \\ \hline
		№ 1 & \multicolumn{2}{c|}{4} & 8800 & 149600 \\ \hline
		№ 2 & \multicolumn{2}{c|}{3} & 4150 & 52912,5 \\ \hline
		\multicolumn{5}{|c|}{Дополнительное оборудование ПС2} \\ \hline
		Вариант & \multicolumn{2}{c|}{\(N_\textup{БСК}\)} & \(\textup{К}_\textup{БСК.баз}\), тыс. руб & \(\textup{К}_\textup{доп2(2)}\), тыс. руб \\ \hline
		№ 1 & \multicolumn{2}{c|}{8} & 375 & - \\ \hline
		№ 2 & \multicolumn{2}{c|}{10} & 375 & 3187,5 \\ \hline
		\multicolumn{5}{c|}{Постоянная часть затрат на сооружение ПС2} \\ \hline
		Вариант & \multicolumn{2}{c|}{Схема РУВН (ВН/НН)} & \(\textup{К}_\textup{пост.баз}\), тыс. руб & \(\textup{К}_\textup{пост2}\), тыс. руб. \\ \hline
		№ 1 & \multicolumn{2}{c|}{Четырехугольник (220/10)} & 19500 & 82875 \\ \hline
		№ 2 & \multicolumn{2}{c|}{Мостик (110/10)} & 9000 & 38250 \\ \hline
		\multicolumn{5}{|c|}{ПС3} \\ \hline
		Вариант & \multicolumn{2}{c|}{Тип ТР} & \(\textup{К}_\textup{тр.баз}\), тыс. руб & \(\textup{К}_\textup{ТР.3}\), тыс. руб \\ \hline
		№ 1 & \multicolumn{2}{c|}{АТДЦТН-125000/220/110} & 15525 & 131962,5 \\ \hline
		№ 2 & \multicolumn{2}{c|}{АТДЦТН-200000/220/110} & 21050 & 178925,0 \\ \hline
		\multicolumn{5}{|c|}{РУCН ПС3} \\ \hline
		Вариант & \multicolumn{2}{c|}{\(n_\textup{яч}^{110}\)} & \(\textup{К}_\textup{выкл. баз}\), тыс. руб & \(\textup{К}_\textup{РУСН3(2)}\), тыс. руб \\ \hline
		№ 1 & \multicolumn{2}{c|}{7} & 4150 & - \\ \hline
		№ 2 & \multicolumn{2}{c|}{9} & 4150 & 35275 \\ \hline
		\multicolumn{5}{|c|}{ПС1} \\ \hline
		\multicolumn{5}{|c|}{Постоянная часть затрат на сооружение ПС1} \\ \hline
		Вариант & \multicolumn{2}{c|}{Схема РУВН (ВН/НН)} & \(\textup{К}_\textup{пост.баз}\), тыс. руб & \(\textup{К}_\textup{пост1}\), тыс. руб. \\ \hline
		№ 1 & \multicolumn{2}{c|}{Четырехугольник (220/10)} & 19500 & - \\ \hline
		№ 2 & \multicolumn{2}{c|}{Сборные шины (220/10)} & 19500 & - \\ \hline
		\multicolumn{5}{|c|}{РУВН ПС1} \\ \hline
		№ 1 & \multicolumn{2}{c|}{4} & 8800 & - \\ \hline
		№ 2 & \multicolumn{2}{c|}{9} & 8800 & 187000 \\ \hline
		\multicolumn{5}{|c|}{Дополнительное оборудование ПС1} \\ \hline
		Вариант & \multicolumn{2}{c|}{\(N_\textup{БСК}\)} & \(\textup{К}_\textup{БСК.баз}\), тыс. руб & \(\textup{К}_\textup{доп1(2)}\), тыс. руб \\ \hline
		№ 1 & \multicolumn{2}{c|}{8} & 375 & - \\ \hline
		№ 2 & \multicolumn{2}{c|}{10} & 375 & 3187,5 \\ \hline
		\multicolumn{5}{|c|}{ПС4} \\ \hline
		\multicolumn{5}{|c|}{Дополнительное оборудование ПС4} \\ \hline
		Вариант & \multicolumn{2}{c|}{\(N_\textup{БСК}\)} & \(\textup{К}_\textup{БСК.баз}\), тыс. руб & \(\textup{К}_\textup{доп4(2)}\), тыс. руб \\ \hline
		№ 1 & \multicolumn{2}{c|}{8} & 375 & - \\ \hline
		№ 2 & \multicolumn{2}{c|}{10} & 375 & 3187,5 \\ \hline
		\multicolumn{5}{|c|}{ПС5} \\ \hline
		\multicolumn{5}{|c|}{Дополнительное оборудование ПС5} \\ \hline
		Вариант & \multicolumn{2}{c|}{\(N_\textup{БСК}\)} & \(\textup{К}_\textup{БСК.баз}\), тыс. руб & \(\textup{К}_\textup{доп5(1)}\), тыс. руб \\ \hline
		№ 1 & \multicolumn{2}{c|}{4} & 375 & - \\ \hline
		№ 2 & \multicolumn{2}{c|}{6} & 375 & 3187,5 \\ \hline
		\multicolumn{5}{|c|}{Итого} \\ \hline
		Вариант & \multicolumn{2}{c|}{\(\textup{К}_{\textup{ЛЭП}\Sigma}\), тыс. руб.} & \(\textup{К}_{\textup{ПС.220}\Sigma}\), тыс. руб & \(\textup{К}_{\textup{ПС.110}\Sigma}\), тыс. руб \\ \hline
		№ 1 & \multicolumn{2}{c|}{842348,5} & 471750 & 0 \\ \hline
		№ 2 & \multicolumn{2}{c|}{1065830,3} & 365925 & 215687,5 \\ \hline
	\end{xltabular}
}
%\end{table}

\textit{Издержки на возмещение потерь электроэнергии}

Приведем пример расчета нагрузочных потерь активной мощности для линии К-1 варианта схемы сети 1:
\[\Delta P_{K-1(1)} = \frac{S_{K-1}^2\cdot R_{K-1}}{U_\textup{ном}^2} = \frac{134,9^2\cdot 3,05}{220^2} = 1,15\; \textup{МВт}\]

Расчет для других линий вариантов схемы сети 1 и 2 проведен аналогично. Результаты расчета сведены в таблицы \ref{tab:нагр_потери_1} и \ref{tab:нагр_потери_2}.

\begin{table}[h]
	\small
	\caption{Нагрузочные потери в линиях для варианта схемы сети 1}
	\label{tab:нагр_потери_1}
	\begin{tabularx}{\linewidth}{|Z|Z|Z|Z|Z|Z|Z|Z|}
		\hline
		Линия                           & К-1   & К-2   & 1-3  & 2-3  & 3-4  & 4-5  & Итого \\ \hline
		\(U_\textup{ном}\), кВ          & 220   & 220   & 220  & 220  & 110  & 110  & -     \\ \hline
		\textit{S}, МВА                 & 134,9 & 129,0 & 60,3 & 54,4 & 80,2 & 37,7 & -     \\ \hline
		\(R_\textup{л}\), Ом            & 3,05  & 3,59  & 5,81 & 4,93 & 1,53 & 3,16 & -     \\ \hline
		\(\Delta P_\textup{л(1)}\), МВт & 1,15  & 1,23  & 0,44 & 0,30 & 0,81 & 0,37 & 4,30  \\ \hline
	\end{tabularx}
\end{table}

\begin{table}[h]
	\small
	\caption{Нагрузочные потери в линиях для варианта схемы сети 2}
	\label{tab:нагр_потери_2}
	\begin{tabularx}{\linewidth}{|Z|Z|Z|Z|Z|Z|Z|}
		\hline
		Линия                           & К-1   & 1-3   & 3-2  & 3-4  & 4-5  & Итого \\ \hline
		\(U_\textup{ном}\), кВ & 220 & 220 & 110 & 110 & 110 & - \\ \hline
		\textit{S}, МВА                          & 262,5 & 188,7 & 73,9 & 78,6 & 36,9 &  -    \\ \hline
		\(R_\textup{л}\), Ом            & 1,53  & 2,36  & 2,47 & 1,53 & 3,16 &  -    \\ \hline
		\(\Delta P_\textup{л(2)}\), МВт & 2,18  & 1,73  & 1,11 & 0,78 & 0,36 & 6,17  \\ \hline
	\end{tabularx}
\end{table}

В соответствии с исходными данными время наибольших нагрузок \(T_\textup{нб} = 4350\; \frac{\textup{ч}}{\textup{год}}\). Тогда время наибольших потерь:
\[\tau = \frac{1}{3}\cdot T_\textup{нб} + \frac{2}{3}\cdot \frac{T_\textup{нб}^2}{8760} = \frac{1}{3}\cdot 4350 + \frac{2}{3}\cdot \frac{4350^2}{8760} = 2890,1\; \frac{\textup{ч}}{\textup{год}}\]

Рассчитаем нагрузочные потери активной мощности в трансформаторах: 2xТРДН-63000/220 для варианта схемы сети 1 и 2xТРДН-63000/110 для варианта схемы сети 2, установленных на ПС2 (с учетом компенсации реактивной мощности):
\[\Delta P_\textup{Т2(1)} = \frac{S_\textup{нб2(1)}^{''2}\cdot R_\textup{Т2(1)}}{U_\textup{ном}^2\cdot n_\textup{т}} = \frac{72,9^2\cdot 3,9}{220^2\cdot 2} = 0,214\; \textup{МВт};\]
\[\Delta P_\textup{Т2(2)} = \frac{S_\textup{нб2(2)}^{''2}\cdot R_\textup{Т2(2)}}{U_\textup{ном}^2\cdot n_\textup{т}} = \frac{72,2^2\cdot 0,87}{110^2\cdot 2} = 0,187\; \textup{МВт},\]
где \(R_\textup{т2(1)}\) и \(R_\textup{т2(2)}\) "--- активные сопротивления обмоток трансформаторов ТРДН-63000/220 и ТРДН-63000/110, соответственно.

Проведем расчет нагрузочных потерь активной мощности в двух автотрансформаторах АТДЦТН-125000/220/110, установленных на ПС3, для варианта схемы сети 1 (с учетом компенсации реактивной мощности):
\[\Delta P_\textup{ат3(1)ВН} = \frac{S_\textup{нб.ВН(1)}^{''2}\cdot R_\textup{ат3(1)(ВН)}}{U_\textup{ном}^2\cdot n_\textup{ат}} = \frac{111,4^2\cdot 0,52}{220^2\cdot 2} = 0,0667\; \textup{МВт};\]
\[\Delta P_\textup{ат3(1)CН} = \frac{S_\textup{нб.CН(1)}^{''2}\cdot R_\textup{ат3(1)(CН)}}{U_\textup{ном}^2\cdot n_\textup{ат}} = \frac{80,2^2\cdot 0,52}{220^2\cdot 2} = 0,0346\; \textup{МВт};\]
\[\Delta P_\textup{ат3(1)НН} = \frac{S_\textup{нб.НН(1)}^{''2}\cdot R_\textup{ат3(1)(НН)}}{U_\textup{ном}^2\cdot n_\textup{ат}} = \frac{31,3^2\cdot 3,2}{220^2\cdot 2} = 0,0323\; \textup{МВт},\]
где \(R_\textup{ат3(1)(ВН)}\), \(R_\textup{ат3(1)(СН)}\) и \(R_\textup{ат3(1)(НН)}\) "--- каталожные данные по активным сопротивлениям обмоток автотрансформатора АТДЦТН-125000/220/110 \cite{файбисович}.
\[\Delta P_\textup{ат(1)} = \Delta P_\textup{ат3(1)ВН} + \Delta P_\textup{ат3(1)CН} + \Delta P_\textup{ат3(1)НН} =\] \[= 0,0667 + 0,0346 + 0,0323 = 0,134\; \textup{МВт}\]

Проведем расчет нагрузочных потерь активной мощности в двух автотрансформаторах АТДЦТН-200000/220/110, установленных на ПС3, для варианта схемы сети 2 (с учетом компенсации реактивной мощности):
\[\Delta P_\textup{ат3(2)ВН} = \frac{S_\textup{нб.ВН3(2)}^{''2}\cdot R_\textup{ат3(2)(ВН)}}{U_\textup{ном}^2\cdot n_\textup{ат}} = \frac{183,8^2\cdot 0,3}{220^2\cdot 2} = 0,105\; \textup{МВт};\]
\[\Delta P_\textup{ат3(2)CН} = \frac{S_\textup{нб.CН3(2)}^{''2}\cdot R_\textup{ат3(2)(CН)}}{U_\textup{ном}^2\cdot n_\textup{ат}} = \frac{152,4^2\cdot 0,3}{220^2\cdot 2} = 0,0720\; \textup{МВт};\]
\[\Delta P_\textup{ат3(2)НН} = \frac{S_\textup{нб.НН3(2)}^{''2}\cdot R_\textup{ат3(2)(НН)}}{U_\textup{ном}^2\cdot n_\textup{ат}} = \frac{31,3^2\cdot 0,6}{220^2\cdot 2} = 0,0061\; \textup{МВт},\]
где \(R_\textup{ат3(2)(ВН)}\), \(R_\textup{ат3(2)(СН)}\) и \(R_\textup{ат3(2)(НН)}\) "--- каталожные данные по активным сопротивлениям обмоток автотрансформатора АТДЦТН-200000/220/110 \cite{файбисович}.
\[\Delta P_\textup{ат3(2)} = \Delta P_\textup{ат3(2)ВН} + \Delta P_\textup{ат3(2)CН} + \Delta P_\textup{ат3(2)НН} =\] \[=0,105 + 0,0720 + 0,0061 = 0,183\; \textup{МВт}\]

Для остальных ПС процентное соотношение мощностей \(S_\textup{нб}^{''}\) составляет не более 1 \%, поэтому учитывать нагрузочные потери в обмотках трансформаторов этих подстанций не будем.

Суммарные годовые нагрузочные потери электроэнергии для вариантов схемы сети 1 и 2:
\[\Delta \textup{Э}_\textup{нагр(1)} = \Delta P_\textup{нагр(1)}\cdot \tau = (\Delta P_\textup{л(1)} + \Delta P_\textup{Т2(1)} + \Delta P_\textup{ат3(1)})\cdot \tau =\] \[= (4,30 + 0,214 + 0,134)\cdot 2890,1 = 13433,2\; \frac{\textup{МВт}\cdot \textup{ч}}{\textup{год}}\]
\[\Delta \textup{Э}_\textup{нагр(2)} = \Delta P_\textup{нагр(2)}\cdot \tau = (\Delta P_\textup{л(2)} + \Delta P_\textup{Т2(2)} + \Delta P_\textup{ат3(2)})\cdot \tau =\] \[= (6,17 + 0,187 + 0,183)\cdot 2890,1 = 18901,3\; \frac{\textup{МВт}\cdot \textup{ч}}{\textup{год}}\]

\textit{Условно-постоянные потери активной мощности}

Варианты схемы сети 1 и 2 отличаются потерями холостого хода в разных трансформаторах на ПС 2 и автотрансформаторах на ПС 3.. Величина условно постоянных потерь для вариантов схемы сети 1 и 2 будет определяться:
\[\Delta P_\textup{усл.пост(1)} = \Delta P_\textup{х.экв.тр2(1)} + \Delta P_\textup{х.экв.ат3(1)} =\] \[= 82\cdot 2 + 65\cdot 2 = 294\; \textup{кВт} = 0,294\; \textup{МВт};\]
\[\Delta P_\textup{усл.пост(2)} = \Delta P_\textup{х.экв.тр2(2)} + \Delta P_\textup{х.экв.ат3(2)} =\] \[= 59\cdot 2 + 125\cdot 2 = 310\; \textup{кВт} = 0,368\; \textup{МВт},\]
где \(\Delta P_\textup{х.экв.тр2(1)}\) и \(\Delta P_\textup{х.экв.тр2(2)}\) "--- каталожные данные по потерям холостого хода трансформаторов ТРДН-63000/220 и ТРДН-63000/110 соответственно; \(\Delta P_\textup{х.экв.ат3(1)}\) и \(\Delta P_\textup{х.экв.ат3(2)}\) "--- каталожные данные по потерям холостого хода автотрансформаторов АТДЦТН-125000/220/110 и АТДЦТН-200000/220/110 соответственно, приведенные в справочнике \cite{файбисович}.

Суммарные годовые условно-постоянные потери электроэнергии для вариантов схемы сети 1 и 2:
\[\Delta \textup{Э}_{\textup{усл.пост(1)}\Sigma} = \Delta P_\textup{усл.пост(1)}\cdot T_\textup{год} = 0,294\cdot 8760 = 2575,4\; \frac{\textup{МВт}\cdot \textup{ч}}{\textup{год}}\]
\[\Delta \textup{Э}_{\textup{усл.пост(2)}\Sigma} = \Delta P_\textup{усл.пост(2)}\cdot T_\textup{год} = 0,368\cdot 8760 = 3223,7\; \frac{\textup{МВт}\cdot \textup{ч}}{\textup{год}}\]

Суммарные ежегодные потери электроэнергии для вариантов схемы сети 1 и 2:
\[\Delta \textup{Э}_{(1)\Sigma} = \Delta \textup{Э}_\textup{нагр(1)} + \Delta \textup{Э}_{\textup{усл.пост(1)}\Sigma} = 13433,2 + 2575,4 = 16008,6\; \frac{\textup{МВт}\cdot \textup{ч}}{\textup{год}}\]
\[\Delta \textup{Э}_{(2)\Sigma} = \Delta \textup{Э}_\textup{нагр(2)} + \Delta \textup{Э}_{\textup{усл.пост(2)}\Sigma} = 18901,3 + 3223,7 = 22125\; \frac{\textup{МВт}\cdot \textup{ч}}{\textup{год}}\]

Суммарные ежегодные издержки на возмещение потерь электроэнергии в элементах электрической сети:
\[\textup{И}_{\textup{пот}\Sigma} = \textup{с}_\textup{э}\cdot \Delta \textup{Э}_\Sigma,\]
где \(\textup{с}_\textup{э} = 1,75\; \frac{\textup{тыс. руб.}}{\textup{МВт}\cdot \textup{ч}}\) "--- стоимость потерь электроэнергии в Костромской области на 2010 год \cite{глазунов_шведов}.

Рассчитаем суммарные ежегодные издержки на возмещение потерь электроэнергии в элементах электрической сети для двух вариантов схемы сети:
\[\textup{И}_{\textup{пот}\Sigma(1)} = \textup{с}_\textup{э}\cdot \Delta \textup{Э}_{(1)\Sigma} = 1,75\cdot 16008,6 = 28015,1\; \frac{\textup{тыс. руб}}{\textup{год}}\]
\[\textup{И}_{\textup{пот}\Sigma(2)} = \textup{с}_\textup{э}\cdot \Delta \textup{Э}_{(2)\Sigma} = 1,75\cdot 22125 = 38718,75\; \frac{\textup{тыс. руб}}{\textup{год}}\]

Определим по таблицам П2.12, П2.14, П2.15, П2.16 приложения 2 \cite{глазунов_шведов} значения множителей, необходимых для расчета дисконтированных затрат:
\begin{itemize}
	\item дисконтирующий множитель \(D_\textup{д} = 5,759\);
	\item расчетный множитель для воздушных линий напряжением 35 кВ и выше на железобетонных опорах \(D_\textup{р.ЛЭП} = 0,813\);
	\item расчетный множитель для силового электрооборудования и коммутационной аппаратуры ПС при высшем напряжении 110 кВ \(D_\textup{р.ПС.110} = 1,107\);
	\item расчетный множитель для силового электрооборудования и коммутационной аппаратуры ПС при высшем напряжении 220 кВ \(D_\textup{р.ПС.220} = 1,049\).
\end{itemize}

Для удобства расчета при учете компонентов на ПС разного уровня напряжения, преобразуем формулу \eqref{eqn:упрощ.формула_затрат} в следующий вид:
\[\textup{З}_\textup{д} = \textup{К}_{\textup{ЛЭП}\Sigma}\cdot D_\textup{р.ЛЭП} + \textup{К}_{\textup{ПС.110}\Sigma}\cdot D_\textup{р.ПС.110} + \textup{К}_{\textup{ПС.220}\Sigma}\cdot D_\textup{р.ПС.220} + \textup{И}_{\textup{пот}\Sigma}\cdot D_\textup{д}\]

Величина дисконтированных затрат для вариантов схемы сети 1 и 2:
\[\textup{З}_\textup{д1} = \textup{К}_{\Sigma ЛЭП(1)}\cdot D_\textup{р.ЛЭП} + \textup{К}_{\Sigma ПС.110(1)}\cdot D_\textup{р.ПС.110} + \textup{К}_{\Sigma ПС.220(1)}\cdot D_\textup{р.ПС.220} + \textup{И}_{\textup{пот(1)}\Sigma}\cdot D_\textup{д} =\] \[= 842348,5\cdot 0,813 + 0\cdot 1,107 +\] \[+ 471750\cdot 1,049 + 28015,1\cdot 5,759 = 1341034,0\; \textup{тыс. руб}\]
\[\textup{З}_\textup{д2} = \textup{К}_{\Sigma ЛЭП(2)}\cdot D_\textup{р.ЛЭП} + \textup{К}_{\Sigma ПС.110(2)}\cdot D_\textup{р.ПС.110} + \textup{К}_{\Sigma ПС.220(2)}\cdot D_\textup{р.ПС.220} + \textup{И}_{\textup{пот(2)}\Sigma}\cdot D_\textup{д} =\] \[= 1065830,3\cdot 0,813 + 215687,5\cdot 1,107 +\] \[ 365925\cdot 1,049 + 38718,8\cdot 5,759 = 1712122,9\; \textup{тыс. руб}\]

Разница дисконтированных затрат вариантов схемы сети 1 и 2:
\[\Delta \textup{З}_\textup{д} = \frac{\textup{З}_\textup{д2} - \textup{З}_\textup{д1}}{\textup{З}_\textup{д2}}\cdot 100\% = \frac{1712122,9 - 1341034,0}{1712122,9}\cdot 100\% = 21,7 \%\]

Варианты схемы сети 1 и 2 отличаются по дисконтированным затратам на 21,7 \% > 5 \%, следовательно, варианты схемы сети 1 и 2 не являются равноэкономичными. В дальнейшем будем рассматривать вариант схемы сети № 1, так как он более выгодный.

%%% Local Variables:
%%% mode: latex
%%% TeX-master: "rpz"
%%% End: