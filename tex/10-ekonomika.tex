\chapter{Основные технико-экономические показатели спроектированной сети}
\label{cha:экономика}

\section{Суммарные капиталовложения на сооружение спроектированной сети}

Суммарные капиталовложения \(K_\Sigma\) учитывают полную стоимость сооружения всех линий электропередачи и понижающих подстанций сети от шин источника питания до шин 10 кВ понижающих ПС.

\textit{Капиталовложения на сооружение линий электропередачи}

Расчеты для кольцевой сети 220 кВ возьмем из раздела \ref{cha:рациональная_схема}.

В качестве примера рассчитаем капиталовложения на сооружение линии 3-4 длиной 25,9 км, сооружаемой на железобетонных двухцепных опорах с проводами марки АС 240/32 в I-м районе по гололеду \eqref{eqn:капиталовложения_лэп}:
\[K_\textup{ЛЭП(34)} = \textup{К}_{0\textup{ЛЭП.баз(34)}}\cdot k_\textup{зон}\cdot k_\textup{усл}\cdot k_\textup{деф}\cdot L_{34} =\] \[= 1403\cdot 1\cdot 1\cdot 4,25\cdot 25,9 = 154435,2\; \textup{тыс. руб}\]

Результаты расчетов для всех линий электропередачи сведены в таблицу \ref{tab:капиталовложения_вл}.

\begin{table}[H]
	\small
	\caption{Капиталовложения на сооружение ВЛ}
	\label{tab:капиталовложения_вл}
	\begin{tabularx}{\linewidth}{|>{\hsize=1.3\hsize\linewidth=\hsize}Z|>{\hsize=0.95\hsize\linewidth=\hsize}Z|>{\hsize=0.95\hsize\linewidth=\hsize}Z|>{\hsize=0.95\hsize\linewidth=\hsize}Z|>{\hsize=0.95\hsize\linewidth=\hsize}Z|>{\hsize=0.95\hsize\linewidth=\hsize}Z|>{\hsize=0.95\hsize\linewidth=\hsize}Z|}
		\hline
		ЛЭП & К-1 & К-2 & 1-3 & 2-3 & 3-4 & 4-5 \\ \hline
		\(U_\textup{ном}\), кВ & 220 & 220 & 220 & 220 & 110 & 110  \\ \hline
		\(F_a\; \textup{мм}^2\) & 400/51 & 400/51 & 240/32 & 240/32 & 240/32 & 120/19\\ \cline{1-7}
		\(n_\textup{ц}\) & 1 & 1 & 1 & 1 & 2 & 2 \\ \hline
		\(\textup{К}_\textup{0.ЛЭП},\; \frac{\textup{тыс. руб}}{\textup{км}}\) & 1188 & 1188 & 990 & 990 & 1403 & 1148 \\ \hline
		\(L\), км & 41,8 & 49,2 & 49,2 & 41,8 & 25,9 & 25,9 \\ \hline
		\(\textup{К}_\textup{ЛЭП}\), тыс. руб & 211048,2 & 248410,8 & 207009,0 & 175873,5 & 154435,2 & 126366,1 \\ \hline
	\end{tabularx}
\end{table}

Итого: \(K_{\textup{ЛЭП}\Sigma} = 1123142,8\) тыс. руб.

\textit{Капиталовложения на сооружение понижающих подстанций}

Капиталовложения на сооружение понижающих подстанций рассчитываются по формуле \eqref{eqn:капит_на_сооруж_пс}:
\begin{equation*}
	\textup{К}_\textup{ПС} = \textup{К}_{\textup{ТР}\Sigma} + \textup{К}_{\textup{РУ}\Sigma} + \textup{К}_{\textup{доп}\Sigma} + \textup{К}_{\textup{пост}}
\end{equation*}
%где  \(\textup{К}_{\textup{тр}\Sigma}\) – суммарные капиталовложения в \(n_{т}\) однотипных трансформаторов, устанавливаемых на ПС, тыс.руб; \(\textup{К}_{\textup{ру}\Sigma}\) – суммарные капиталовложения в РУ всех классов напряжения ПС, тыс.руб; \(\textup{К}_{\textup{доп}\Sigma}\) – суммарные капиталовложения в дополнительное оборудование, устанавливаемое на ПС, тыс.руб; \(\textup{К}_{пост}\) – постоянная часть затрат, тыс.руб.

Приведем пример расчета капиталовложения на сооружение подстанций для ПС1. На ПС1 установлены 2 двухобмоточных трансформатора ТРДН-63000/220, для которых \(\textup{К}_{\textup{ТР.баз}_1} = 12625\) тыс. руб. Тогда стоимость двух трансформаторов на подстанции:
\[K_\textup{ТР1} = K_{\textup{ТР.баз}_1}\cdot k_\textup{зон}\cdot k_\textup{деф}\cdot n_\textup{т} = 12625\cdot 1\cdot 4,25\cdot 2 = 107312,5\; \textup{тыс. руб.}\]

РУВН выполнено по схеме «четырехугольник», в которой используются 4 воздушных выключателя с базовым укрупненным показателем стоимости \(\textup{К}_\textup{выкл.220.баз} = 8800\) тыс. руб.

Капиталовложения в РУВН ПС1:
\[\textup{К}_\textup{РУВН1} = \textup{К}_\textup{выкл.220.баз}\cdot k_\textup{зон}\cdot k_\textup{деф}\cdot n_\textup{яч1}^{220} = 8800\cdot 1\cdot 4,25\cdot 4 = 149600\; \textup{тыс. руб.}\]

Суммарное количество ячеек выключателей 10 кВ распределительного устройства низшего напряжения ПС1 было рассчитано в разделе 7: \(n_\textup{яч1}^{10} = 30\).

В РУНН 10 кВ используются вакуумные выключатели, для которых:
\[\textup{К}_\textup{выкл.10.баз} = 120\; \textup{тыс. руб.}\]

Тогда стоимость распределительного устройства низшего напряжения:
\[\textup{К}_\textup{РУНН1} = \textup{К}_\textup{выкл.10.баз.}\cdot k_\textup{зон}\cdot k_\textup{деф}\cdot n_\textup{яч1}^{10} = 120\cdot 1\cdot 4,25\cdot 30 = 15300\; \textup{тыс. руб.}\]

Суммарные капиталовложения в распределительные устройства ПС1:
\[\textup{К}_{\textup{РУ1}\Sigma} = \textup{К}_{\textup{РУВН}_1} + \textup{К}_{\textup{РУНН}_1} = 149600 + 15300 = 164900\; \textup{тыс. руб}\]

Определим стоимость дополнительного оборудования. На ПС1 установлены 8 батарей статических конденсаторов. Базовая укрупненная стоимость шунтовой конденсаторной батареи 10 кВ единичной мощностью 1,2 Мвар: \(\textup{К}_\textup{БСК.баз} = 375\) тыс. руб.

Суммарные капиталовложения в дополнительное оборудование ПС1:
\[\textup{К}_\textup{доп1} = \textup{К}_\textup{БСК.баз}\cdot k_\textup{зон}\cdot k_\textup{деф}\cdot N_{\textup{БСК}_1} = 375\cdot 1\cdot 4,25\cdot 8 = 12750\; \textup{тыс. руб.}\]

Определим постоянную часть затрат на сооружение ПС1 напряжением 220/10 кВ со схемой РУВН «четырехугольник>>, для которой \(\textup{К}_\textup{пост.220.баз} = 19500\) тыс.руб.

Постоянная часть затрат на сооружение ПС1:
\[\textup{К}_\textup{пост1} = \textup{К}_\textup{пост.220.баз}\cdot k_\textup{зон}\cdot k_\textup{деф} = 19500\cdot 1\cdot 4,25 = 82875\; \textup{тыс. руб.}\]

Тогда суммарные капиталовложения на сооружение ПС1:
\[\textup{К}_\textup{ПС1} = \textup{К}_{\textup{ТР1}\Sigma} + \textup{К}_{\textup{РУ1}\Sigma} + \textup{К}_\textup{доп1} + \textup{К}_\textup{пост1} =\] \[= 107312,5 + 164900 + 12750 + 82875 = 367837,5\; \textup{тыс. руб.}\]

Для остальных подстанций расчет аналогичный, результаты расчетов сведены в таблицу \ref{tab:расчет_капиталовложений_пс}.

%\begin{table}[H]
{	\small
	\begin{xltabular}{\linewidth}{|>{\hsize=1.5\hsize\linewidth=\hsize}Z|>{\hsize=0.9\hsize\linewidth=\hsize}Z|>{\hsize=0.9\hsize\linewidth=\hsize}Z|>{\hsize=0.9\hsize\linewidth=\hsize}Z|>{\hsize=0.9\hsize\linewidth=\hsize}Z|>{\hsize=0.9\hsize\linewidth=\hsize}Z|}
		\caption{Капиталовложения на сооружение подстанций проектируемой сети} 
		\label{tab:расчет_капиталовложений_пс} \\ \hline
		\endfirsthead
		\caption{\textit{(Продолжение)} Капиталовложения на сооружение подстанций проектируемой сети}\\
		\hline
		\endhead
		\multicolumn{6}{r}{\textit{Продолжение на следующей странице}} \\
		\endfoot
		\endlastfoot
		№ ПС                                           & 3 (АТ)   & 1        & 2        & 4        & 5       \\ \hline
		\(\textup{К}_\textup{ТР.баз}\), тыс. руб       & 15525    & 12625    & 12625    & 7300     & 5500    \\ \hline
		\(\textup{К}_{\textup{ТР}\Sigma}\), тыс. руб   & 131962,5 & 107312,5 & 107312,5 & 62050,0  & 46750   \\ \hline
		\(\textup{К}_\textup{выкл.220.баз}\), тыс. руб & 8800     & 8800     & 8800     & -        & -       \\ \hline
		\(n_\textup{яч}^{220}\)                        & 4        & 4        & 4        & -        & -       \\ \hline
		\(\textup{К}_\textup{РУ.220}\), тыс. руб       & 149600   & 149600   & 149600   & -        & -       \\ \hline
		\(\textup{К}_\textup{выкл.110.баз}\), тыс. руб & 4150     & -        & -        & 4150     & 4150    \\ \hline
		\(n_\textup{яч}^{110}\)                        & 7        & -        & -        & 9        & 3       \\ \hline
		\(\textup{К}_\textup{РУ.110}\), тыс. руб       & 123462,5 & -        & -        & 158737,5 & 52912,5 \\ \hline
		\(\textup{К}_\textup{выкл.10.баз}\), тыс. руб  & 120      & 120      & 120      & 120      & 120     \\ \hline
		\(n_\textup{яч}^{10}\)                         & 15       & 30       & 30       & 22       & 22      \\ \hline
		\(\textup{К}_\textup{РУ.10}\), тыс. руб        & 7650     & 15300    & 15300    & 11220    & 11220   \\ \hline
		\(\textup{К}_{\textup{РУ}\Sigma}\), тыс. руб   & 280712,5 & 164900   & 164900   & 169957,5 & 64132,5 \\ \hline
		\(\textup{К}_\textup{БСК.баз}\), тыс. руб      & 375      & 375      & 375      & 375      & 375     \\ \hline
		\(N_\textup{БСК}\)                             & 4        & 8        & 8        & 8        & 4       \\ \hline
		\(\textup{К}_\textup{БСК}\), тыс. руб          & 6375     & 12750    & 12750    & 12750    & 6375    \\ \hline
		\(\textup{К}_\textup{ЛРТ.баз}\), тыс. руб      & 3750     & -        & -        & -        & -       \\ \hline
		\(n_\textup{ЛРТ}\)                             & 2        & -        & -        & -        & -       \\ \hline
		\(\textup{К}_\textup{ЛРТ}\), тыс. руб          & 31875    & -        & -        & -        & -       \\ \hline
		\(\textup{К}_\textup{пост.баз}\), тыс. руб     & 30000    & 19500    & 19500    & 12250    & 9000    \\ \hline
		\(\textup{К}_\textup{пост}\), тыс. руб         & 127500   & 82875    & 82875    & 52062,5  & 38250   \\ \hline
		\(\textup{К}_\textup{ПС}\), тыс. руб           & 578425   & 367837,5 & 367837,5 & 296820   & 155507,5  \\ \hline
	\end{xltabular}
}
%\end{table}

Капиталовложения в источник питания:
\[\textup{К}_\textup{ИП} = \textup{К}_\textup{выкл.220.баз}\cdot k_\textup{зон}\cdot k_\textup{деф}\cdot n_\textup{яч} = 8800\cdot 1\cdot 4,25\cdot 2 = 74800\; \textup{тыс. руб}\]

Суммарные капиталовложения в оборудование подстанций 220 кВ:
\[\textup{К}_\textup{ПС220} = \textup{К}_\textup{ПС1} + \textup{К}_\textup{ПС2} + \textup{К}_\textup{ПС3} + \textup{К}_\textup{ИП} =\] \[367837,5 + 367837,5 + 578425 + 74800 = 1388900\; \textup{тыс. руб.}\]

Суммарные капиталовложения в оборудование подстанций 110 кВ:
\[\textup{К}_\textup{ПС110} = \textup{К}_\textup{ПС4} + \textup{К}_\textup{ПС5} = 296820 + 155507,5 = 452327,5\; \textup{тыс. руб.}\]

Суммарные капиталовложения на сооружение понижающих подстанций:
\[\textup{К}_{\textup{ПС}\Sigma} = \textup{К}_\textup{ПС220} + \textup{К}_\textup{ПС110} = \] \[= 1388900 + 452327,5 = 1841227,5\; \textup{тыс. руб.}\]

Суммарные капиталовложения на сооружение спроектированной сети:
\[\textup{К}_\Sigma = \textup{К}_{\textup{ПС}\Sigma} + \textup{К}_{\textup{ЛЭП}\Sigma} =\] \[= 1841227,5 + 1123142,8 = 2964370,3\; \textup{тыс. руб.}\]

\section{Потери активной мощности и годовые потери электроэнергии в сети}

\textit{Потери активной мощности в линиях электропередачи на корону}

По табл. 3.10 \cite{файбисович} находим удельные среднегодовые потери мощности на корону для ВЛ 220 кВ с одним проводом в фазе, имеющим сечение алюминиевой части \(F_a = 300\; \textup{мм}^2\):
\[\Delta P_\textup{уд.кор.300} = 0,84\; \frac{\textup{кВт}}{\textup{км}}\]

Для ВЛ 110 кВ при \(F_a = 120\; \textup{мм}^2\):
\[\Delta P_\textup{уд.кор.120} = 0,08\; \frac{\textup{кВт}}{\textup{км}}\]

Для проводов ВЛ 220 кВ, имеющих сечение алюминиевой части, отличное от \(300\; \textup{мм}^2\), значение удельных среднегодовых потерь мощности на корону можно оценить по выражению:
\begin{equation}
	\Delta P_\textup{уд.кор.} = \Delta P_\textup{уд.кор.300}\cdot \frac{300}{F_a}
	\label{eqn:потери_на_корону_220}
\end{equation}

Для проводов ВЛ 110 кВ:
\begin{equation}
	\Delta P_\textup{уд.кор.} = \Delta P_\textup{уд.кор.120}\cdot \frac{120}{F_a}
	\label{eqn:потери_на_корону_110}
\end{equation}

Тогда с учетом формулы \eqref{eqn:потери_на_корону_220} среднегодовые потери мощности на корону для одноцепной ВЛ 220 кВ с проводами АС 400/51 протяженностью 41,8 км будут равны:
\[\Delta P_\textup{кор.К1} = \Delta P_\textup{уд.кор.300}\cdot \frac{300}{F_{a\textup{К1}}}\cdot  L_\textup{К1}\cdot n_\textup{ц.К1} = 0,84\cdot 10^{-3}\cdot \frac{300}{400}\cdot 41,8\cdot 1 = 0,0263\; \textup{МВт}\]

Для ВЛ 110 кВ среднегодовые потери мощности на корону считаются с учетом формулы \eqref{eqn:потери_на_корону_110}.

Для остальных линий расчет проводится аналогично. Результаты расчетов сведены в таблицу \ref{tab:потери_на_корону}.

\begin{table}[H]
	\small
	\caption{Потери активной мощности в линиях на корону}
	\label{tab:потери_на_корону}
	\begin{tabularx}{\linewidth}{|Z|Z|Z|Z|Z|Z|Z|Z|}
		\hline
		Линия                          & К-1    & К-2    & 1-3    & 2-3    & 3-4    & 3-5    & Итого                 \\ \hline
		\(n_\textup{ц}\)               & 1      & 1      & 1      & 1      & 2      & 2      & \multirow{5}{*}{0,159} \\ \cline{1-7}
		\textit{L}, км                 & 41,8   & 49,2   & 49,2   & 41,8   & 25,9   & 25,9   &                       \\ \cline{1-7}
		\(U_\textup{ном}\), кВ         & 220    & 220    & 220    & 220    & 110    & 110    &                       \\ \cline{1-7}
		\(F\;, \textup{мм}^2\)          & 400/51 & 400/51 & 240/32 & 240/32 & 240/32 & 120/19 &                       \\ \cline{1-7}
		\(\Delta P_\textup{кор}\), МВт & 0,0263  & 0,0310  & 0,0517  & 0,0439  & 0,00207 & 0,00414 &                       \\ \hline
	\end{tabularx}
\end{table}

\textit{Диэлектрические потери в батареях статических конденсаторов:}
\[\Delta P_{\textup{БСК}\Sigma} = 0,003\cdot N_{\textup{БСК}\Sigma}\cdot Q_\textup{БСК(1)} = 0,003\cdot 32\cdot 1,2 = 0,115\; \textup{МВт},\]
где \(N_{\textup{БСК}\Sigma}\) "--- суммарное количество БСК, установленных на всех ПС; \(Q_\textup{БСК(1)}\) "--- установленная мощность одного комплекта БСК.

Суммарные потери активной мощности в любом режиме работы спроектированной электрической сети, за исключением потерь на корону в проводах воздушных линий и диэлектрических потерь в батареях статических конденсаторов, наиболее просто могут быть определены по разности суммарной активной мощности, поступающей в сеть с шин источника питания, и полезно отпускаемой потребителям активной мощности \cite{глазунов_шведов}.

\textit{Потери активной мощности без учета потерь на корону и диэлектрических потерь в БСК:}
\[\Delta P = P_\textup{г} - P_{\textup{нб}\Sigma} = 249,97 - 245 = 4,97\; \textup{МВт},\]
где \(P_\textup{г} = 249,97\) МВт "--- активная мощность, поступающая в спроектированную сеть с шин 220 кВ ПС К,  рассчитанная в ПВК RastrWin3 (см. рис. \ref{fig:узлы_нб_отрег}); \(P_{\textup{нб}\Sigma} = 245\) МВт "--- полезно отпускаемая потребителям активная мощность.

\textit{Суммарные потери активной мощности в спроектированной сети:}
\[\Delta P_\Sigma = \Delta P + \Delta P_{\textup{кор}\Sigma} + \Delta P_{\textup{БСК}\Sigma} = 4,97 + 0,159 + 0,115 = 5,24\; \textup{МВт}\]
\[\Delta P_{\Sigma \%} = \frac{\Delta P_\Sigma}{P_{\textup{нб}\Sigma}}\cdot 100\% = \frac{5,24}{245}\cdot 100\% = 2,14 \%\]

Полученное значение показывает допустимость суммарных потерь активной мощности в режиме наибольших нагрузок (\(\Delta P_{\Sigma \%}\) для сетей 110-220 кВ не должно превышать 4-5\%) \cite{глазунов_шведов}.

Эквивалентные активные потери холостого хода трансформаторов и автотрансформаторов, установленных на ПС проектируемой сети, представлены в таблице \ref{tab:потери_хх}

\begin{table}[H]
	\small
	\caption{Эквивалентные активные потери холостого хода ТР и АТ, установленных на ПС проектируемой сети}
	\label{tab:потери_хх}
	\begin{tabularx}{\linewidth}{|Z|Z|Z|Z|Z|Z|Z|}
		\hline
		№ ПС                             & 1 & 2 & 3 & 4 & 5 & \(\Delta P_{\textup{х}\Sigma}\) \\ \hline
		\(\Delta P_\textup{х.экв}\), МВт & 0,164  & 0,164 & 0,130 & 0,072 & 0,054 & 0,584 \\ \hline 
	\end{tabularx}
\end{table}

\textit{Условно-постоянные потери активной мнощности:}
\[\Delta P_\textup{усл-пост} = \Delta P_{\textup{кор}\Sigma} + \Delta P_{\textup{БСК}\Sigma} + \Delta P_{\textup{х}\Sigma} = 0,159 + 0,115 + 0,584 = 0,858\; \textup{МВт}\]

\textit{Нагрузочные потери активной мощности:}
\[\Delta P_\textup{нагр} = \Delta P_\Sigma - \Delta P_\textup{усл-пост} = 5,24 - 0,858 = 4,38\; \textup{МВт}\]

\textit{Расчет годовых потерь электроэнергии в сети:}
\[T_\textup{нб} = 4350\; \frac{\textup{ч}}{\textup{год}}; \ \tau = 2890,1\; \frac{\textup{ч}}{\textup{год}}\; \textup{(см. пункт 8.1)}\]

Суммарная полезно отпущенная потребителям с шин 10 кВ всех подстанций электроэнергия:

\[\textup{Э}_{\textup{отп}\Sigma} = P_{\textup{нб}\Sigma} \cdot T_\textup{нб} = 245\cdot 4350 = 1065750\; \frac{\textup{МВт}\cdot \textup{ч}}{\textup{год}}\]

Нагрузочные годовые потери электроэнергии:
\[\Delta \textup{Э}_\textup{нагр} = \Delta P_\textup{нагр}\cdot \tau = 4,38\cdot 2890,1 = 12658,6\; \frac{\textup{МВт}\cdot \textup{ч}}{\textup{год}}\]

Время работы линий электропередачи и трансформаторов в течение года принимаются равным 8760 ч, конденсаторных батарей "--- 6000 ч \cite{глазунов_шведов}. Тогда условно-постоянные годовые потери электроэнергии:
\[\Delta \textup{Э}_\textup{усл-пост} = \Delta P_\textup{БСК} \cdot 6000 + (\Delta P_{\textup{кор}\Sigma} + \Delta P_{\textup{х}\Sigma})\cdot 8760 =\] \[ = 0,115\cdot 6000 + (0,159 + 0,584)\cdot 8760 = 7198,7\; \frac{\textup{МВт}\cdot \textup{ч}}{\textup{год}}\]

\textit{Суммарные годовые потери электроэнергии в рассматриваемой сети:}
\[\Delta \textup{Э}_\Sigma = \Delta \textup{Э}_\textup{нагр} + \Delta \textup{Э}_\textup{усл-пост} = \] \[= 12658,6 + 7198,7 = 19857,3\; \frac{\textup{МВт}\cdot \textup{ч}}{\textup{год}}\]
\[\Delta \textup{Э}_{\Sigma \%} = \frac{\Delta \textup{Э}_\Sigma}{\textup{Э}_{\textup{отп}\Sigma}}\cdot 100 \% = \frac{19857,3}{1065750}\cdot 100 \% = 1,86\%\]

\section{Суммарные ежегодные издержки на передачу электроэнергии по спроектированной сети}

Суммарные ежегодные издержки на передачу электроэнергии по спроектированной электрической сети состоят из суммарных ежегодных издержек на эксплуатацию элементов электрической сети \(\textup{И}_{\textup{экспл}\Sigma}\) и суммарных ежегодных издержек на возмещение потерь электроэнергии при её передаче по элементам электрической сети \(\textup{И}_{\textup{пот}\Sigma}\):
\begin{equation}
	 \textup{И}_\Sigma = \textup{И}_{\textup{экспл}\Sigma} + \textup{И}_{\textup{пот}\Sigma},
	 \label{eqn:сумм_издержки}
\end{equation}
где \(\textup{И}_{\textup{экспл}\Sigma}\) "--- ежегодные издержки на эксплуатацию элемента электрической сети, состоящие из издержек на амортизацию \(\textup{И}_\textup{ам}\), ремонты \(\textup{И}_\textup{рем}\) и обслуживание \(\textup{И}_\textup{обсл}\), тыс.руб/год; \(\textup{И}_{\textup{пот}\Sigma}\) "--- суммарные ежегодные издержки на возмещение потерь электроэнергии в элементах электрической сети, тыс.руб/год.

\begin{equation}
	\textup{И}_{\textup{пот}\Sigma} = \textup{с}_\textup{э}\cdot \Delta \textup{Э}_\Sigma,
	\label{eqn:издержки_на_потери}
\end{equation}
где \(\textup{с}_\textup{э}\) "--- стоимость потерь электроэнергии, \(\frac{\textup{руб}}{\textup{кВт}\cdot \textup{ч}}\) или \(\frac{\textup{тыс.руб}}{\textup{МВт}\cdot \textup{ч}}\); \(\Delta \textup{Э}_\Sigma\) "--- суммарные годовые потери электроэнергии в элементах электрической сети, \(\frac{\textup{МВт}\cdot \textup{ч}}{\textup{год}}\).

Суммарные ежегодные издержки на эксплуатацию сети:
\[\textup{И}_{\textup{экспл}\Sigma} = \textup{И}_\textup{экспл}^\textup{ЛЭП} + \textup{И}_\textup{экспл}^\textup{ПС.110} + \textup{И}_\textup{экспл}^\textup{ПС.220}\]

Ежегодные издержки эксплуатации элемента электрической сети \(\textup{И}_\textup{экспл}\) складываются из ежегодных издержек на амортизацию (реновацию) \(\textup{И}_\textup{ам}\), ремонты \(\textup{И}_\textup{рем}\) и обслуживание \(\textup{И}_\textup{обсл}\) объекта:
\begin{equation}
	\textup{И}_\textup{экспл} = \textup{И}_\textup{ам} + \textup{И}_\textup{рем} + \textup{И}_\textup{обсл} = (\textup{а}_\textup{ам} + \textup{а}_\textup{рем} + \textup{а}_\textup{обсл}) \cdot K = \textup{а}_\Sigma \cdot K,
	\label{eqn:издержки_экспл}
\end{equation}
где \(\textup{а}_\textup{рем}\), \(\textup{а}_\textup{ам}\), \(\textup{а}_\textup{обсл}\) "--- нормы отчисления \(\frac{\textup{о.е.}}{\textup{год}}\).

Рассчитанные по формуле \eqref{eqn:сумм_издержки} издержки используются для расчета себестоимости передачи электроэнергии по сети:
\begin{equation}
	\textup{с} = \frac{\textup{И}_\Sigma}{\textup{Э}_{\textup{отп}\Sigma}}
	\label{eqn:себестоимость_э/э}
\end{equation}

Тогда по формуле \eqref{eqn:издержки_экспл} суммарные ежегодные издержки на эксплуатацию линий электропередачи:
\[\textup{И}_\textup{экспл}^\textup{ЛЭП} = (\textup{а}_\textup{ам}^\textup{ЛЭП} + \textup{а}_\textup{рем}^\textup{ЛЭП} + \textup{а}_\textup{обсл}^\textup{ЛЭП})\cdot \textup{К}_{\textup{ЛЭП}\Sigma} = \textup{а}_\Sigma^\textup{ЛЭП}\cdot \textup{К}_{\textup{ЛЭП}\Sigma} = \] \[= \frac{5,8}{100}\cdot 1123142,8 = 65142,3\; \frac{\textup{тыс.руб.}}{\textup{год}}\]

Суммарные ежегодные издержки на эксплуатацию подстанций при высшем напряжении 110 кВ:
\[\textup{И}_\textup{экспл}^\textup{ПС.110} = (\textup{а}_\textup{ам}^\textup{ПС.110} + \textup{а}_\textup{рем}^\textup{ПС.110} + \textup{а}_\textup{обсл}^\textup{ПС.110})\cdot \textup{К}_{\textup{ПС.110}\Sigma} = \textup{а}_\Sigma^\textup{ПС.110}\cdot \textup{К}_{\textup{ПС.110}\Sigma} = \] \[= \frac{10,9}{100}\cdot 452327,5 = 49303,7\; \frac{\textup{тыс.руб.}}{\textup{год}}\]

Суммарные ежегодные издержки на эксплуатацию подстанций при высшем напряжении 220 кВ:
\[\textup{И}_\textup{экспл}^\textup{ПС.220} = (\textup{а}_\textup{ам}^\textup{ПС.220} + \textup{а}_\textup{рем}^\textup{ПС.220} + \textup{а}_\textup{обсл}^\textup{ПС.220})\cdot \textup{К}_{\textup{ПС.220}\Sigma} = \textup{а}_\Sigma^\textup{ПС.220}\cdot \textup{К}_{\textup{ПС.220}\Sigma} = \] \[= \frac{9,9}{100}\cdot 1388900 = 137501,1\; \frac{\textup{тыс.руб.}}{\textup{год}}\]

Суммарные ежегодные издержки на эксплуатацию сети:
\[\textup{И}_{\textup{экспл}\Sigma} = \textup{И}_\textup{экспл}^\textup{ЛЭП} + \textup{И}_\textup{экспл}^\textup{ПС.110} + \textup{И}_\textup{экспл}^\textup{ПС.220} =\] \[ = 65142,3 + 49303,7 + 137501,1 = 251947,1\; \frac{\textup{тыс.руб.}}{\textup{год}}\]

Суммарные ежегодные издержки на возмещение потерь электроэнергии по формуле \eqref{eqn:издержки_на_потери}:
\[\textup{И}_{\textup{пот}\Sigma} = \textup{с}_\textup{Э}\cdot \Delta \textup{Э}_\Sigma = 1,75\cdot 19857,3 = 34750,3\; \frac{\textup{тыс.руб.}}{\textup{год}}\]

Суммарные ежегодные издержки на передачу электроэнергии по спроектированной сети по формуле \eqref{eqn:сумм_издержки}:
\[\textup{И}_\Sigma = \textup{И}_{\textup{экспл}\Sigma} + \textup{И}_{\textup{пот}\Sigma} = 251947,1 + 34750,3 = 286697,4\; \frac{\textup{тыс.руб.}}{\textup{год}}\]

Себестоимость передачи электроэнергии по спроектированной сети по формуле \eqref{eqn:себестоимость_э/э}:
\[\textup{с} = \frac{\textup{И}_\Sigma}{\textup{Э}_{\textup{отп}\Sigma}} = \frac{286697,4}{1065750} = 0,269\; \frac{\textup{руб}}{\textup{кВт}\cdot \textup{ч}}\]


%%% Local Variables:
%%% mode: latex
%%% TeX-master: "rpz"
%%% End: