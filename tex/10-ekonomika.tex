\chapter{Основные технико-экономические показатели спроектированной сети}
\label{cha:экономика}

\section{Суммарные капиталовложения на сооружение спроектированной сети}

Суммарные капиталовложения \(K_\Sigma\) учитывают полную стоимость сооружения всех линий электропередачи и понижающих подстанций сети от шин источника питания до шин 10 кВ понижающих ПС.

\textit{Капиталовложения на сооружение линий электропередачи}

Расчеты для кольцевой сети 220 кВ возьмем из пункта \ref{cha:рациональная_схема}.

В качестве примера рассчитаем капиталовложения на сооружение линии 3-4 длиной 25,9 км, сооружаемой на железобетонных двухцепных опорах с проводами марки АС 240/32 в I-м районе по гололеду \eqref{eqn:капиталовложения_лэп}:
\[K_\textup{ЛЭП(34)} = 1403\cdot 1\cdot 4,25\cdot 1\cdot 25,9 = 154435,2\; \textup{тыс. руб}\]

Результаты расчетов для всех линий электропередачи сведены в таблицу \ref{tab:}.

\begin{table}[H]
	\small
	\begin{tabularx}{\linewidth}{|Z|Z|Z|Z|Z|Z|Z|}
		\hline
		ЛЭП & К-1 & К-2 & 1-3 & 2-3 & 3-4 & 4-5 \\ \hline
		\(U_\textup{ном}\), кВ & 220 & 220 & 220 & 220 & 110 & 110  \\ \hline
		\(F_a\; \textup{мм}^2\) & 400/51 & 400/51 & 240/32 & 240/32 & 240/32 & 120/19\\ \cline{1-7}
		\(n_\textup{ц}\) & 1 & 1 & 1 & 1 & 2 & 2 \\ \hline
		\(\textup{К}_\textup{0.ЛЭП},\; \frac{\textup{тыс. руб}}{\textup{км}}\) & 1188 & 1188 & 990 & 990 & 1403 & 1148 \\ \hline
		\(L\), км & 41,8 & 49,2 & 49,2 & 41,8 & 25,9 & 25,9 \\ \hline
		\(\textup{К}_\textup{ЛЭП}\), тыс. руб & 211048,2 & 248410,8 & 207009,0 & 175873,5 & 154435,2 & 126366,1 \\ \hline
	\end{tabularx}
\end{table}

Итого: \(K_{\textup{ЛЭП}\Sigma} = 1123142,8\) тыс. руб.

\textit{Капиталовложения на сооружение понижающих подстанций}

Капиталовложения на сооружение понижающих подстанций рассчитывается по формуле \eqref{eqn:капит_на_сооруж_пс}:
\begin{equation}
	\textup{К}_\textup{ПС} = \textup{К}_{\textup{ТР}\Sigma} + \textup{К}_{\textup{РУ}\Sigma} + \textup{К}_{\textup{доп}\Sigma} + \textup{К}_{\textup{пост}}
\end{equation}

Приведем пример расчета капиталовложения на сооружение подстанций для ПС1:
На ПС1 установлены 2 двухобмоточных трансформатора ТРДН-63000/220, для которых \(\textup{К}_\textup{ТР.баз} = 12625\) тыс. руб. Тогда стоимость двух трансформаторов на подстанции:
\[K_\textup{ТР1} = K_\textup{ТР.баз}\cdot k_\textup{зон}\cdot k_\textup{деф}\cdot n_\textup{т} = 12625\cdot 1\cdot 4,25\cdot 2 = 107312,5\; \textup{тыс. руб}\]

РУВН выполнено по схеме «четырехугольник», в которой используется 4 воздушных выключателя базовой укрупненной стоимостью \(\textup{К}_\textup{выкл.220.баз} = 8800\) тыс. руб.

Капиталовложения в РУВН ПС1:
\[\textup{К}_\textup{РУВН1} = \textup{К}_\textup{выкл.220.баз}\cdot k_\textup{зон}\cdot k_\textup{деф}\cdot n_\textup{яч1}^{220} = 8800\cdot 1\cdot 4,25\cdot 4 = 149600\; \textup{тыс. руб.}\]

Суммарное количество ячеек выключателей 10 кВ распределительного устройства низшего напряжения ПС1 было рассчитано в разделе 7: \(n_\textup{яч}^{10} = 30\).

В РУНН 10 кВ используются вакуумные выключатели, для которых:
\[\textup{К}_\textup{выкл.10.баз} = 120\; \textup{тыс. руб}\]

Тогда стоимость распределительного устройства низшего напряжения:
\[\textup{К}_\textup{РУНН1} = \textup{К}_\textup{выкл.10.баз.}\cdot k_\textup{зон}\cdot k_\textup{деф}\cdot n_\textup{яч1}^{10} = 120\cdot 1\cdot 4,25\cdot 30 = 15300\; \textup{тыс. руб.}\]

Суммарные капиталовложения в распределительные устройства ПС1:
\[\textup{К}_{\textup{РУ1}\Sigma} = \textup{К}_\textup{РУВН} + \textup{К}_\textup{РУНН} = 149600 + 15300 = 164900\; \textup{тыс. руб}\]

Определим стоимость дополнительного оборудования. На ПС1 установлены 8 батарей статических конденсаторов. Базовая укрупненная стоимость шунтовой конденсаторной батареи 10 кВ единичной мощностью 1,2 Мвар: \(\textup{К}_\textup{БСК.баз} = 375\) тыс. руб.

Суммарные капиталовложения в дополнительное оборудование ПС1:
\[\textup{К}_\textup{доп1} = \textup{К}_\textup{БСК.баз}\cdot k_\textup{зон}\cdot k_\textup{деф}\cdot N_\textup{БСК} = 375\cdot 1\cdot 4,25\cdot 8 = 12750\; \textup{тыс. руб.}\]

Определим постоянную часть затрат на сооружение ПС1 напряжением 220/10 кВ со схемой РУВН «четырехугольник>>, для которой \(\textup{К}_\textup{пост.220.баз} = 19500\) тыс.руб.

Постоянная часть затрат на сооружение ПС1:
\[\textup{К}_\textup{пост1} = \textup{К}_\textup{пост.220.баз}\cdot k_\textup{зон}\cdot k_\textup{деф} = 19500\cdot 1\cdot 4,25 = 82875\; \textup{тыс. руб.}\]

Тогда суммарные капиталовложения на сооружение ПС1:
\[\textup{К}_\textup{ПС1} = \textup{К}_{\textup{ТР1}\Sigma} + \textup{К}_{\textup{РУ1}\Sigma} + \textup{К}_\textup{доп1} + \textup{К}_\textup{пост1} = 107312,5 + 164900 + 12750 + 82875 = 367837,5\; \textup{тыс. руб.}\]

Для остальных подстанций расчет аналогичный, результаты расчетов сведен в таблицу \ref{tab:расчет_капиталовложений_пс}.

\begin{table}[H]
	\small
	\caption{Капиталовложения на сооружение подстанций проектируемой сети.}
	\begin{tabularx}{\linewidth}{|Z|Z|Z|Z|Z|Z|}
		\hline
		№ ПС                                           & 3 (АТ)   & 1 & 2 & 4 & 5 \\ \hline
		\(\textup{К}_\textup{ТР.баз}\), тыс. руб       & 15525    &   &   &   &   \\ \hline
		\(\textup{К}_{\textup{ТР}\Sigma}\), тыс. руб   & 131962,5 &   &   &   &   \\ \hline
		\(\textup{К}_\textup{выкл.220.баз}\), тыс. руб & 8800     &   &   &   &   \\ \hline
		\(n_\textup{яч}^{220}\)                        & 4        &   &   &   &   \\ \hline
		\(\textup{К}_\textup{РУ.220}\), тыс. руб       & 149600   &   &   &   &   \\ \hline
		\(\textup{К}_\textup{выкл.110.баз}\), тыс. руб & 4150     &   &   &   &   \\ \hline
		\(n_\textup{яч}^{110}\)                        & 7        &   &   &   &   \\ \hline
		\(\textup{К}_\textup{РУ.110}\), тыс. руб       & 123462,5 &   &   &   &   \\ \hline
		\(\textup{К}_\textup{выкл.10.баз}\), тыс. руб  & 120      &   &   &   &   \\ \hline
		\(n_\textup{яч}^{10}\)                         & 15       &   &   &   &   \\ \hline
		\(\textup{К}_\textup{РУ.10}\), тыс. руб        & 7650     &   &   &   &   \\ \hline
		\(\textup{К}_{\textup{РУ}\Sigma}\), тыс. руб   & 280712,5 &   &   &   &   \\ \hline
		\(\textup{К}_\textup{БСК.баз}\), тыс. руб      & 375      &   &   &   &   \\ \hline
		\(n_\textup{БСК}\)                             & 4        &   &   &   &   \\ \hline
		\(\textup{К}_\textup{БСК}\), тыс. руб          & 6375     &   &   &   &   \\ \hline
		\(\textup{К}_\textup{ЛРТ.баз}\), тыс. руб      & 3750     &   &   &   &   \\ \hline
		\(n_\textup{ЛРТ}\)                             & 2        &   &   &   &   \\ \hline
		\(\textup{К}_\textup{ЛРТ}\), тыс. руб          & 31875    &   &   &   &   \\ \hline
		\(\textup{К}_\textup{пост.баз}\), тыс. руб     & 30000    &   &   &   &   \\ \hline
		\(\textup{К}_\textup{пост}\), тыс. руб         & 127500   &   &   &   &   \\ \hline
		\(\textup{К}_\textup{ПС}\), тыс. руб           & 578425   &   &   &   &   \\ \hline
	\end{tabularx}
\end{table}


%%% Local Variables:
%%% mode: latex
%%% TeX-master: "rpz"
%%% End: